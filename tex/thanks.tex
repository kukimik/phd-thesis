\chapter{Podziękowania}

Chciałbym podziękować Bogu za to, że żyję, za Jego Miłość i~opiekę nade mną i~światem, za wszystko i~wszystkich wokół. Dziękuję mojej mamie Elżbiecie oraz mojemu zmarłemu niedawno tacie Bronisławowi za to, że jestem, oraz za to, jaki dzięki ich miłości, trosce i~trudowi jestem. Mojej żonie Natalii dziękuję gorąco za wytrwanie ze mną w~różnorakich trudnościach oraz jej stałą miłość, wierność i~uczciwość; dziękuję również za cierpliwość do mnie jej licznej rodzinie i~mojemu kochanemu rodzeństwu.

Dziękuję mojemu promotorowi, prof. dr. hab. Markowi Golasińskiemu, za sprawowanie nie zawsze łatwej opieki naukowej nad moją osobą, liczne uwagi, komentarze, wskazówki i~rozmowy, dotyczące nie tylko matematyki. Jestem również wdzięczny Zbigniewowi Błaszczykowi oraz Jakubowi Kiszkielowi za wspólną pracę na seminarium doktoranckim. Dziękuję prof. dr. hab. Marianowi Mrozkowi za opiekę naukową podczas półrocznego stażu na Uniwersytecie Jagiellońskim.

Nie sposób wymienić z~nazwiska wszystkich przyjaciół, członków rodziny, nauczycieli, kolegów i znajomych, a~także osób nieznajomych, dzięki którym ta rozprawa mogła powstać, wobec czego nie będę próbował tego robić. Z~różnych względów pragnę jednak uczynić wyjątek dla dr. Adama Hajduka, prof. dr. hab. Stanisława Kasjana oraz mojego opiekuna w~czasie studiów magisterskich, dr. hab. Dariusza Miklaszewskiego.

Nie do przecenienia jest rola pań pracujących w~administracji oraz w~bibliotece naszego Wydziału, jak również osób, które czuwają nad tym, aby mógł on sprawnie funkcjonować, w~tym oczywiście byłych i~obecnych władz Wydziału. Dziękuję im za stworzenie tak przyjaznego środowiska pracy.

Przyjemnością była dla mnie współpraca z~profesorami Berndem Schr{\"o}derem oraz Aleksandrem Rutkowskim, wyniki której nie zostały wprawdzie zawarte w~rozprawie, ale która pośrednio wpłynęła na jej kształt. Profesorowi Kennethowi Baclawskiemu dziękuję za udostępnienie notatek \cite{Baclawski} oraz pozwolenie na zamieszczenie w rozprawie uogólnień jego niepublikowanych wyników.

Dziękuję K.~Adiprasito, B.~Benedettiemu, B.~Hughesowi, L.~Górniewiczowi, V.~Okhezinowi, D.~Osajdzie, A.~Ranickiemu, K.~Rykaczewskiemu, jak również uczestnikom internetowych forów dyskusyjnych MathOverflow, Mathematics Stack Exchange oraz~\TeX~--~\LaTeX~Stack Exchange, spośród których chciałbym wymienić  H.~Brandsma, N.~Diepeveen, S.~Melikhova, V.~Nanda, D.~Panova, N.~Stricklanda, C.~Westerlanda, E.~Wofseya oraz R.~Woodroofe'a, za odpowiedzi na pytania związane (w~różnym stopniu) z~treścią i~formą rozprawy.

Jestem wdzięczny za możliwość uczestnictwa w~trakcie studiów doktoranckich w~kilku konferencjach naukowych ich organizatorom oraz instytucjom finansującym moje wyjazdy. Szczególnie cenny był dla mnie udział w~szkole letniej Discrete Morse Theory and Commutative Algebra, która odbyła się w~2012 roku w~Instytucie \mbox{Mittag-Lefflera} w~Szwecji.

Wreszcie, podziękowania składam organizatorom Środowiskowych Studiów Doktoranckich z~Nauk Matematycznych oraz podatnikom, dzięki którym otrzymywałem pokaźne stypendium.\\

\hfill Michał Jerzy Kukieła
