\chapter*{Spis problemów otwartych}\addcontentsline{toc}{chapter}{Spis problemów otwartych}
\chaptermark{Spis problemów otwartych}
W~niniejszym dodatku zebrane zostały postawione w~rozprawie problemy otwarte, wraz z~odnośnikami do numerów stron, na których zostały one sformułowane.

\begin{description}

\item[\ref{prob1} \normalfont (s. \pageref{prob1}):] Czy $\mathcal{R}$-rozbieralność $X$~do~$A$, gdzie $\mathcal{R}\in\{\mathcal{I},\mathcal{C}\}$, implikuje \mbox{$\mathcal{R}$-korozbieralność} $X$~z~$A$~dla dowolnej pary przestrzeni Aleksandrowa $(X,A)$?

\item[\ref{prob2} \normalfont (s. \pageref{prob2}):] Niech $X$~będzie przestrzenią Aleksandrowa nie zawierającą nieskończonych łańcuchów i~nieskończonych palisad, zaś $A$~jej podzbiorem. Czy $X\dism A$ wtedy i~tylko wtedy, gdy $A\codism X$?

\item[\ref{prob3} \normalfont (s. \pageref{prob3}):]Czy dla odwzorowań $f,g\colon X\to Y$ przestrzeni Aleksandrowa  oraz odwzorowań symplicjalnych $\phi, \psi\colon K\to L$ kompleksów symplicjalnych zachodzi któraś z~poniższych implikacji:
\begin{compactitem}
\item[---] jeżeli $\phi\stackrel{\infty}{\sim}\psi$, to $\mP(\phi)\simeq\mP(\psi)$;
\item[---] jeżeli $f\simeq g$, to $\mK(f)\stackrel{\infty}{\sim}\mK(g)$;
\item[---] jeżeli $\phi\stackrel{\infty}{\sim}\psi$, to $|\phi|\simeq |\psi|$?
\end{compactitem}
Co jeśli założymy dodatkowo, że $X,Y$ nie zawierają promieni (albo, ogólniej, nieskończonych łańcuchów) oraz $K,L$ nie zawierają promieni (albo nieskończonych sympleksów)?

\item[\ref{prob4} \normalfont (s. \pageref{prob4}):] Czy stwierdzenie \ref{stw-rozbieralany_kompleks_wtw_gdy_podzial_barycentryczny} uogólnia się na kompleksy symplicjalne bez nieskończonych sympleksów i~przestrzenie Aleksandrowa bez nieskończonych łańcuchów, bądź na dowolne kompleksy symplicjalne i~przestrzenie Aleksandrowa?

\item[\ref{prob5} \normalfont (s. \pageref{prob5}):] Niech $P$~będzie częściowym porządkiem z~gradacją oraz~zadanym skojarzeniem Morse'a $M'$~takim, że zbiór $\mathcal{R}_{M'}(P)$ jest nieskończony. 
Przy jakich założeniach o~$P$~oraz $M'$~możliwe jest uzyskanie na $P$~skojarzenia Morse'a $M$~bez promieni malejących i~o~tej własności, że \[\mathcal{C}_{M}(P)=\mathcal{C}_{M'}(P)\cup \left\{c_{[r]}:[r]\in\mathcal{R}_{M'}(P)\right\},\] gdzie $c_{[r]}\in P\smallsetminus \mathcal{C}_{M'}(P)$, $c_{[r]}\not=c_{[r']}$ dla $[r]\not=[r']$ oraz $\Ht\bigl(c_{[r]}\bigr)=\Ht(r)$ dla wszystkich $[r],[r']\in\mcR_{M'}$? 

\item[\ref{prob6} \normalfont (s. \pageref{prob6}):] Czy jeśli $L$~jest dowolną kratą z~zerem i~jedynką, bez mocnych dopełnień, to $\mK\left(\check{L}\right)\infcoll *$?

\item[\ref{prob7} \normalfont (s. \pageref{prob7}):] Scharakteryzować inne klasy skojarzeń Morse'a niż skojarzenia Morse'a bez promieni malejących za pomocą uogólnionych dyskretnych funkcji Morse'a o~odpowiednich kodziedzinach.

\item[\ref{PROBLEM-twierdzenie-lefschetza-o-punkcie-lub-koncu-stalym} \normalfont (s. \pageref{PROBLEM-twierdzenie-lefschetza-o-punkcie-lub-koncu-stalym}):] Niech $X$ będzie lokalnie zwartym ANR-em o~homologiach skończonego typu, zaś $f\colon X\to X$ właściwym odwzorowaniem. Czy jeśli \mbox{$\lambda(f)\not=0$}, to $\Fix(f)\cup\FixEnd(f)\not=\emptyset$?

Ogólniej, czy jeśli $X$~jest lokalnie zwartym ANR-em, natomiast~\mbox{$f\colon X\to X$} jest właściwym, dopuszczalnym odwzorowaniem oraz $\Lambda(f)\not=0$, to $\Fix(f)\cup\FixEnd(f)\not=\emptyset$?

\item[\ref{PROBLEM-twierdzenie-o-indeksie} \normalfont (s. \pageref{PROBLEM-twierdzenie-o-indeksie}):] Niech $X$ będzie spójnym, lokalnie zwartym, metrycznym \mbox{ANR-em}, zaś $f\colon X\to X$ niech będzie właściwym, dopuszczalnym odwzorowaniem. Czy jeśli $\FixEnd(f)=\emptyset$, to $\Lambda(f)=\Ind(f)$?

\item[\ref{PROBLEM-sciagalny-ma-fpep} \normalfont (s. \pageref{PROBLEM-sciagalny-ma-fpep}):] Czy każdy ściągalny, lokalnie zwarty, metryczny ANR ma własność punktu lub końca stałego?

\item[\ref{problem-twierdzenie_lefschetza_dla_przestrzeni_bez_promieni} \normalfont (s. \pageref{problem-twierdzenie_lefschetza_dla_przestrzeni_bez_promieni}):] Załóżmy, że $K$~jest kompleksem symplicjalnym bez promieni, zaś~$f\colon |K|\to |K|$ jest ciągłym, dopuszczalnym odwzorowaniem. Czy jeśli $\Lambda(f)\not=0$, to $\Fix(f)\not=\emptyset$? Co jeżeli założymy dodatkowo, że $f$~jest realizacją geometryczną odwzorowania symplicjalnego $K\to K$?

\item[\ref{problem-acykliczny_to_fpp} \normalfont (s. \pageref{problem-acykliczny_to_fpp}):] Czy jeżeli $K$~jest acyklicznym kompleksem symplicjalnym bez promieni, to $|K|\in\FPP$ lub $K$~ma własność sympleksu stałego?

%\item[\ref{prob8} \normalfont (s. \pageref{prob8}):] Zdefiniować kompleksy symplicjalne ,,lokalnie bez promieni''. Podać twierdzenie o~istnieniu punktu lub końca stałego przy założeniu, że $K$~jest ściągalnym kompleksem symplicjalnym ,,lokalnie bez promieni'' spełniającym odpowiednik warunku oswojoności do wewnątrz, zaś $f\colon |K|\to |K|$ jest odwzorowaniem ciągłym i~takim, że zbiór $f^{-1}(A)$~jest bez promieni dla każdego podzbioru $A\subseteq |K|$ bez promieni.

\item[\ref{prob9} \normalfont (s. \pageref{prob9}):] Niech $P$~będzie lokalnie skończonym częściowym porządkiem z~zadanym działaniem skończonej grupy $\Gamma$. Czy jeśli porządek $P$~jest \mbox{$\mathcal{C}$-korozbieralny}, to \mbox{$\mathcal{C}$-korozbieralny} jest również zbiór punktów stałych $P^\Gamma$~działania $\Gamma$~na~$P$?

\item[\ref{prob10} \normalfont (s. \pageref{prob10}):] Czy jeżeli $P$~jest częściowym porządkiem takim, że $*\codism P$, zaś $f\colon P\to P$ jest zachowującym porządek odwzorowaniem o~tej własności, że $\Fix(f)\not=\emptyset$, to $*\codism \Fix(f)$? Co jeśli o~zbiorze częściowo uporządkowanym $P$~założymy dodatkowo, że jest łańcuchowo zupełny lub nie zawiera nieskończonych łańcuchów?

\item[\ref{problem-connected_collapsibility} \normalfont (s. \pageref{problem-connected_collapsibility}):] Czy pojęcie \textit{connected collapsibility} można przenieść na porządki bez promieni i~udowodnić, że zbiór punktów stałych zachowującego porządek odwzorowania określonego na \textit{connectedly collapsible} porządku bez promieni jest spójny?

\item[\ref{prob100} \normalfont (s. \pageref{prob100}):] Czy są prawdziwe odpowiedniki twierdzeń Segeva \ref{tw-segeva_o_zgniatalnych}, \ref{tw-segeva_o_acyklicznych}, stwierdzenia \ref{stw-baclawski_dla_wymiarow_2_i_3} i~wniosku \ref{wn-baclawski_wym_2_3_dla_kompleksow_symplicjalnych} dla kompleksów symplicjalnych bez promieni i~częściowych porządków bez promieni?

\item[\ref{prob101} \normalfont (s. \pageref{prob101}):] Niech $P$~będzie częściowym porządkiem bez promieni, zaś $f\colon P\to P$ zachowującym porządek odwzorowaniem. Czy $\chi(\Fix(f))=1$, o~ile kompleks $\mK(P)$ jest zgniatalny (ściągalny, acykliczny)?
\end{description}
\newpage\thispagestyle{empty}
