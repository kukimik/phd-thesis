\chapter{Dyskretna teoria Morse'a}\label{chapter3}
\begin{center}
\begin{minipage}{14cm}
{\small  W bieżącym rozdziale dowodzimy uogólnienia głównego twierdzenia dyskretnej teorii Morse'a, pozwalającego w~pewnych sytuacjach zbudować CW kompleks homotopijnie równoważny danemu niezwartemu, regularnemu CW kompleksowi, składający się z~tzw.~komórek krytycznych. Podajemy algebraiczną wersję tego twierdzenia. Jako wniosek otrzymujemy dyskretne nierówności Morse'a. Wyniki te formułujemy przy użyciu pojęcia skojarzenia Morse'a.\\

Podajemy także lematy pozwalające modyfikować skojarzenie Morse'a celem usunięcia tzw.~promieni malejących oraz kryteria pozwalające rozpoznać CW kompleksy z~kołnierzykiem na zewnątrz i~do wewnątrz. Badamy związki skojarzeń Morse'a z~tzw.~uogólnionymi dyskretnymi funkcjami Morse'a. Język dyskretnej teorii Morse'a stosujemy do opisu własności %$\CAT(0)$ kompleksów symplicjalnych oraz 
ściętych krat bez dopełnień.\\

Rozdział oparty jest częściowo na artykule autora rozprawy \cite{Kukiela13} i~uogólnia wyniki zawarte w~pracach kilku innych autorów \cite{Adiprasito13,Ayala07,Ayala09,Baclawski12,Forman98,Forman02}.}
\end{minipage}\\[1.7cm]
\end{center}

W~rozdziale \ref{chapter2} rozważaliśmy metodę upraszczania struktury kompleksu symplicjalnego przy zachowaniu kombinatorycznej własności zwanej mocnym typem homotopijnym. W~bieżącym rozdziale zamujemy się w~pewnym stopniu podobną techniką, inspirowaną teorią Morse'a na rozmaitościach gładkich, która pozwala zredukować liczbę komórek CW kompleksu przy zachowaniu jego typu homotopijnego.

Przypomnijmy, że klasyczna teoria Morse'a \cite{Banyaga04, Milnor63} wiąże punkty krytyczne gładkiej funkcji $M\to \mathbb{R}$, określonej na gładkiej rozmaitości $M$, z~topologią tej rozmaitości. Została ona zapoczątkowana w~pierwszej połowie XX wieku przez Morse'a \cite{Morse25} i~stanowi użyteczne narzędzie, wykorzystywane w~geometrii, topologii i~fizyce teoretycznej. Jeden z~podstawowych wyników tej teorii stwierdza istnienie homotopijnej równoważności pomiędzy rozmaitością $M$~a~pewnym CW kompleksem, którego $d$-wymiarowe komórki są we wzajemnie jednoznacznej odpowiedniości z~punktami krytycznymi o~indeksie $d$~funkcji Morse'a na $M$, tj.~gładkiej funkcji $M\to \mR$, której wszystkie punkty krytyczne są niezdegenerowane.

W~latach 90-tych~ubiegłego wieku Forman \cite{Forman98} zaproponował dyskretny odpowiednik tej teorii, w~którym rozmaitości zastąpione zostały zwartymi CW~kompleksami, zaś gładkie funkcje Morse'a odwzorowaniami ze zbioru komórek CW~kompleksu w~zbiór liczb rzeczywistych o~odpowiednich własnościach. Ta dyskretna teoria Morse'a znalazła liczne zastosowania, przykładowo w~kombinatoryce \cite{Jonsson08}, topologii obliczeniowej \cite{Harker13,Sergeraert}, algebrze przemiennej \cite{Jollenbeck05}, analizie obrazów \cite{Robins11}, fizyce \cite{Engstrom09}, teorii grup \cite{Farley05}. Można przypuszczać, iż powodem jest jej prostota, łatwość implementacji komputerowej (teoria ta dotyczy wszak obiektów skończonych), a~także możliwość uzyskania przy jej użyciu rezultatów porównywalnych z~osiąganymi przez klasyczną teorię Morse'a~\cite{Benedetti13,Gallais10}.

Głównym twierdzeniem dyskretnej teorii Morse'a bywa nazywany wynik Formana \cite[Corollary 3.5]{Forman98} pozwalający dla regularnego, zwartego CW kompleksu $X$~z~zadaną dyskretną funkcją Morse'a znaleźć homotopijnie równoważny mu CW kompleks, którego \mbox{$d$-wymiarowe} komórki są, dla każdego $d\in\mN$, we wzajemnie jednoznacznej odpowiedniości z~tzw.~\mbox{$d$-wymiarowymi} komórkami krytycznymi kompleksu $X$ względem tej dyskretnej funkcji Morse'a. Jako wniosek z~tego twierdzenia Forman \cite[Corollaries 3.6, 3.7]{Forman98} otrzymał dyskretne nierówności Morse'a, wiążące liczby Bettiego CW kompleksu z~liczbą jego komórek krytycznych względem dyskretnej funkcji Morse'a zadanej na tym CW kompleksie.

Dyskretna teoria Morse'a doczekała się kilku uogólnień i~alternatywnych sformułowań. Przykładowo: Forman \cite{Forman98a,Forman02a,Forman02b} rozszerzył swoją pierwotną teorię; Freij, J\"ollenbeck, Kozlov i~Sk\"oldberg \cite{Freij07,Jollenbeck05,Kozlov05,Skoldberg06} podali jej wersje algebraiczne; % Jonsson \cite{Jonsson11} uogólnił ją na $n$-kompleksy Kapranova;
Minian \cite{Minian12} zastąpił regularne CW kompleksy pewną klasą częściowych porządków, zawierającą klasę porządków ścian regularnych CW kompleksów; Chariemu \cite{Chari00} przypisuje się sformułowanie głównych wyników teorii w~języku skojarzeń w~diagramie Hassego uporządkowanego zbioru ścian rozważanego CW~kompleksu.

Stosunkowo niewielka część badań związanych z~dyskretną teorią Morse'a dotyczy niezwartych CW kompleksów. Forman \cite{Forman02} postawił pytanie o~uogólnienie swoich wyników na niezwarte CW kompleksy. Zaproponował także częściowe rozwiązanie tego zagadnienia przy pomocy tzw.~właściwych dyskretnych funkcji Morse'a, które jednak uznał za niesatysfakcjonujące. W~serii prac \cite{Ayala07,Ayala08,Ayala09a,Ayala09,Ayala10,Ayala11} Ayala i~jego współpracownicy uzyskali między innymi nierówności Morse'a dla pewnej klasy dyskretnych funkcji Morse'a na niektórych lokalnie skończonych, co najwyżej $2$-wymiarowych kompleksach symplicjalnych. Prace dotyczące algebraicznej wersji teorii \cite{Freij07,Jollenbeck05,Kozlov05,Skoldberg06} omawiają przypadki nieskończenie generowanych kompleksów łańcuchowych i~podają dla nich, przy pewnych założeniach, algebraiczną wersję głównego twierdzenia dyskretnej teorii Morse'a. W~trakcie przygotowywania rozprawy autor dowiedział się o~artykule K.S.~Browna \cite{Brown92} z~1992~roku, w~którym udowodnione (i~zastosowane do upraszczania struktury przestrzeni klasyfikujących grup i~monoidów) zostało główne twierdzenie dyskretnej teorii Morse'a dla nieskończonych zbiorów symplicjalnych. (Wcześniej podobną technikę wykorzystali Brown i~Geoghegan \cite{Brown84}.) Wyniki Browna poprzedziły odkrycia Formana; być może ze względu na mniej ,,chwytliwą'' terminologię nie są szerzej znane. Również stosunkowo późno autor poznał wykład P.~Orlika i~V.~Welkera \cite{Orlik07}, zawierający dowód głównego twierdzenia dyskretnej teorii Morse'a dla~niezwartych CW kompleksów. Praca autora \cite{Kukiela13}, na~której w~dużej mierze opiera się bieżący rozdział, od wyników Browna \cite{Brown92} oraz Orlika i~Welkera \cite{Orlik07} różni się przede wszystkim nieco słabszymi założeniami, przy których dowodzi się w~niej głównego twierdzenia dyskretnej teorii Morse'a, jak również rozpatrywaną klasą obiektów: obok CW kompleksów autor dowodzi twierdzenia dla wprowadzonych przez Miniana \cite{Minian12} tzw.~dopuszczalnych nieskończonych częściowych porządków. (Wspomnieć należy, że dyskretnej teorii Morse'a używali do badania niezwartych przestrzeni także Mathai i~Yates \cite{Mathai99}, lecz w~innym duchu, niż prezentujemy to w~tym rozdziale.) 

Najważniejszy wynik rozdziału stanowi twierdzenie \ref{maintw}, tj.~główne twierdzenie dyskretnej teorii Morse'a dla tzw.~skojarzeń Morse'a bez promieni malejących na \mbox{h-regularnych} częściowych porządkach \cite{Minian12}, czyli obiektach ogólniejszych niż regularne CW kompleksy. (Twierdzenie to, udowodnione w~publikacji autora \cite{Kukiela13}, jest podobne do wyników Browna \cite[Proposition 1]{Brown92} oraz Orlika i~Welkera \cite[Theorem 4.2.14]{Orlik07}). Wniosek \ref{maincor}, uzyskany z~niego przy użyciu techniki ,,odwracania promieni'' (stwierdzenie \ref{reversing_stw}), dotyczy skojarzeń Morse'a o~skończonej liczbie klas równoważności promieni malejących i~jest dzięki temu ogólniejszy niż twierdzenia Browna \cite{Brown92} oraz Orlika i~Welkera \cite{Orlik07}; wynikają z~niego również główne wyniki prac \mbox{Ayali} i~współpracowników~\cite{Ayala07,Ayala09}. Najistotniejszymi nowościami rozdziału w~stosunku do wspomnianej publikacji autora \cite{Kukiela13} są, obok bardziej eleganckich i~dopracowanych dowodów, twierdzenia \ref{tw-baclawskiego-kraty-bez-dopelnien-sa-zgniatalne} oraz \ref{tw-baclawskiego-o-kratach-bez-mocnych-dopelnien}, dotyczące \mbox{$\infty$-zgniatalności} kompleksów symplicjalnych stowarzyszonych ze ściętymi kratami bez dopełnień. Uogólniają one rezultaty uzyskane przez Baclawskiego \cite{Baclawski12} oraz Kozlova \cite{Kozlov98}. W~ich dowodzie wykorzystuje się powiązania pomiędzy dyskretną teorią Morse'a a~(ko)rozbieralnością. Ciekawe i~warte rozwinięcia wydają się również wyniki wiążące dyskretną teorię Morse'a z~topologią w~nieskończoności (stwierdzenia \ref{stw-dyskretna-teoria-morsea-oswojone-do-wewnatrz}, \ref{stw-dyskretna-teoria-morsea-oswojone-na-zewnatrz}).

Struktura rozdziału jest następująca. W~podrozdziale \ref{chap3-sec1}, mającym charakter wprowadzenia i~nie zawierającym nowych wyników, przypominamy podstawowe twierdzenia gładkiej oraz dyskretnej teorii Morse'a i~dokonujemy krótkiego ich porównania. Następnie w~podrodziale \ref{rayless} wprowadzamy kluczowe dla dalszej części rozdziału pojęcia skojarzenia Morse'a na częściowym porządku i~promienia malejącego. Podrozdział \ref{reversing} zawiera ważne wyniki pozwalające modyfikować dane skojarzenie Morse'a celem ,,usunięcia'' indukowanych przez nie promieni malejących. Uogólnienie głównego twierdzenia dyskretnej teorii Morse'a na obiekty niezwarte (nieskończone) stanowi temat podrodziału \ref{homotopical}. W~podrozdziale \ref{homological} czynimy uwagi odnośnie algebraicznej wersji głównego twierdzenia dyskretnej teorii Morse'a. Podrozdział \ref{sec-infty-zgniatalnosc} dotyczy pojęcia $\infty$-zgniatalności, stanowiącego uogólnienie zgniatalności regularnego, zwartego CW~kompleksu. Przedstawiamy jego związki z~(ko)rozbieralnością, które następnie wykorzystujemy dowodząc $\infty$-zgniatalności kompleksów symplicjalnych stowarzyszonych z~nieskończonymi kratami bez dopełnień. W~podrozdziale \ref{topologia-w-nsk} stosujemy dyskretną teorię Morse'a do opisu własności topologii w~nieskończoności. Rozdział kończymy znajdującymi się w~podrozdziale \ref{skojarzenia-a-funkcje} uwagami o~związku pojęcia skojarzenia Morse'a z~uogólnionymi dyskretnymi funkcjami Morse'a.

%==============================================================
%==============================================================
%==============================================================

\section{Klasyczne teorie Morse'a: gładka oraz dyskretna}\label{chap3-sec1}
W~tym podrozdziale przypominamy, w~oparciu o~książkę Milnora \cite{Milnor63}, prace Formana \cite{Forman98} i~J{\"o}llenbecka \cite{Jollenbeck05}, niektóre podstawowe twierdzenia gładkiej teorii Morse'a oraz ich dyskretne odpowiedniki, a~następnie, korzystając z~wyników Benedettiego \cite{Benedetti13}, krótko porównujemy te dwa podejścia.

\subsection{Gładka teoria Morse'a}
Teoria Morse'a, wiążąca punkty krytyczne odwzorowania gładkiej rozmaitości w~zbiór liczb rzeczywistych z~topologią tej rozmaitości, zapoczątkowana badaniami Morse'a \cite{Morse25}, stanowi ważne narzędzie topologii geometrycznej i~różniczkowej.  Klasyczne wprowadzenie do tej teorii, na którym opiera się niniejsza sekcja, stanowi książka Milnora \cite{Milnor63}; interesujące wydają się również nowsze wydawnictwa \cite{Banyaga04,Nicolaescu11}. Poniżej ograniczamy się do podania tych jej wyników, które są kluczowe z~punktu widzenia niniejszej rozprawy.

Zakładamy znajomość podstawowych pojęć topologii różniczkowej. \textbf{Przez gładkie rozmaitości i~odwzorowania rozumiemy rozmaitości (bez brzegu, o~ile nie zaznaczono inaczej) i~odwzorowania klasy $C^\infty$.} 

Niech $M$~będzie gładką, $n$-wymiarową rozmaitością, zaś $f\colon M\to \mathbb{R}$ gładkim przekształceniem. Mówimy, że $p\in M$ jest \textit{punktem krytycznym}\index{punkt!krytyczny|see{krytyczność}}\index{krytycznoszzzczzz wzglezzzdem@krytyczność względem!funkcji glzzzadkiej@funkcji gładkiej} funkcji $f$, jeżeli homomorfizm $T_p f\colon T_p M\to T_{f(p)} \mR$ przestrzeni stycznych jest zerowy.

Jeśli na pewnym otoczeniu $U$~punktu $p\in M$ zadamy lokalne współrzędne $\left(x^1,\ldots,x^n\right)$, to krytyczność punktu $p$~jest równoważna następującym równościom:
\[\frac{\partial f}{\partial x^1}(p)=\ldots=\frac{\partial f}{\partial x^n}(p)=0.\]

Dla $a\in\mR$ wprowadźmy oznaczenie $M(a)=\{x\in M:f(x)\leq a\}$. Jeśli $a$~nie jest wartością krytyczną funkcji $f$ (tzn.~zbiór $f^{-1}(a)$ nie zawiera punktów krytycznych), to $M(a)\subseteq M$ jest gładką rozmaitością z~brzegiem. Ponadto brzeg rozmaitości $M(a)$, równy $f^{-1}(a)$, jest gładką podrozmaitością $M$.

Punkt krytyczny $p\in M$ nazywamy \textit{niezdegenerowanym}, gdy macierz \[\left[\frac{\partial^2 f}{\partial x^i \partial x^j}(p) \right]_{i,j=1,\ldots,n}\]
jest nieosobliwa. (Definicja nie zależy od wyboru lokalnych współrzędnych.) 

\begin{lem}[Morse'a, {\cite[Lemma 2.2]{Milnor63}}]
Niech $M$~będzie gładką rozmaitością, zaś $p\in M$~niezdegenerowanym punktem krytycznym gładkiej funkcji $f\colon M\to \mR$. Istnieją wówczas otwarte otoczenie $U$~punktu $p$~oraz lokalny układ współrzędnych $\left(y^1,\ldots,y^n\right)$ na $U$ takie, że $y^j(p)=0$ dla wszystkich $j=1,\ldots,n$ oraz \[f=f(p)-\bigl(y^1\bigr)^2-\ldots -\bigl(y^i\bigr)^2+\bigl(y^{i+1}\bigr)^2+\ldots+\bigl(y^n\bigr)^2\] na $U$~dla pewnej liczby $0\leq i\leq n$.
\end{lem}

Liczbę $i$~z~powyższego lematu nazywamy \textit{indeksem}\index{indeks!niezdegenerowanego punktu krytycznego} niezdegenerowanego punktu krytycznego $p$.

\begin{wn}[{\cite[Corollary 2.3]{Milnor63}}]\label{wn312}
Niezdegenerowane punkty krytyczne gładkiego odwzorowania gładkiej rozmaitości w~zbiór liczb rzeczywistych są izolowane. 
\end{wn}

Punkty krytyczne gładkiej funkcji $f\colon M\to \mR$ mają związek z~typem homotopijnym rozmaitości $M$.

\begin{tw}[{\cite[Theorem 3.1]{Milnor63}}]\label{tw313}
Niech $M$~będzie gładką rozmaitością, zaś $f\colon M\to \mathbb{R}$ gładkim przekształceniem. Jeśli $a,b\in\mathbb{R}$, $a<b$ oraz zbiór $f^{-1}([a,b])$ jest zwarty i~nie zawiera punktów krytycznych funkcji $f$, to rozmaitość z~brzegiem $M(a)$ jest dyfeomorficzna z~rozmaitością z~brzegiem $M(b)$, a~ponadto jest jej mocnym retraktem deformacyjnym.
\end{tw}

\begin{tw}[{\cite[Theorem 3.2]{Milnor63}}]\label{tw314}
Niech $M$~będzie gładką rozmaitością, zaś $p\in M$ niezdegenerowanym punktem krytycznym o~indeksie $i$~gładkiej funkcji $f\colon M\to \mathbb{R}$. Jeżeli dla pewnego $\epsilon>0$ zbiór $f^{-1}([f(p)-\epsilon,f(p)+\epsilon])$ jest zwarty i~nie zawiera punktów krytycznych innych niż $p$, to dla wszystkich odpowiednio małych $\epsilon>0$ rozmaitość $M(f(p)+\epsilon)$ ma typ homotopijny przestrzeni powstałej przez doklejenie $i$-wymiarowej komórki do rozmaitości $M(f(p)-\epsilon)$.
\end{tw}

Jeżeli $M$~jest gładką rozmaitością, zaś $f\colon M\to \mathbb{R}$ gładkim odwzorowaniem, to $f$~nazywamy \textit{funkcją Morse'a}\index{funkcja!Morse'a}, o~ile wszystkie punkty krytyczne tej funkcji są niezdegenerowane. Dla funkcji Morse'a $f\colon M\to \mathbb{R}$ przez $m^f_i(M)$ oznaczamy liczbę niezdegenerowanych punktów krytycznych funkcji $f$~o~indeksie $i$. Oczywiście $m^f_i(M)=0$ dla $i>\dim(M)$.

W~oparciu o~wniosek \ref{wn312} oraz twierdzenia \ref{tw313}, \ref{tw314} udowodnić można następujący wynik.

\begin{tw}[{\cite[Theorem 3.5]{Milnor63}}]\label{tw-main_thm_morse_theory}
Jeżeli $f\colon M\to \mathbb{R}$ jest określoną na gładkiej rozmaitości $M$~funkcją Morse'a oraz dla każdego $a\in\mR$ zbiór $M(a)\subseteq M$ jest zwarty, to $M$~ma typ homotopijny CW kompleksu, który dla każdego $i\in\mN$ ma dokładnie $m_i^f(M)$  komórek \mbox{$i$-wymiarowych}.
\end{tw}

\begin{wn}[Nierówności Morse'a, {\cite[Section I.5]{Milnor63}}]\label{wn-nier_morsea}
Niech $M$~będzie gładką, zwartą rozmaitością, zaś~$f\colon M\to \mR$ funkcją Morse'a. Dla każdego $n\in\mN$ zachodzą nierówności
\[\sum_{i=0}^{n}(-1)^{n-i}m_i^f(M)\geq \sum_{i=0}^{n}(-1)^{n-i}\beta_i(M)\]
oraz
\[m_n^f(M)\geq \beta_n(M).\] Ponadto charakterystyka Eulera rozmaitości $M$~wyraża się wzorem \[\chi(M)=\sum_{i=0}^{\dim(M)}(-1)^i m_i^f(M).\]
\end{wn} 

Określić można tzw.~\textit{homologie Morse'a}\index{homologie!Morse'a} gładkiej, zwartej rozmaitości, izomorficzne jej homologiom singularnym, a~będące homologiami pewnego wolnego kompleksu łańcuchowego (występującego w~różnych wariantach, pochodzących od  Morse'a, Thoma, Smale'a, Milnora, Wittena, Floera), generatory którego utożsamiać można z~punktami krytycznymi funkcji Morse'a zadanej na tej rozmaitości. Bardziej szczegółowe omówienie  tego typu konstrukcji wymagałoby znacznego odbiegnięcia od zasadniczej tematyki rozprawy. Zainteresowanego Czytelnika odsyłamy do literatury \cite{Banyaga04,Nicolaescu11}.


%------------------------------------------------------------------
%------------------------------------------------------------------
%------------------------------------------------------------------


\subsection{Dyskretna teoria Morse'a dla skończonych CW kompleksów}\label{subsec-forman_classical_discrete_morse}
Dyskretna teoria Morse'a wprowadzona została jako kombinatoryczny odpowiednik klasycznej teorii Morse'a przez Robina Formana \cite{Forman98}, na artykule którego w~znacznym stopniu opiera się bieżąca sekcja. Dobre wprowadzenie do tej teorii stanowią również późniejsza praca Formana \cite{Forman02}, a~także inne opracowania, np.~\cite{Blanchet05,Kozlov08,Orlik07}. (Kilku autorów niezależnie uzyskało zbliżone do Formana wyniki \cite{Brown92,Jones88,Bestvina08,Bestvina97}; w~tym kontekście zob.~też \cite{Banchoff67,Bloch13,Lewiner13}.)

Niech $X$~będzie regularnym CW kompleksem. Funkcję $f\colon \mP(X)\to \mathbb{R}$ ze zbioru komórek CW kompleksu $X$~w~zbiór liczb rzeczywistych nazywamy \textit{dyskretną funkcją Morse'a}\index{funkcja!dyskretna Morse'a|see{dyskretna funkcja Morse'a}}\index{dyskretna funkcja Morse'a!na regularnym CW kompleksie}, jeżeli dla każdej komórki $\sigma$~kompleksu $X$~zbiory \begin{align*}u_f(\sigma)&=\{\tau\succ \sigma:f(\tau)\leq f(\sigma)\},\\ d_f(\sigma)&=\{\tau\prec \sigma:f(\tau)\geq f(\sigma)\}\end{align*} są co najwyżej jednoelementowe. 
\begin{lem}[{\cite[Lemma 2.5]{Forman98}}]\label{lem-lemat_formana_ze_tylko_jeden_niepusty}
Niech $f\colon \mP(X)\to \mathbb{R}$ będzie dyskretną funkcją Morse'a na regularnym CW kompleksie $X$. Wówczas dla każdej komórki $\sigma\in \mP(X)$ co najwyżej jeden ze zbiorów $u_f(\sigma),d_f(\sigma)$ jest niepusty.
\end{lem}
Komórkę $\sigma\in \mP(X)$ nazywamy \textit{krytyczną}\index{komozzzrka CW kompleksu@komórka CW kompleksu!krytyczna|see{krytyczność}}\index{krytycznoszzzczzz wzglezzzdem@krytyczność względem!dyskretnej funkcji Morse'a!na CW kompleksie} względem dyskretnej funkcji Morse'a $f\colon \mP(X)\to \mathbb{R}$, gdy $u_f(\sigma)=d_f(\sigma)=\emptyset$. Jeżeli dla pewnej liczby $c\in\mathbb{R}$ istnieje komórka krytyczna $\sigma\in f^{-1}(c)$, to $c$~nazywamy \textit{wartością krytyczną} funkcji $f$.

Dla regularnego CW kompleksu $X$, dyskretnej funkcji Morse'a $f\colon \mP(X)\to \mathbb{R}$ oraz $c\in\mathbb{R}$ rozpatrzmy podkompleks kompleksu $X$~zadany wzorem \[X(c)=\bigcup_{f(\tau)\leq c}\bigcup_{\sigma\subseteq \tau}\sigma,\] to znaczy najmniejszy podkompleks zawierający wszystkie te komórki, na których funkcja $f$~przyjmuje wartość co najwyżej $c$.

Następujące twierdzenia są dyskretnymi odpowiednikami twierdzeń \ref{tw313}, \ref{tw314}.

\begin{tw}[{\cite[Theorem 3.3]{Forman98}}]\label{tw-formana_o_homotopijnej_rownowaznosci_miedzy_poziomami}
Niech $X$~będzie zwartym, regularnym CW kompleksem, zaś $f\colon \mP(X)\to\mathbb{R}$ dyskretną funkcją Morse'a. Jeżeli $a,b\in\mathbb{R}$, $a<b$ oraz przedział $[a,b]$ nie zawiera wartości krytycznych funkcji $f$, to $X(b)\searrow X(a)$.
\end{tw}

\begin{tw}[{\cite[Theorem 3.4]{Forman98}}]\label{tw319}
Niech $X$~będzie zwartym, regularnym CW kompleksem, $f\colon \mP(X)\to\mathbb{R}$ dyskretną funkcją Morse'a, zaś $\sigma\in \mP(X)$ komórką krytyczną względem $f$. Jeżeli dla $a,b\in\mathbb{R}$, $a<b$, jedyną komórką krytyczną należącą do zbioru $f^{-1}[a,b]$ jest $\sigma$, to kompleks $X(b)$ jest homotopijnie równoważny CW~kompleksowi powstałemu przez doklejenie $\dim(\sigma)$-wymiarowej komórki do kompleksu $X(a)$.
\end{tw}

Dla $i\in\mN$ oznaczmy przez $c_i^f(X)$ liczbę $i$-wymiarowych komórek krytycznych kompleksu $X$~względem dyskretnej funkcji Morse'a $f\colon \mP(X)\to \mathbb{R}$. Dowód następującego twierdzenia, czasem nazywanego głównym twierdzeniem dyskretnej teorii Morse'a i~będącego odpowiednikiem twierdzenia \ref{tw-main_thm_morse_theory}, opiera się na twierdzeniach \ref{tw-formana_o_homotopijnej_rownowaznosci_miedzy_poziomami}, \ref{tw319}.

\begin{tw}[{\cite[Corollary 3.5]{Forman98}}]\label{tw-glowne_tw_klasycznej_dyskretnej_teorii_morsea}
Zwarty, regularny CW kompleks $X$~z~zadaną dyskretną funkcją Morse'a $f$~ma typ homotopijny CW kompleksu, który dla każdego $i\in\mN$ ma dokładnie $c_i^f(X)$ komórek $i$-wymiarowych.
\end{tw}

Uogólnienie twierdzenia \ref{tw-glowne_tw_klasycznej_dyskretnej_teorii_morsea} na nieskończone CW kompleksy stanowi jeden z~głównych celów bieżącego rozdziału.

Wnioskiem z~twierdzenia \ref{tw-glowne_tw_klasycznej_dyskretnej_teorii_morsea} są tak zwane dyskretne nierówności Morse'a, stanowiące odpowiednik wniosku \ref{wn-nier_morsea}.

\begin{wn}[{\cite[Corollaries 3.6, 3.7]{Forman98}}]\label{wn-klasyczne_dyskretne_nierownosci_morsea}
Niech $X$~będzie zwartym, regularnym CW kompleksem, zaś $f\colon \mP(X)\to\mathbb{R}$ dyskretną funkcją Morse'a. Dla każdego $n\in\mN$ mają miejsce nierówności
\[\sum_{i=0}^{n}(-1)^{n-i}c_i^f(X)\geq \sum_{i=0}^{n}(-1)^{n-i}\beta_i(X)\]
oraz
\[c_n^f(X)\geq \beta_n(X).\] Ponadto charakterystyka Eulera CW kompleksu $X$~wyraża się wzorem \[\chi(X)=\sum_{i=0}^{\dim(X)}(-1)^i c_i^f(X).\]
\end{wn}

Forman \cite{Forman02} przypisuje Chariemu \cite{Chari00}  obserwację, że dyskretne funkcje Morse'a opisać można za pomocą pewnego typu skojarzeń w~diagramie Hassego uporządkowanego zbioru ścian, zwanych skojarzeniami Morse'a, i~w~języku skojarzeń Morse'a wysłowić powyższe twierdzenia. (Forman \cite{Forman98} korzystał ze zbliżonego pojęcia dyskretnego gradientowego pola wektorowego.) Poniżej przybliżamy ten język, stosowany w~dalszej części rozdziału.

Niech $X$~będzie regularnym CW kompleksem, zaś $f\colon \mP(X)\to \mR$ dyskretną funkcją Morse'a. Wobec lematu \ref{lem-lemat_formana_ze_tylko_jeden_niepusty} istnieje skojarzenie \[M(f)=\{(p,q)\in \mH(\mP(X)):f(q)\geq f(p)\}\] w~grafie skierowanym $\mH(\mP(X))$, które nazywamy \textit{skojarzeniem Morse'a indukowanym przez $f$}\index{skojarzenie Morse'a!indukowane przez dyskretnazzz funkcjezzz Morse'a@indukowane przez dyskretną funkcję Morse'a!na CW kompleksie}. Oznaczmy symbolem $\mH_{M(f)}(\mP(X))$ graf skierowany powstały z~$\mH(\mP(X))$ przez zmianę orientacji krawędzi należących do $M(f)$, tzn.~zastąpienie w~grafie $\mH(\mP(X))$ wszystkich krawędzi $(p,q)\in M(f)$ krawędziami $(q,p)$. Nietrudno zauważyć, że graf skierowany $\mH_{M(f)}(\mP(X))$ nie zawiera cykli. Z~drugiej strony, jeśli $N$~jest skojarzeniem w~grafie $\mH(\mP(X))$ o~tej własności, że graf skierowany $\mH_N(\mP(X))$ nie zawiera cykli, to można wykazać, że istnieje dyskretna funkcja Morse'a $f(N)\colon \mP(X)\to \mR$ taka, że $N=M(f(N))$ (por.~\cite[Theorem 3.6]{Forman02}). Skojarzenie $N$~w~grafie $\mH(\mP(X))$ o~powyższej własności nazywamy \textit{skojarzeniem Morse'a} na $X$. Jak łatwo sprawdzić, dla skojarzenia Morse'a $N$~na $X$~zbiór komórek krytycznych względem funkcji $f(N)$ pokrywa się ze zbiorem tych elementów $X$, które nie należą do żadnej krawędzi z~$N$. Możemy zatem mówić o~\textit{zbiorze komórek krytycznych} względem skojarzenia $N$.

\subsection{Algebraiczne spojrzenie na dyskretną teorię Morse'a}\label{subsec-homol_dysk_teo_mors}
Główne twierdzenie dyskretnej teorii Morse'a \ref{tw-glowne_tw_klasycznej_dyskretnej_teorii_morsea} pozwala dla zwartego, regularnego CW kompleksu z~zadaną dyskretną funkcją Morse'a zbudować homotopijnie równoważny mu CW kompleks, którego komórki są we wzajemnie jednoznacznej odpowiedniości z~komórkami krytycznymi wyjściowego kompleksu. Od początku rozwijana była również algebraiczna wersja tej teorii, dająca możliwość konstrukcji, w~oparciu o~dyskretną funkcję (czy skojarzenie) Morse'a, pewnego kompleksu łańcuchowego o~homologiach izomorficznych z~homologiami wyjściowego CW kompleksu (por.~\cite{Forman98}).

Podejście to pozwoliło na wyabstrahowanie od topologicznych korzeni i~zastosowanie dyskretnej teorii Morse'a w~sytuacjach czysto algebraicznych, do upraszczania struktury kompleksów łańcuchowych przy zachowaniu ich łańcuchowego typu homotopijnego. Wyniki osiągnęło w~tej dziedzinie niezależnie kilku autorów \cite{Jollenbeck05,Kozlov05,Skoldberg06} (zob.~też \cite{Mrozek09,Freij07,Skoldberg13}), przy czym swoich rozważań nie ograniczali oni do skończenie generowanych kompleksów łańcuchowych.
Poniżej podajemy główne twierdzenie algebraicznej wersji dyskretnej teorii Morse'a w~ujęciu J\"ollenbecka \cite{Jollenbeck05}.

Niech $R$~będzie pierścieniem z~jedynką, zaś $C_*=\left(C_i,\partial_i\right)_{i\in\mN}$ wolnym kompleksem łańcuchowym nad $R$. Ustalmy dla każdego $i\in \mN$ bazę $B_i$~modułu $C_i$. \textit{Wolnym kompleksem łańcuchowym nad $R$~z~bazą}\index{kompleks!lzzzanzzzcuchowy wolny z bazazzz@łańcuchowy wolny z bazą} nazywamy trójkę $C=\left(C_i,\partial_i,B_i\right)_{i\in\mN}$. Ustalmy $i\in \mN$. Elementy $c\in C_i$ utożsamiamy z~funkcjami $c\colon B_i\to R$ takimi, że $c(b)=0$ dla prawie wszystkich $b\in B_i$. Dla $c\in B_i$ oraz $c'\in B_{i-1}$ przyjmijmy oznaczenie $\left[c:c'\right]=\partial_i(c)(c')$.

Rozważmy graf skierowany $\mcV(C)$\nomenclature[6j]{$\mcV(C)$}{graf skierowany stowarzyszony z~wolnym kompleksem łańcuchowym z~bazą $C$} o~zbiorze wierzchołków $\coprod_{i\in\mN} B_i$ oraz zbiorze krawędzi \[\left\{\left(c,c'\right):c\in B_i,\ c'\in B_{i-1} \text{ dla pewnego } i\in\mN \text{ oraz } \left[c:c'\right]\not=0\right\}.\] Skończone skojarzenie $M$~w~grafie $\mcV(C)$ nazywamy \textit{skojarzeniem Morse'a}\index{skojarzenie Morse'a!na wolnym kompleksie lzzzanzzzcuchowym z~bazazzz@na wolnym kompleksie łańcuchowym z~bazą} na wolnym kompleksie łańcuchowym z~bazą $C$, o~ile spełnione są następujące warunki:
\begin{compactitem}
\item[---]graf skierowany $\mcV_M(C)$ powstały z~$\mcV(C)$ przez zmianę orientacji krawędzi należących do $M$~nie zawiera cykli;\nomenclature[6k]{$\mcV_M(C)$}{graf skierowany powstały z~$\mcV(C)$~przez zmianę orientacji krawędzi należących do skojarzenia $M$}
\item[---]dla każdej krawędzi $\left(c,c'\right)\in M$ współczynnik $\left[c:c'\right]$ należy do centrum pierścienia $R$~i~jest odwracalny w~$R$.
\end{compactitem}

Mówimy, że element $c\in B_i$ jest \textit{krytyczny}\index{krytycznoszzzczzz wzglezzzdem@krytyczność względem!skojarzenia Morse'a!na wolnym kompleksie lzzzanzzzcuchowym z~bazazzz@na wolnym kompleksie łańcuchowym z~bazą}~względem skojarzenia $M$, jeżeli $c$~nie należy do żadnej krawędzi z~$M$. Niech \[B_i^M=\left\{c\in B_i:c\text{ jest krytyczny względem } M\right\}.\] 

Dla  wierzchołków $c,c'\in \mcV_M(C)$ niech $\operatorname{P}^M_C\left(c,c'\right)$ oznacza zbiór ścieżek prostych w~$\mcV_M(C)$ prowadzących z~$c$~do $c'$. Dla krawędzi $(c,c')\in \mcV_M(C)$ (tzn.~dwuelementowej ścieżki prostej) przyjmujemy \[w^M_C\left((c,c')\right)=\begin{cases}\left[c:c'\right], &\text{gdy } \left(c',c\right)\not\in M,\\ -\frac{1}{\left[c:c'\right]}, & \text{gdy } \left(c',c\right)\in M.\end{cases}\] Jeżeli $p=(c_0,c_1,\ldots,c_n)\in \operatorname{P}^M_C(c_0,c_n)$, definiujemy \textit{wagę}\index{waga szzzciezzzzki prostej@waga ścieżki prostej} tej ścieżki \[w^M_C(p)=\prod_{i=0}^{n-1}w^M_C\bigl((c_i,c_{i+1})\bigr).\]Dla wierzchołków $c,c'\in\mcV_M(C)$ niech \[\Gamma^M\left(c,c'\right)=\sum_{p\in \operatorname{P}^M_C\left(c,c'\right)} w_C^M(p),\]\nomenclature[5i]{$\Gamma^M\left(c,c'\right)$}{suma wag ścieżek między wierzchołkami $c, c'$ grafu $\mcV_M(C)$}o ile $w^M_C(p)=0$ dla prawie wszystkich $p\in\operatorname{P}^M_C(c,c')$. (Zauważmy, że jeśli skojarzenie Morse'a $M$~jest skończone, to dla wszystkich $c,c'\in\mcV_M(C)$~skończony jest również zbiór $\operatorname{P}^M_C(c,c')$, więc element $\Gamma^M(c,c')\in R$ jest dobrze określony.)

\begin{tw}[{\cite[Theorem 1.2]{Jollenbeck05}}]\label{tw-glowne_tw_algebraicznej_dyskretnej_teorii_morsea}
Jeżeli $R$~jest pierścieniem z~jedynką, zaś $M$~jest skończonym skojarzeniem Morse'a na wolnym kompleksie łańcuchowym nad $R$~z~bazą $C=(C_i,\partial_i,B_i)_{i\in\mN}$, to istnieje łańcuchowo homotopijnie równoważny kompleksowi \mbox{$C_*=(C_i,\partial_i)_{i\in\mN}$} wolny kompleks łańcuchowy \mbox{$C_*^M=(C_i^M,\partial_i^M)_{i\in\mN}$} nad $R$~taki, że dla każdego $i\in\mN$ bazą wolnego $R$-modułu $C_i^M$ jest zbiór $B_i^M\subseteq B_i$ elementów krytycznych względem skojarzenia $M$, zaś homomorfizm $\partial_i^M\colon C_i^M\to C_{i-1}^M$ jest dla $c\in B_i^M$, $c'\in B_{i-1}^M$ zadany wzorem \[\partial^M(c)(c')=\Gamma^M(c,c').\]
Odwzorowania łańcuchowe $f\colon C_*\to C_*^M$, $g\colon C_*^M\to C_*$ wyznaczające łańcuchową homotopijną równoważność są zadane wzorami:
\begin{align*}
f_i(c)(c')&=\Gamma^M\left(c,c'\right)\quad \text{dla } c\in B_i,\ c'\in B_{i}^M\\
g_i(c)(c')&=\Gamma^M\left(c,c'\right)\quad \text{dla } c\in B^M_i,\ c'\in B_i.
\end{align*}
\end{tw}

Za $C_*=(C_i,\partial_i)_{i\in\mN}$~przyjmijmy kompleks łańcuchów komórkowych \cite[Section 2.2]{Hatcher02} nad $R$~pewnego regularnego CW kompleksu $X$, zaś $B_i$~niech oznacza, dla $i\in \mN$, zbiór generatorów wolnego $R$-modułu $C_i$~odpowiadających \mbox{$i$-wymiarowym} komórkom CW~kompleksu $X$. Każde skończone skojarzenie Morse'a $M$~na $X$~indukuje w~naturalny sposób skończone skojarzenie Morse'a $\widetilde{M}$ na $(C_i,\partial_i,B_i)_{i\in\mN}$, przy czym \mbox{$i$-wymiarowe} komórki $X$~krytyczne względem $M$~odpowiadają elementom zbioru $B_i^{\widetilde{M}}$. Czytelnik z~łatwością sformułuje, korzystając z~twierdzenia \ref{tw-glowne_tw_algebraicznej_dyskretnej_teorii_morsea}, homologiczny odpowiednik twierdzenia \ref{tw-glowne_tw_klasycznej_dyskretnej_teorii_morsea}. 

Można rozpatrywać skojarzenia Morse'a na kompleksach łańcuchów komórkowych CW kompleksów, które nie są regularne. Musimy jednak pamiętać, że współczynniki $[c:c']$ powinny być odwracalne i~należeć do centrum pierścienia $R$~dla wszystkich par komórek $(c,c')$ będących elementami takiego skojarzenia.

\subsection{Porównanie}
W~bieżącej sekcji porównujemy, w~oparciu o~pracę Benedettiego \cite{Benedetti13}, niektóre aspekty gładkiej oraz dyskretnej teorii Morse'a, wskazując, że w~wielu przypadkach pozwalają one osiągnąć zbliżone wyniki, a~niekiedy rezultaty uzyskiwane przy użyciu dyskretnej teorii mogą być nawet ,,lepsze''. 

Obiektami, do których stosuje się gładka teoria Morse'a, są gładkie rozmaitości; teoria dyskretna służy badaniu skończonych CW kompleksów. Każda zwarta, gładka rozmaitość jest triangulowalna (jeden z~dowodów podał Cairns \cite{Cairns61}), zatem rozmaitości tego typu stanowią naturalne pole do porównania możliwości obu teorii.

\textit{Wektorem Morse'a} realizowanym przez gładką (dyskretną) funkcję Morse'a $f$~określoną na zwartej, $n$-wymiarowej rozmaitości $M$ (zwartym, $n$-wymiarowym CW kompleksie $X$) nazywamy wektor $\bigl(m_0^f(M),\ldots,m_n^f(M)\bigr)$ (wektor $\bigl(c_0^f(X),\ldots,c_n^f(X)\bigr)$), którego $i$-tą współrzędną jest liczba punktów krytycznych (komórek krytycznych) funkcji $f$~o~indeksie (wymiarze) równym $i$.

Ponieważ wektor Morse'a opisuje liczbę komórek, z~których składa się pewien CW kompleks homotopijnie równoważny obiektowi, na którym funkcja ta jest zadana, daje on pewne pojęcie o~tym, na ile CW kompleks ten jest ,,skomplikowany''. Jeśli więc naszym celem jest uproszczenie opisu topologii danego obiektu, wektor Morse'a stanowi miarę tego, na ile dobrze dana funkcja Morse'a spełnia to zadanie. (Zagadnieniu znajdowania funkcji Morse'a o~optymalnym, tj.~mającym jak najmniejsze współrzędne, wektorze Morse'a poświęcono wiele uwagi. Dla gładkich funkcji Morse'a informacje na ten temat można znaleźć w~książce Sharko \cite{Sharko93}; w~przypadku dyskretnym jego omówienie zawiera np.~praca Benedettiego i~Lutza \cite{Benedetti14}.)

Aby porównać wektory Morse'a możliwe do osiągnięcia przy podejściu gładkim i~dyskretnym Benedetti \cite{Benedetti13}~wykorzystuje elementy PL topologii.\footnote{Skrót ,,PL'' pochodzi od ang.~\textit{piecewise linear}.} Przypomnijmy, że jeśli $K$~jest kompleksem symplicjalnym, to jego \textit{podpodziałem} nazywamy każdy taki kompleks symplicjalny $K'$, że istnieje homeomorfizm \mbox{$h\colon |K|\to |K'|$} o~tej własności, że dla każdego sympleksu $\sigma\in K'$ istnieje sympleks $\tau\in K$ taki, że $h(|\sigma|)\subseteq |\tau|$. Dwa kompleksy symplicjalne $K,L$ nazywamy \textit{PL homeomorficznymi}, o~ile pewne ich podpodziały $K', L'$ są izomorficzne (jako abstrakcyjne kompleksy symplicjalne). Przez \textit{PL kulę} (odpowiednio \textit{PL sferę}) rozumiemy kompleks symplicjalny PL homeomorficzny sympleksowi (brzegowi sympleksu). Jeżeli kompleks symplicjalny $K$~triangulujący pewną rozmaitość topologiczną ma tę własność, że dla każdego wierzchołka $v\in K$ kompleks $\st_K(v)$ jest PL kulą, to triangulację tę nazywamy \textit{PL triangulacją}. Wspomniana wyżej triangulacja gładkiej, zwartej rozmaitości, opisana przez Cairnsa \cite{Cairns61}, jest PL triangulacją.

Jeśli ustalimy kompleks symplicjalny $K$~będący PL triangulacją zwartej, gładkiej rozmaitości $M$, może zdarzyć się, że znajdziemy gładką funkcję Morse'a $M\to \mathbb{R}$ realizującą wektor Morse'a o~wiele ,,lepszy'', niż jakikolwiek wektor Morse'a realizowany przez dyskretną funkcję Morse'a $\mP(K)\to \mathbb{R}$. Przykładowo, na każdej sferze (ze standardową strukturą gładką) istnieje gładka funkcja Morse'a o~dokładnie dwóch komórkach krytycznych. Z~drugiej strony, Benedetti \cite{Benedetti12} otrzymał następujący wynik.

\begin{tw}[{\cite[Theorem 4.18]{Benedetti12}}]
Dla dowolnych $k,d\in \mathbb{N}$ takich, że $d\geq 3$, istnieje $d$-wymiarowa PL sfera $K$~o~tej własności, że każda dyskretna funkcja Morse'a na $K$~ma co najmniej $k$~krytycznych komórek $(d-1)$-wymiarowych.
\end{tw}

Inaczej sytuacja przedstawia się, gdy wyniki teorii dyskretnej stosujemy nie do ustalonej triangulacji danej rozmaitości, ale do odpowiedniego jej podpodziału. (Przykładem podobnego podejścia jest twierdzenie o~aproksymacji symplicjalnej ciągłych odwzorowań między wielościanami.) Korzystając z~rozkładu PL rozmaitości na rączki Benedetti \cite{Benedetti13} udowodnił poniższe twierdzenie. (Wcześniej zbliżony wynik, z~drobną luką w~dowodzie \cite[Remark 2.29]{Benedetti13}, opublikowany został przez Gallais \cite[Theorem 3.1]{Gallais10}.)

\begin{tw}[{\cite[Main Theorems A, B]{Benedetti13}}]\label{tw-porownawcze_benedettiego}
Niech $M$~będzie zwartą, gładką rozmaitością bez brzegu, zaś $K$~jej ustaloną PL triangulacją. Ustalmy wektor liczb naturalnych $\nu\in\mathbb{N}^{\dim(M)}$.

Jeżeli $v$~jest wektorem Morse'a realizowanym przez pewną gładką funkcję Morse'a $M\to \mathbb{R}$, to istnieje $n\in\mN$~takie, że $\nu$~jest wektorem Morse'a pewnej dyskretnej funkcji Morse'a $\mP\left((\mK\circ \mP)^n(K)\right)\to \mathbb{R}$, zadanej na $n$-tym podziale barycentrycznym kompleksu $K$.

Jeśli $\dim(M)\leq 7$ oraz $\nu$~jest wektorem Morse'a pewnej dyskretnej funkcji Morse'a $\mP(K)\to\mathbb{R}$, to $\nu$~jest również realizowany przez pewną gładką funkcję Morse'a $M\to \mathbb{R}$.
\end{tw}

Najmniejsza liczba $n$, dla której zachodzi teza pierwszej części twierdzenia \ref{tw-porownawcze_benedettiego}, zależy od~wyboru triangulacji $K$~i~może być (nawet w~przypadku $M$~będącego sferą) dowolnie duża \cite[Proposition 3.20]{Benedetti13}.

Wobec twierdzenia \ref{tw-porownawcze_benedettiego} w~niskich wymiarach podejście gładkie i~dyskretne są ,,równoważne'', o~ile ograniczymy się do PL triangulacji. Dyskretna teoria Morse'a może jednak być stosowana również do triangulacji nie mających własności PL. Okazuje się, że pozwala to uzyskać przy jej użyciu lepsze wektory Morse'a, niż stosując teorię gładką.

\begin{tw}[{\cite[Subsection 3.3]{Adiprasito13}}]
Dla każdego $d\geq 5$ istnieje gładka, \mbox{$d$-wymiarowa}, zwarta rozmaitość $M$~taka, że $(1,0,\ldots,0)$ jest wektorem Morse'a dyskretnej funkcji Morse'a zadanej na pewnej triangulacji rozmaitości $M$, ale wektor ten nie jest realizowany przez żadną gładką funkcję Morse'a na $M$.
\end{tw}


%==============================================================
%==============================================================
%==============================================================

\section{Nieskończone skojarzenia Morse'a i promienie}\label{rayless}
W~niniejszym podrozdziale definiujemy pojęcie skojarzenia Morse'a na dowolnym częściowym porządku. Często nie będą one miały wiele wspólnego z~zaproponowanym przez Formana pojęciem dyskretnej funkcji Morse'a, choćby z~tego powodu, że rozważane częściowe porządki mogą pod żadnym względem nie przypominać uporządkowanego zbioru ścian regularnego CW kompleksu. Wprowadzone w~tym podrozdziale pojęcia traktować można jako ,,abstrakcyjny nonsens'', którego bardziej konkretne zastosowania znajdują się w~dalszej części rozdziału.

Jeśli $D=(V,E)$ jest grafem skierowanym, zaś $M\subseteq E$ podzbiorem zbioru jego krawędzi, to \textit{grafem skierowanym powstałym przez zmianę orientacji krawędzi należących do $M$}\index{graf!skierowany!powstalzzzy przez zmianezzz orientacji krawezzzdzi@powstały przez zmianę orientacji krawędzi} nazywamy graf skierowany $D'=(V,E')$, gdzie \[E'=(E\smallsetminus M)\cup\{(w_2,w_1):(w_1,w_2)\in M\}.\] Jeśli $M$ jest podzbiorem zbioru krawędzi~diagramu Hassego $\mH(P)$ pewnego częściowego porządku $P$, to graf skierowany powstały przez zmianę orientacji krawędzi należących do $M$~oznaczamy symbolem $\mH_M(P)$.\nomenclature[6i]{$\mH_M(P)$}{graf skierowany powstały z~$\mH_M(P)$~przez zmianę orientacji krawędzi należących do zbioru $M$}

Skojarzenie $M$ w~diagramie Hassego $\mcH(P)$ częściowego porządku $P$ o~tej własności, że graf skierowany $\mH_M(P)$ nie zawiera cykli, nazywamy \textit{skojarzeniem Morse'a na $P$}\index{skojarzenie Morse'a!na czezzzszzzciowym porzazzzdku@na częściowym porządku}. Element $x\in P$ nazywamy \textit{krytycznym względem $M$}\index{krytycznoszzzczzz wzglezzzdem@krytyczność względem!skojarzenia Morse'a!na czezzzszzzciowym porzazzzdku@na częściowym porządku}, o~ile $(x,y)\not\in M$ oraz $(y,x)\not\in M$ dla wszystkich $y\in P$. Przez skojarzenie Morse'a na regularnym CW kompleksie $X$ rozumiemy skojarzenie Morse'a na $\mcP(X)$.\index{skojarzenie Morse'a!na regularnym CW kompleksie}

Ciąg $(x_i)_{i\in\mN}$ wierzchołków grafu skierowanego $D$ nazywamy \textit{promieniem malejącym}\index{promienzzz@promień!malejazzzcy@malejący} \cite{Ayala11}, o~ile $(x_i,x_{i+1})\in D$ oraz $x_i\not=x_j$ dla wszystkich $i,j\in\mN$, $i\not=j$. Graf $D$~nazywamy \textit{grafem skierowanym bez promieni malejących}\index{graf!skierowany!bez promieni malejazzzcych@bez promieni malejących}, jeżeli $D$~nie zawiera promienia malejącego. Mówimy, że skojarzenie Morse'a $M$~na porządku $P$~jest \textit{bez promieni malejących}\index{skojarzenie Morse'a!bez promieni malejazzzcych@bez promieni malejących}, gdy graf skierowany $\mcH_M(P)$ jest bez promieni malejących. 

Wybór nazwy ,,promień malejący'' uzasadniony jest faktem, że jeśli $P=\mP(X)$~jest uporządkowanym zbiorem ścian regularnego CW kompleksu $X$~oraz istnieje dyskretna funkcja Morse'a na~$X$ indukująca (w~sposób opisany w~sekcji \ref{subsec-forman_classical_discrete_morse}) skojarzenie Morse'a $M$~na $P$, to jej wartości maleją wzdłuż każdego promienia malejącego w~$\mH_M(P)$.

\textit{Promieniem rosnącym}\index{promienzzz@promień!rosnazzzcy@rosnący} \cite{Ayala11} nazywamy ciąg $(x_i)_{i\in\mN}$ wierzchołków grafu skierowanego $D$~taki, że $(x_{i+1},x_{i})\in D$ oraz $x_i\not=x_j$ dla wszystkich $i,j\in\mN$, $i\not=j$.

Skojarzenia Morse'a bez promieni malejących odgrywają w~niniejszym rozdziale szczególnie ważną rolę. Użyteczne okażą się ich alternatywne charakteryzacje.
\begin{lem}\label{lem-niesk_sciezka_a_skojarzenie_morse}
Skojarzenie $M$~w~diagramie Hassego częściowego porządku $P$ jest skojarzeniem Morse'a bez promieni malejących wtedy i~tylko wtedy, gdy graf skierowany $\mH_M(P)$ nie zawiera nieskończonej ścieżki.
\end{lem}
\begin{proof}
Ustalmy częściowy porządek $P$~oraz skojarzenie $M$~w~grafie skierowanym $\mH(P)$. Jeśli $\mH_M(P)$ zawiera nieskończoną ścieżkę $(x_i)_{i\in\mN}$, to albo jest ona promieniem malejącym, albo istnieją liczby naturalne $i_1<i_2$ takie, że $x_{i_1}=x_{i_2}$. Jeżeli zachodzi drugi z~tych warunków, możemy liczby $i_1,i_2$~wybrać w~ten sposób, że nie istnieją $j_1,j_2\in\mN$ takie, że $i_1\leq j_1< j_2\leq i_2$, $(i_1,i_2)\not=(j_1,j_2)$ oraz~$x_{j_1}=x_{j_2}$. Ścieżka $\left(x_{i_1},x_{i_1+1},\ldots,x_{i_2-1}\right)$ jest wówczas cyklem.

Z~drugiej strony, każdy promień malejący w~$\mH_M(P)$ jest nieskończoną ścieżką. Natomiast jeśli $\mH_M(P)$ zawiera cykl $\left(x_0,\ldots,x_{n-1}\right)$, to ciąg $(y_i)_{i\in\mN}$, gdzie $y_i=x_{i\!\mod n\ }$, jest nieskończoną ścieżką.
\end{proof}

Poniższy lemat inspirowany jest wynikiem z~książki Kozlova \cite[Theorem 11.2]{Kozlov08}.

\begin{lem}\label{lem-kozlov_lemma}
Niech $P$~będzie dobrze ufundowanym częściowym porządkiem oraz
niech $M$~będzie skojarzeniem w~grafie $\mH(P)$. Skojarzenie $M$~jest skojarzeniem Morse'a bez promieni malejących na $P$~wtedy~i~tylko wtedy, gdy istnieje liniowe rozszerzenie \mbox{$P^*=(P,\leq^*)$} częściowego porządku $P$ będące dobrym porządkiem o~tej własności, że jeśli $(p,q)\in M$ dla pewnych $p,q\in P$, to $p$~jest pokryciem górnym $q$ w~$P^*$.
\end{lem}
\begin{proof}
Załóżmy najpierw, że istnieje liniowe rozszerzenie $P^*$ porządku $P$ o~powyższych własnościach. Przypuśćmy, że w~$\mH_M(P)$ istnieje nieskończona ścieżka $(x_i)_{i\in\mN}$. Ponieważ porządek $P$ jest dobrze ufundowany, ścieżka $(x_i)_{i\in\mN}$ nie może zawierać nieskończonego ciągu zstępującego w~$P$, więc w~szczególności dla każdego $j\in\mN$ istnieje liczba naturalna $i>j$ taka, że $(x_{i},x_{i-1})\in M$.

Niech $\bigl(x_{i_k}\bigr)_{k\in\mN}$ będzie podciągiem ciągu $(x_i)_{i\in\mN}$ składającym się z~tych elementów $x_i$, dla których $(x_{i},x_{i-1})\in M$. Ustalmy $k\in\mN$. Ponieważ $M$ jest skojarzeniem oraz $\left(x_{i_k},x_{i_k-1}\right)\in M$, nie jest możliwym, aby $\left(x_{i_k+1},x_{i_k}\right)\in M$. Mamy zatem $x_{i_k}\succ x_{i_k+1}$, a~stąd również $x_{i_k}>^* x_{i_k+1}$. Ale $x_{i_k}\succ^* x_{i_k-1}$, a~porządek $P^*$~jest liniowy, więc \begin{equation}x_{i_k-1}>^*x_{i_k+1}.\label{kozlov_lemma-eq1}\end{equation} Ponieważ $(x_i,x_{i-1})\not\in M$ dla $i_k+1\leq i<i_{k+1}$, mamy \[x_{i_k+1}\succ x_{i_k+2}\succ\ldots \succ x_{i_{k+1}-1},\] a~zatem również \begin{equation}x_{i_k+1}>^*x_{i_{k+1}-1}.\label{kozlov_lemma-eq2}\end{equation} Zestawiając nierówności (\ref{kozlov_lemma-eq1}) oraz~(\ref{kozlov_lemma-eq2}) otrzymujemy, że $x_{i_k-1}>^*x_{i_{k+1}-1}$. Ponieważ indeks $k$~był ustalony w~sposób dowolny, $\left(x_{i_k-1}\right)_{k\in\mN}$ jest nieskończonym  zstępującym ciągiem w~$P^*$. Ale zbiór $P^*$ jest dobrze uporządkowany, nie może więc zawierać takiego ciągu. Otrzymaliśmy sprzeczność; nie istnieje zatem nieskończona ścieżka $(x_i)_{i\in\mN}$ w~$\mH_M(P)$. Wobec lematu \ref{lem-niesk_sciezka_a_skojarzenie_morse} skojarzenie $M$~jest skojarzeniem Morse'a bez promieni malejących.

Załóżmy teraz, że $M$~jest skojarzeniem Morse'a na $P$~bez promieni malejących. Skonstruujemy liczbę porządkową $\alpha$~oraz ciąg pozaskończony dobrych porządków $\left(\left(P_\phi, \leq_\phi\right)\right)_{\phi<\alpha}$, za pomocą którego zdefiniujemy liniowe rozszerzenie $P^*$~częściowego porządku $P$. 

Niech $P_0=\emptyset$. Załóżmy, że dla pewnej liczby porządkowej $\phi>0$ jest dany ciąg dobrych porządków $\left(\left(P_\psi,\leq_\psi\right)\right)_{\psi<\phi}$ spełniający poniższe warunki:
\begin{itemize}
\item[$(a_{\phi})$] dobry porządek $\left(P_{\psi_1},\leq_{\psi_1}\right)$ jest odcinkiem początkowym dobrego porządku $\left(P_{\psi_2},\leq_{\psi_2}\right)$ dla wszystkich $\psi_1\leq\psi_2<\phi$;
\item[$(b_{\phi})$] zachodzi zawieranie zbiorów elementów $P_\psi\subseteq P$ dla wszystkich $\psi<\phi$;
\item[$(c_{\phi})$] jeśli przyjmiemy oznaczenie $A_{\phi}=\min\left(P\smallsetminus\bigcup_{\psi<\phi}P_\psi\right)$, to dla każdego elementu $x\in A_{\phi}$ albo jest on elementem krytycznym, albo istnieje element $u(x)\in P\smallsetminus \bigcup_{\psi<\phi}P_\psi$ taki, że $(u(x),x)\in M$.
\end{itemize}
Przy tych założeniach określimy dobry porządek $\left(P_{\phi},\leq_{\phi}\right)$.

Jeżeli istnieje $x_0\in A_{\phi}$ będące elementem krytycznym względem $M$, to przyjmujemy \[P_{\phi}=\left(\bigcup_{\psi<\phi}P_\psi\right)\oplus \{x_0\}.\] 

Jeśli $A_{\phi}\not=\emptyset$ i~dla każdego $x\in A_{\phi}$ istnieje $u(x)\in P\smallsetminus \bigcup_{\psi<\phi}P_\psi$ takie, że $(u(x),x)\in M$, to rozważmy podgraf $D$~grafu skierowanego $\mH_M(P)$ indukowany na zbiorze wierzchołków $A_{\phi}\cup \left\{u(x):x\in A_{\phi}\right\}\subseteq P\smallsetminus \bigcup_{\psi<\phi}P_\psi$.  Ponieważ graf skierowany $D$ nie zawiera promieni malejących ani cykli, jak łatwo zauważyć musi istnieć wierzchołek $y_0\in D$, z~którego nie wychodzi żadna krawędź grafu $D$. Liczba krawędzi grafu $D$~wychodzących z~$x$~jest równa $1$~dla wszystkich $x\in A_{\phi}$, więc wierzchołek $y_0$ musi być postaci $y_0=u(x_0)$ dla pewnego $x_0\in A_{\phi}$. Definiujemy $P_{\phi}=\left(\bigcup_{\psi<\phi}P_\psi\right)\oplus \{x_0\}\oplus\{y_0\}$.

Nietrudno zauważyć, że w~obu powyższych przypadkach ciąg $\left(\left(P_\psi,\leq_\psi\right)\right)_{\psi<\phi+1}$ spełnia warunki $(a_{\phi+1})$, $(b_{\phi+1})$, $(c_{\phi+1})$.

Gdy natomiast $A_{\phi}=\emptyset$, to przyjmujemy $\alpha=\phi$ i~kończymy konstrukcję. Zauważmy, że w~tym przypadku $P\smallsetminus \bigcup_{\psi<\phi}P_\psi=\emptyset$, gdyż porządek $P$~jest dobrze ufundowany.

Łatwo sprawdzić, że $P^*=\bigcup_{\psi<\alpha}P_\psi$ jest liniowym rozszerzeniem porządku $P$~o~żądanych własnościach.
\end{proof}


%==============================================================
%==============================================================
%==============================================================


\section{Odwracanie promieni}\label{reversing}
Niech $r=(r_i)_{i\in\mN}$ oraz $s=(s_i)_{i\in\mN}$ będą promieniami malejącymi w~grafie skierowanym $D$. Pisząc $x\in r$ mamy na myśli istnienie takiego indeksu $i\in\mN$, że $x=r_i$, tzn.~przynależność $x$~do zbioru wyrazów ciągu $r$.

Promienie malejące $r, s$ nazywamy \textit{równoważnymi}\index{promienzzz@promień!malejazzzcy@malejący!rozzzwnowazzzzny innemu promieniowi@równoważny innemu promieniowi}\index{rozzzwnowazzzznoszzzczzz@równoważność!promieni malejazzzcych@promieni malejących} \cite{Ayala11}, o~ile istnieją liczby $m,n\in\mN$ takie, że $r_{m+i}=s_{n+i}$ dla wszystkich $i\in\mN$. Klasę abstrakcji promienia malejącego $r$ względem tej relacji oznaczamy przez $[r]$.\nomenclature[5b]{$[r]$}{klasa abstrakcji promienia malejącego $r$}

Mówimy, że dla pewnych liczb naturalnych $n\geq 0$, $k>0$ promień malejący $r$ ma \textit{obwodnicę}\index{obwodnica} $(b_i)_{i=0}^{m}$ zaczynającą się w~$r_n$~i~kończącą w~$r_{n+k}$, jeżeli $(b_i)_{i=0}^{m}$ jest ścieżką prostą w~grafie $D$, $b_0=r_n$, $b_m=r_{n+k}$ oraz spełniony jest jeden z~poniższych warunków:
\begin{compactitem}
\item[---] $m\geq 2$ oraz~$b_i\not\in r$ dla $1\leq i\leq m-1$,
\item[---] $m=1$ oraz $k\geq 1$.
\end{compactitem}

Przykłady obwodnic przedstawione są na rysunku \ref{fig-bypass}. 

\begin{figure}[h]
\centering
$\xymatrix@C=22pt{
& & x_1\ar[r] & x_2\ar[r]& x_3\ar[r]& x_4\ar[r] & x_5\ar[r] & x_6\ar[r] & x_7\ar[r] & x_8\ar[dl] \\ 
r_0\ar[r] & r_1\ar[r]\ar@/_3pc/[rrrrr] & r_2\ar[r] & r_3\ar[r]\ar[ul] & r_4\ar[r] & r_5\ar[r] & r_6\ar[r] & r_7\ar[r] & r_8\ar[r] & r_9\ar[r] & \cdots \\\\}$
\caption{Promień malejący $(r_i)_{i\in\mN}$ ma obwodnicę $(r_3,x_1,x_2,\ldots,x_6,r_8)$, zaczynającą się w~$r_3$~a~kończącą w~$r_8$, oraz obwodnicę $(r_1,r_6)$, zaczynającą się w~$r_1$~i~kończącą w~$r_6$.}\label{fig-bypass}
\end{figure}

\begin{lem}\label{lem-sk_przekroj_promieni_nierownowaznych}
Niech $r,s$~będą promieniami malejącymi w~grafie skierowanym $D$~nie zawierającym cykli, przy czym $r$~niech nie ma obwodnic. Jeżeli część wspólna zbiorów wyrazów ciągów $r$~oraz $s$~jest nieskończonym zbiorem, to $[r]=[s]$.
\end{lem}
\begin{proof}
Załóżmy, że zbiór wierzchołków $D$~będących wyrazami każdego z~ciągów $r, s$~jest nieskończony. Przypuśćmy, że $[r]\not=[s]$. Istnieją zatem liczby naturalnie $i_0<i_1$ oraz $a\not=b$ takie, że $s_{i_0}=r_a,s_{i_1}=r_b$ oraz $s_j\not\in r$ dla wszystkich $i_0<j<i_1$, przy czym $(i_1,b)\not=(i_0+1,a+1)$.

Gdyby $b<a$, to $\left(s_{i_0},s_{i_0+1},\ldots,s_{i_1},r_{b+1},\ldots,r_{a-1}\right)$ byłoby cyklem w~$D$, co jest niemożliwe. Zatem $a<b$, co oznacza, że $\left(s_{i_0},s_{i_0+1},\ldots,s_{i_1}\right)$ jest obwodnicą promienia $r$. Otrzymaliśmy sprzeczność z~założeniem lematu. Wobec tego $[r]=[s]$.
\end{proof}

Promień malejący $r$~nazywamy \textit{multipromieniem}\index{multipromienzzz@multipromień}, o~ile dla każdej liczby $n\in\mN$ istnieje liczba $i\in\mN$ taka, że promień $r$~ma obwodnicę zaczynającą się w~$r_{n+i}$.

\begin{lem}\label{prop_multiray}
Jeżeli graf skierowany  $D$~nie zawiera cykli oraz zawiera multipromień, to moc rodziny klas abstrakcji promieni malejących zawartych w~grafie $D$~jest nie mniejsza niż $2^{\aleph_0}$.
\end{lem}
\begin{proof}
Ustalmy graf skierowany $D$~nie zawierającyc cykli oraz multipromień $r=(r_i)_{i\in\mN}$ zawarty w~$D$. Przez $\bigl(r_{i_j}\bigr)_{j\in\mN}$ oznaczmy taki podciąg ciągu $(r_i)_{i\in\mN}$, że dla każdego $j\in\mN$ promień malejący $r$~ma obwodnicę $b(j)$ zaczynającą się w~$r_{i_j}$, a~kończącą w~$r_{i_j+k}$ dla pewnego $k\in\mN$ takiego, że $i_j+k<i_{j+1}$. Ponieważ graf skierowany $D$~nie zawiera cykli, łatwo zauważyć, że dla $i\not=j$ ścieżki proste $b(i),b(j)$ nie mają elementów wspólnych. Przez $r(j)$ oznaczmy dla $j\in\mN$ ścieżkę prostą $\bigl(r_{i_j},r_{i_j+1},\ldots,r_{i_{j+1}}\bigr)$ będącą fragmentem promienia malejącego~$r$.

Dla $A\subseteq\mN$ przez $r_A$ oznaczmy promień malejący powstały przez zastąpienie w~promieniu $r$~fragmentu $r(j)$ ścieżką prostą $b(j)$ dla wszystkich $j\in A$. Dla $A,B\subseteq \mN$ promienie malejące $r_A, r_B$ są równoważne wtedy i~tylko wtedy, gdy różnica symetryczna zbiorów $A$~i~$B$~jest skończonym zbiorem. Wobec tego zbiór $\left\{[r_A]:A\subseteq\mN\right\}$ jest mocy $2^{\aleph_0}$.
\end{proof}

Rozważmy skojarzenie Morse'a $M$~na częściowym porządku z~gradacją $P$~oraz ciąg $r=(r_i)_{i\in\mN}$ będący promieniem malejącym w~$\mcH_M(P)$. Niech
\begin{align*}i_0& =\min\left\{i:\Ht(r_j)=i \text{ dla pewnego } j\in\mN\right\},\\
j_0& =\min\left\{j:\Ht(r_j)=i_0\right\}.
\end{align*}
Nietrudno spostrzec, że ponieważ $M$~jest skojarzeniem, promień malejący $\left(r_{j_0}, r_{j_0+1}, r_{j_0+2},\ldots\right)$ zawiera jedynie elementy rangi $i_0$ oraz $i_0+1$. Mówimy, że $i_0$ jest \textit{rangą promienia $r$}\index{ranga!promienia malejazzzcego@promienia malejącego}, czy też że $r$ jest \textit{$i_0$-promieniem}\index{promienzzz@promień!malejazzzcy@malejący!n-promienzzz@$n$-promień}; symbolicznie: $\Ht(r)=i_0$.\nomenclature[5a]{$\Ht(r)$}{ranga promienia malejącego $r$} Zauważmy, że równoważne promienie mają równą rangę.

Dla skojarzenia Morse'a $M$~na częściowym porządku $P$~przez $\mcC^M(P)$\nomenclature[5c]{$\mcC^M(P)$}{zbiór elementów częściowego porządku $P$ krytycznych ze względu na skojarzenie Morse'a $M$} oznaczmy zbiór elementów częściowego porządku $P$~krytycznych ze względu na $M$, zaś przez $\mcR^M(P)$\nomenclature[5e]{$\mcR^M(P)$}{rodzina klas równoważności promieni malejących w~grafie $\mH_M(P)$} rodzinę klas równoważności promieni malejących w~$\mH_M(P)$. Ponadto, jeśli $P$~jest porządkiem z~rangą, niech \[\mcC^M_n(P)=\left\{c\in\mcC^M(P):\rk(c)=n\right\}\]\nomenclature[5d]{$\mcC_n^M(P)$}{zbiór elementów rangi $n$ częściowego porządku $P$~krytycznych ze względu na skojarzenie Morse'a $M$} oraz, o~ile $P$~jest porządkiem z~gradacją, niech \[\mcR_n^M(P)=\left\{[r]\in\mcR^M(P):\Ht(r)=n\right\}.\]\nomenclature[5f]{$\mcR_n^M(P)$}{rodzina klas równoważności promieni malejących rangi $n$~w~grafie $\mH_M(P)$}

Niech $D$~będzie grafem skierowanym oraz niech \[\mfR\subseteq\{[r]:r\text{ jest promieniem malejącym w } D\}.\] Rodzinę $\left\{r^\mfr\right\}_{\mfr\in\mfR}$ promieni malejących w~$D$~taką, że $[r^\mfr]=\mfr$ dla każdego $\mfr\in \mfR$, nazywamy \textit{rodziną reprezentującą zbiór} $\mfR$.\index{rodzina reprezentujazzzca@rodzina reprezentująca}

Jeżeli $P$~jest częściowym porządkiem z~gradacją, zaś $M$~jest skojarzeniem Morse'a na $P$, to rodzinę $\left\{r^\mfr=\bigl(r^\mfr_i\right)_{i\in\mN}\bigr\}_{\mfr\in\mfR}$ reprezentującą zbiór $\mfR\subseteq \mcR^M(P)$ nazywamy \textit{wspaniałą}\index{rodzina reprezentujazzzca@rodzina reprezentująca!wspanialzzza@wspaniała}\index{wspanialzzza rodzina reprezentujazzzca@wspaniała rodzina reprezentująca}, o~ile dla wszystkich $\mfr,\mfr'\in\mfR$ takich, że $\mfr\not=\mfr'$, spełnione są następujące warunki:
\begin{compactitem}
\item[---] promień malejący $r^\mfr$ nie ma obwodnic;
\item[---] nie istnieje ścieżka prosta w~$\mH_M(P)$ prowadząca z~elementu należącego do $r^{\mfr}$ do elementu należącego do $r^{\mfr'}$ (w~szczególności $r^{\mfr}$ nie ma elementów wspólnych z~$r^{\mfr'}$);
\item[---] $\Ht\left(r^\mfr_0\right)=\Ht(r^\mfr)$.
\end{compactitem}

Szczególnie istotny dla~dalszej części rozprawy będzie przypadek, gdy zbiór $\mfR$~jest jednoelementowy.

\begin{lem}\label{istnieje_rodzina_wspaniala_dla_jednoelementowego}
Niech $P$ będzie częściowym porządkiem z~gradacją oraz z~zadanym skojarzeniem Morse'a $M$. Dla każdego promienia malejącego $r$~w~$\mH_{M}(P)$ nie będącego multipromieniem istnieje wspaniała rodzina reprezentująca zbiór $\{[r]\}$.
\end{lem}
\begin{proof}
Ustalmy promień malejący $r$~w~$\mH_M(P)$ nie będący multipromieniem. Dla odpowiednio dużego $n\in\mN$ równoważny $r$~promień malejący $r^*=(r^*_{i})_{i\in\mN}=(r_{n+i})_{i\in\mN}$ nie ma obwodnic oraz $\Ht(r)=\Ht(r^*)=\Ht(r^*_0)$. Jednoelementowy zbiór $\{r^*\}$ stanowi wspaniałą rodzinę reprezentującą $\{[r]\}$.
\end{proof}

Wspaniałą rodzinę reprezentującą $\mfR$ nietrudno także znaleźć dla dowolnego zbioru $\mfR\subseteq \mcR^M(\mP(X))$, gdzie $X$~jest $1$-wymiarowym, regularnym CW kompleksem.

\begin{lem}\label{istnieje_rodzina_wspaniala_dla_grafu}
Niech $X$~będzie regularnym, $1$-wymiarowym CW kompleksem, zaś $M$~niech będzie skojarzeniem Morse'a na $X$. Dla każdego zbioru $\mfR\subseteq \mcR^M(\mP(X))$ istnieje wspaniała rodzina reprezentująca $\mfR$.
\end{lem}
\begin{proof}
Zauważmy, że dla każdego promienia malejącego $r$~w~$\mH_{M}(\mP(X))$ zachodzi równość $\Ht(r)=0$. Ponadto, jeżeli $x\in \mP(X)$, $\Ht(x)=1$, to zbiór $\hat{x}\mathord{\downarrow}_{\mP(X)}$ jest dwuelementowy.

Ustalmy zbiór $\mfR\subseteq \mcR^M(\mP(X))$ i~wybierzmy dowolną reprezentującą $\mfR$~rodzinę $\bigl\{r^\mfr=\left(r^\mfr_i\right)_{i\in\mN}\bigr\}_{\mfr\in\mfR}$ promieni malejących w~$\mH_M{\mP(X)})$ o~tej własności, że $0=\Ht(r^\mfr)=\Ht\left(r^\mfr_0\right)$ dla każdego $\mfr\in\mfR$. Wykażemy, że rodzina ta jest wspaniała.

Niech $\mfr\in \mfR$ oraz niech $(c_0,c_1)$ będzi krawędzią grafu $\mH_M(\mP(X))$. Załóżmy, że $c_0=r^\mfr_{i_0}$ dla pewnej liczby naturalnej $i_0$. Nietrudno zauważyć, np.~rozpatrując osobno przypadki $\Ht(c_0)=0$, $\Ht(c_0)=1$, a~przy rozważaniu drugiego z~nich mając na uwadze, że $c_0\not=r^\mfr_0$, zaś zbiór $\hat{c}_0\mathord{\downarrow}_{\mP(X)}$ jest dwuelementowy, że istnieje dokładnie jedna krawędź w~$\mH_M(\mP(X))$ wychodząca z~$c_0$ i~jest nią $\bigl(r^\mfr_{i_0}, r^\mfr_{i_0+1}\bigr)$, czyli $c_1=r^\mfr_{i_0+1}$. Zatem każda skończona ścieżka prosta w~$\mH_M(\mP(X))$ rozpoczynająca się w~elemencie promienia malejącego $r^\mfr$ jest postaci $\bigl(r^\mfr_i,r^\mfr_{i+1},\ldots,r^\mfr_{i+n}\bigr)$ dla pewnych $i,n\in\mN$. 

Wobec powyższej obserwacji żaden z~promieni malejących $r^\mfr$, $\mfr\in\mfR$, nie ma obwodnicy i~dla żadnych $\mfr_0, \mfr_1\in\mfR$ takich, że $\mfr_0\not=\mfr_1$, nie istnieje ścieżka prosta w~$\mH_M(P)$~prowadząca z~elementu należącego do $r^{\mfr_0}$ do elementu należącego do $r^{\mfr_1}$. Rodzina $(r^{\mfr})_{\mfr\in\mfR}$ reprezentująca zbiór $\mfR$~jest więc wspaniała.
\end{proof}

Poniższy lemat pozwala zmodyfikować dane skojarzenie Morse'a poprzez ,,odwrócenie'' promieni malejących tworzących wspaniałą rodzinę reprezentującą dany zbiór klas równoważności promieni malejących. W~wyniku tej operacji promienie malejące zostają przekształcone w~promienie rosnące, a~jednocześnie powstaje po jednym nowy elemencie krytycznym dla każdego z~tych promieni. Technikę ,,odwracania promieni'' wykorzystaną w~poniższym dowodzie stosowali wcześniej Ayala, Fern{\'a}ndez i~Vilches~\cite{Ayala10}, a~jej korzenie można znaleźć w~artykule Formana \cite{Forman98}.

\begin{lem}\label{lem-wspaniali_reprezentanci_sa_odwracalni}
Niech $P$~będzie częściowym porządkiem z~gradacją i~z~zadanym skojarzeniem Morse'a $M'$, zaś $\left\{r^\mfr=(r^\mfr_i)_{i\in\mN}\right\}_{\mfr\in\mfR}$ niech będzie rodziną reprezentującą zbiór $\mfR\subseteq \mcR^{M'}(P)$. Jeżeli rodzina ta jest wspaniała, to zbiór
\[M=\left(M'\smallsetminus \left\{\left(r^\mfr_{2k+1}, r^\mfr_{2k}\right)\right\}_{\begin{subarray}{l}k\in\mN,\\ \mfr\in\mfR\end{subarray}}\right) \cup \left\{\left(r^\mfr_{2k+1}, r^\mfr_{2k+2}\right)\right\}_{\begin{subarray}{l}k\in\mN,\\ \mfr\in\mfR\end{subarray}}\]
jest skojarzeniem Morse'a na $P$~o~tej własności, że 
\[\mcC^{M}(P)=\mcC^{M'}(P)\cup \left\{r^{\mfr}_0:\mfr\in\mfR\right\},\] przy czym $r^{\mfr}_0\in P\smallsetminus \mcC^{M'}(P)$, $r^{\mfr}_0\not=r^{\mfr'}_0$ dla $\mfr\not=\mfr'$ oraz $\Ht(r_0^{\mfr})=\Ht(r^{\mfr})$ dla wszystkich $\mfr,\mfr'\in\mfR$. 
\end{lem}
\begin{proof}
Załóżmy, że rodzina $\left\{r^\mfr\right\}_{\mfr\in\mfR}$ reprezentująca zbiór $\mfR$ jest wspaniała. Ponieważ $\Ht(r^\mfr_0)=\Ht(r^\mfr)$ dla wszystkich $\mfr\in\mcR^{M'}(P)$, $r^\mfr_0$~jest minimalnym elementem zbioru wyrazów ciągu~$r^\mfr$; zatem $(r^\mfr_0,r^\mfr_1)\not\in\mH(P)$. Ale $(r^\mfr_0,r^\mfr_1)\in\mH_{M'}(P)$, więc $(r^\mfr_1,r^\mfr_0)\in M'$. Ponieważ $M'$~jest skojarzeniem, $(r^\mfr_2,r^\mfr_1)\not\in M'$, i~w~konsekwencji $(r^\mfr_1,r^\mfr_2)\in\mH(P)$, więc $\Ht(r^\mfr_2)=\Ht(r^\mfr)$. Podobnie, dla wszystkich $k\in\mN$ krawędzie $\left(r^\mfr_{2k+1}, r^\mfr_{2k}\right)$ należą do skojarzenia $M'$. 

Korzystając z~powyższej obserwacji i~założenia, że promienie malejące $r^\mfr$, $r^{\mfr'}$~są rozłączne dla różnych $\mfr,\mfr'\in\mfR$, łatwo jest zauważyć, że $M$~jest skojarzeniem w~grafie skierowanym $\mH(P)$.

Wykażemy, że graf skierowany~$\mH_{M}(P)$ nie zawiera cykli. W~tym celu udowodnimy najpierw, że jeśli $\mathfrak{A}\subseteq \mathfrak{R}$ jest skończonym zbiorem, to 
\[M_{\mathfrak{A}}=\left(M'\smallsetminus \left\{\left(r^\mfr_{2k+1}, r^\mfr_{2k}\right)\right\}_{\begin{subarray}{l}k\in\mN,\\ \mfr\in\mathfrak{A}\end{subarray}}\right) \cup \left\{\left(r^\mfr_{2k+1}, r^\mfr_{2k+2}\right)\right\}_{\begin{subarray}{l}k\in\mN,\\ \mfr\in\mathfrak{A}\end{subarray}}\]
jest skojarzeniem Morse'a na $P$. (Oczywiście $\left\{r^\mfr=(r^\mfr_i)_{i\in\mN}\right\}_{\mfr\in\mathfrak{A}}$ jest wspaniałą rodziną reprezentującą $\mathfrak{A}$, zatem z~pierwszej części dowodu wiemy, że $M_{\mathfrak{A}}$ jest skojarzeniem w~grafie skierowanym $\mH(P)$.)

Przeprowadzimy indukcję ze względu na liczbę elementów zbioru $\mathfrak{A}$. Jeżeli $\mathfrak{A}=\emptyset$, to $M_{\mathfrak{A}}=M'$ jest skojarzeniem Morse'a. Załóżmy, że $M_{\mathfrak{B}}$ jest skojarzeniem Morse'a dla wszystkich $\mathfrak{B}\subseteq \mfR$ takich, że $\moc{\mathfrak{B}}<\moc{\mathfrak{A}}$. Ustalmy $\mathfrak{a}\in\mathfrak{A}$. Przypuśćmy, że graf skierowany $\mH_{M_\mathfrak{A}}(P)$ zawiera cykl $c=(c_0,c_1,\ldots,c_m)$. Ponieważ $M_{\mathfrak{A}\smallsetminus \{\mathfrak{a}\}}$ jest z~założenia indukcyjnego skojarzeniem Morse'a, muszą istnieć elementy następujące kolejno po sobie w~cyklu $c$~(przez takie elementy rozumiemy również $c_m,c_0$), które należą do $r^{\mathfrak{a}}$; w~przeciwnym wypadku $c$~byłoby cyklem w~$\mH_{M_{\mathfrak{A}\smallsetminus \{\mathfrak{a}\}}}(P)$. Możemy założyć (ewentualnie zmieniając numerację elementów cyklu $c$), że $c_m,c_0\in r^{\mathfrak{a}}$. Z~drugiej strony, musi istnieć element cyklu $c$~nie należący do $r^{\mathfrak{a}}$; w~przeciwnym wypadku ciąg $(c_m,c_{m-1},\ldots,c_0)$ byłby cyklem w~$\mH_{M'}(P)$. Niech $i_a, i_b\in\mN$ będą takie, że $0\leq i_a<i_a+1<i_b\leq m$, $c_{i}\not\in r^{\mathfrak{a}}$ dla $i_a< i< i_b$ oraz $c_{i_a},c_{i_b}\in r^{\mathfrak{a}}$. Wówczas $c_{i_a}=r^{\mathfrak{a}}_{a}, c_{i_b}=r^{\mathfrak{a}}_{b}$ dla pewnych $a,b\in\mN$. Oczywiście $a\not=b$, gdyż $r^{\mathfrak{a}}_a=c_{i_a}\not=c_{i_b}=r^{\mathfrak{a}}_b$. Gdyby $a>b$, to ciąg $\bigl(c_{i_a},\ldots, c_{i_b},r^{\mathfrak{a}}_{b+1},\ldots,r^{\mathfrak{a}}_{a-1}\bigr)$ byłby cyklem w~$\mH_{M_{\mathfrak{A}\smallsetminus \{\mathfrak{a}\}}}(P)$, co jest niemożliwe. Zatem $a<b$. Ale to oznacza, że $\bigl(c_{i_a},\ldots, c_{i_b}\bigr)$ jest obwodnicą $r^{\mathfrak{a}}$~w~$\mH_{M_{\mathfrak{A}\smallsetminus \{\mathfrak{a}\}}}(P)$. Ponieważ $r^{\mathfrak{a}}$ z~założenia nie ma obwodnic w~$\mH_{M'}(P)$, istnieje \[i_0=\min\left\{i_a<i<i_b:c_i\in r^{\mathfrak{a}'}\text{ dla pewnego } \mathfrak{a}'\in\mathfrak{A}\smallsetminus\mathfrak{a}\right\}.\] Wobec tego $\bigl(c_{i_a},\ldots,c_{i_0}\bigr)$ jest ścieżką w~$\mH_{M'}(P)$~prowadzącą z~elementu ciągu $r^{\mathfrak{a}}$~do elementu ciągu $r^{\mathfrak{a}'}$. Jest to sprzeczne ze wspaniałością rodziny $\{r^{\mfr}\}_{\mfr\in\mathfrak{A}}$.

Wykazaliśmy, że dla każdego skończonego zbioru $\mathfrak{A}\subseteq\mfR$ graf skierowany $\mH_{M_\mathfrak{A}}(P)$ nie zawiera cykli, czyli $M_{\mathfrak{A}}$~jest skojarzeniem Morse'a.

Ustalmy skończony ciąg $c=(c_0,c_1,\ldots,c_m)$ wierzchołków~$\mH_{M}(P)$. Udowodnimy, że nie jest on cyklem. Zauważmy, że zbiór $\{c_0,\ldots,c_m\}$ ma niepuste przekroje ze zbiorami wyrazów~co najwyżej $m+1$~różnych promieni malejących z~rodziny $\{r^\mfr\}_{\mfr\in\mfR}$; powiedzmy że są to promienie $r^{\mfr_1},\ldots,r^{\mfr_n}$ dla pewnego $n\leq m+1$. Niech $\mathfrak{A}=\{\mfr_1,\ldots,\mfr_n\}$. Gdyby $c$~było cyklem w~$\mH_M(P)$, to byłoby też cyklem w~$\mH_{M_{\mathfrak{A}}}(P)$. To jednak, jak wykazaliśmy, jest niemożliwe. Zatem $M$~jest skojarzeniem Morse'a na $P$.

Nietrudno zauważyć, że $\mcC^M(P)=\mcC^{M'}(P)\cup \left\{r_0^\mfr:\mfr\in\mfR\right\}$. Bezpośrednio z~założeń lematu wynika, że $r^{\mfr}_0\in P\smallsetminus \mcC^{M'}(P)$, $r^{\mfr}_0\not=r^{\mfr'}_0$ dla $\mfr\not=\mfr'$ oraz $\Ht(r_0^{\mfr})=\Ht(r^{\mfr})$ dla wszystkich $\mfr,\mfr'\in\mfR$.
\end{proof}

Przy ,,odwracaniu'' promieni malejących ze wspaniałej rodziny reprezentującej zbiór $\mfR\subseteq\mcR^M(P)$ nie powstają nowe promienie malejące.

\begin{lem}\label{lem-wspaniali_reprezentanci_nie_tworza_promieni}
Niech $P$~będzie częściowym porządkiem z~gradacją i~z~zadanym skojarzeniem Morse'a $M'$, zaś $\left\{r^\mfr=(r^\mfr_i)_{i\in\mN}\right\}_{\mfr\in\mfR}$ niech będzie rodziną reprezentującą zbiór $\mfR\subseteq \mcR^{M'}(P)$. Załóżmy, że rodzina ta jest wspaniała, zaś $M$~jest skojarzeniem Morse'a na $P$~zadanym wzorem 
\[M=\left(M'\smallsetminus \left\{\left(r^\mfr_{2k+1}, r^\mfr_{2k}\right)\right\}_{\begin{subarray}{l}k\in\mN,\\ \mfr\in\mfR\end{subarray}}\right) \cup \left\{\left(r^\mfr_{2k+1}, r^\mfr_{2k+2}\right)\right\}_{\begin{subarray}{l}k\in\mN,\\ \mfr\in\mfR\end{subarray}}.\]
Wówczas dla każdego promienia malejącego $s$~zawartego w~$\mH_{M'}(P)$ i~takiego, że $[s]\not\in \mfR$, istnieje promień malejący $\tilde{s}\in [s]$ w~$\mH_{M'}(P)$ będący również promieniem malejącym w~$\mH_M(P)$, a~odwzorowanie $\mcR^{M'}(P)\smallsetminus \mfR \to\mcR^M(P)$ zadane przez $[s]\mapsto[\tilde{s}]$
 jest bijekcją.
\end{lem}
\begin{proof}
Ustalmy promień malejący $s=(s_i)_{i\in\mN}$ w~grafie $\mH_{M'}(P)$ taki, że $[s]\not\in\mfR$. Zdefiniujemy promień malejący $\tilde{s}$. 

Jeżeli zbiór elementów promienia malejącego $s$~jest rozłączny ze zbiorem elementów~każdego z~promieni malejących $r^\mfr, \mfr\in\mfR$, to $s$~jest promieniem malejącym w~$\mH_M(P)$ i~możemy przyjąć $\tilde{s}=s$. 

Jeśli natomiast $s$~ma elementy wspólne z~jakimś promieniem należącym do rodziny $\left\{r^\mfr\right\}_{\mfr\in\mfR}$, to promień taki jest dokładnie jeden. Jeśli bowiem $s_{i_0}\in r^{\mfr_0}$, $s_{i_1}\in r^{\mfr_1}$ dla pewnych $\mfr_0,\mfr_1\in\mfR$ oraz $i_0<i_1$, to $\bigl(s_{i_0}, s_{i_0+1},\ldots,s_{i_1}\bigr)$ jest ścieżką prostą w~$\mH_{M'}(P)$ prowadzącą z~elementu promienia $r^{\mfr_0}$~do elementu promienia $r^{\mfr_1}$. Wobec założenia o~wspaniałości rodziny $\left\{r^\mfr\right\}_{\mfr\in\mfR}$ zachodzi równość $\mfr_0=\mfr_1$.

Załóżmy, że dla pewnego $\mfr_*\in\mfR$ promień malejący $s$~ma elementy wspólne z~promieniem $r^{\mfr_*}$. Wobec lematu \ref{lem-sk_przekroj_promieni_nierownowaznych} jeśli zbiór elementów należących zarówno do $r^{\mfr_*}$ jak i~do $s$~jest nieskończony, to $\mfr_*=[r^{\mfr_*}]=[s]$, co jest sprzeczne z~wyborem $s$. Zatem dla odpowiednio dużego $n_0\in\mN$ promień malejący $\tilde{s}=(s_{n_0+i})_{i\in\mN}\in [s]$ jest rozłączny z~$r^{\mfr_*}$. Ciąg $\tilde{s}$ jest rozłączny ze wszystkimi promieniami z~rodziny $\left\{r^\mfr\right\}_{\mfr\in\mfR}$, jest więc też promieniem malejącym w~$\mH_{M}(P)$. 

Wobec tego istnieje odwzorowanie $\mcR^{M'}(P)\smallsetminus \mfR\to \mcR^M(P)$ zadane dla $[s]\in\mcR^{M'}(P)\smallsetminus \mfR$ przez $[s]\mapsto [\tilde{s}]$. Jest ono oczywiście różnowartościowe. Pokażemy, że jest też ,,na''.

W~tym celu ustalmy promień malejący $t=(t_i)_{i\in\mN}$ w~$\mH_{M}(P)$. Jeżeli zbiór wyrazów ciągu $t$~jest rozłączny ze zbiorem wyrazów~każdego spośród~promieni malejących należących do $\left\{r^\mfr\right\}_{\mfr\in\mfR}$, to $t=\tilde{t}$ jest promieniem malejącym w~$\mH_{M'}(P)$ oraz$[t]=[\tilde{t}]$.

Jeśli natomiast istnieje promień malejący z~rodziny $\left\{r^\mfr\right\}_{\mfr\in\mfR}$, który ma wspólne elementy z~$t$, to jest on tylko jeden. (W~przeciwnym wypadku istniałyby $i_0<i_1$ takie, że $t_{i_0}, t_{i_1}$ należałyby do różnych promieni z~rodziny $\left\{r^\mfr\right\}_{\mfr\in\mfR}$, zaś $t_i$ nie należałoby do żadnego promienia z~tej rodziny dla wszystkich $i_0<i<i_1$. Ciąg $\bigl(t_{i_0},t_{i_0+1},\ldots, t_{i_1}\bigr)$ byłby wówczas ścieżką prostą między elementami różnych promieni z~$\left\{r^\mfr\right\}_{\mfr\in\mfR}$, co wobec założenia o~wspaniałości tej rodziny jest wykluczone.)

Załóżmy, że dla pewnego $\mfr_\star\in\mfR$ promień $t$~ma wspólny element z~promieniem malejącym $r^{\mfr_\star}$. Udowodnimy, że zbiór wspólnych elementów tych ciągów jest skończony. Będzie to oznaczało, że istnieje równoważny $t$~promień malejący $t'$~w~$\mH_{M}(P)$ rozłączny z~$r^{\mfr_\star}$. Taki promień $t'$~jest również promieniem malejącym w~$\mH_{M'}(P)$ i~mamy $\left[\tilde{t'}\right]=[t']=[t]$ w~$\mH_M(P)$, a~zatem funkcja $[s]\mapsto [\tilde{s}]$ jest ,,na''.

Przypuśćmy, że $t$~oraz $r^{\mfr_\star}$~mają nieskończenie wiele wspólnych elementów. Istnieją wówczas różnowartościowe ciągi liczb naturalnych $(i_n)_{n\in\mN}$ oraz $(j_n)_{n\in\mN}$ takie, że $t_{i_n}=r^{\mfr_\star}_{j_n}$ dla wszystkich $n\in\mN$. W~szczególności istnieją $a,b\in\mN$ o~tej własności, że $i_a<i_b$ oraz $j_a<j_b$, a~zatem istnieją w~grafie $\mH_{M}(P)$ ścieżki proste $\bigl(r^{\mfr_\star}_{j_a}=t_{i_a},t_{i_a+1},\ldots, t_{i_b}=r^{\mfr_\star}_{j_b}\bigr)$ oraz $\bigl(t_{i_b}=r^{\mfr_\star}_{j_b},r^{\mfr_\star}_{j_b-1},\ldots,r^{\mfr_\star}_{j_a}=t_{i_a}\bigr)$. Oznacza to, że w~grafie skierowanym $\mH_{M}(P)$ istnieje cykl; jest to sprzeczne faktem, że~$M$ jest skojarzeniem Morse'a. Zbiór wspólnych elementów ciągów $t$~oraz $r^{\mfr_\star}$~jest więc skończony, co wobec wcześniejszych rozważań kończy dowód lematu.
\end{proof}

\begin{stw}\label{reversing_stw}
Niech $P$ będzie częściowym porządkiem z~gradacją, na którym zadane jest skojarzenie Morse'a $M'$ takie, że zbiór $\mcR^{M'}(P)$ jest skończony. Wówczas istnieje skojarzenie Morse'a $M$ na $P$ bez promieni malejących i~o~tej własności, że \[\mcC^{M}(P)=\mcC^{M'}(P)\cup \left\{c_{[r]}:[r]\in\mcR^{M'}(P)\right\},\] gdzie $c_{[r]}\in P\smallsetminus \mcC^{M'}(P)$, $c_{[r]}\not=c_{[r']}$ dla $[r]\not=[r']$ oraz $\Ht\bigl(c_{[r]}\bigr)=\Ht(r)$ dla wszystkich $[r],[r']\in\mcR^{M'}$. 
\end{stw}
\begin{proof}
Dowód prowadzimy metodą indukcji matematycznej ze względu na liczbę klas równoważności promieni malejących. Ponieważ liczba ta jest skończona, na podstawie~lematu \ref{prop_multiray} graf skierowany $\mcH_{M'}(P)$ nie zawiera multipromieni, a~w~tej sytuacji lematy \ref{istnieje_rodzina_wspaniala_dla_jednoelementowego}, \ref{lem-wspaniali_reprezentanci_sa_odwracalni} oraz~\ref{lem-wspaniali_reprezentanci_nie_tworza_promieni} pozwalają przekształcić dane skojarzenie Morse'a przez ,,odwrócenie'' jednego z~promieni i~tym samym zmniejszyć o~jeden liczbę klas równoważności promieni malejących.
\end{proof}

Poniższe stwierdzenie stanowi natychmiastowy wniosek z~lematu \ref{istnieje_rodzina_wspaniala_dla_grafu} (dla~$\mfR=\mcR^{M'}(\mP(X))$) oraz lematów \ref{lem-wspaniali_reprezentanci_sa_odwracalni}, \ref{lem-wspaniali_reprezentanci_nie_tworza_promieni}.

\begin{stw}\label{reversing_stw2}
Niech $X$ będzie $1$-wymiarowym, regularnym CW kompleksem z~zadanym skojarzeniem Morse'a $M'$. Istnieje wówczas skojarzenie Morse'a $M$ na $X$ bez promieni malejących i~takie, że \[C_M(\mP(X))=C_{M'}(\mP(X))\cup\left\{c_{[r]}:[r]\in\mcR^{M'}(\mP(X))\right\},\] gdzie $c_{[r]}\in\mP(X)\smallsetminus C_{M'}(\mP(X))$, $c_{[r]}\not=c_{[r']}$ dla $[r]\not=[r']$ oraz $\Ht\bigl(c_{[r]}\bigr)=0$ dla wszystkich $[r],[r']\in\mcR^{M'}$.
\end{stw}

\begin{comment}
Przydatna okaże się również możliwość ,,odwrócenia'' promienia rosnącego kosztem utworzenia promienia malejącego i~elementu krytycznego. Ponieważ odwracanie promieni rosnących ma dla niniejszej rozprawy mniejsze znaczenie niż odwracanie promieni malejących, nie będziemy tworzyć narzędzi w~rodzaju lematów \ref{lem-wspaniali_reprezentanci_sa_odwracalni}, \ref{lem-wspaniali_reprezentanci_nie_tworza_promieni}. Udowodnimy jedynie twierdzenie pozwalające na odwrócenie pojedynczego promienia rosnącego. Wprowadźmy jednak najpierw kilka definicji.

% \reflectbox{$\mcR$}$
Jeżeli $r=(r_i)_i\in\mN$~oraz $s=(s_i)_i\in\mN$ są promieniami rosnącymi w~pewnym grafie skierowanym $D$,to mówimy, że promienie te są \textit{równoważne}, jeśli $s_i=r_i$ dla wszystkich odpowiednio dużych $i\in\mN$. 

Dla skojarzenia Morse'a $M$~na częściowym porządku $P$~symbolem \reflectbox{$\mcR$}$_M(P)$ będziemy oznaczać zbiór klas równoważności promieni rosnących zawartych w~grafie skierowanym $\mH_M(P)$.

Możemy teraz przejść do sformułowania wspomnianego twierdzenia.

\begin{tw}
Niech $M'$~będzie skojarzeniem Morse'a na częściowym porządku z~rangą $P$~o~tej własności, że $(r_i)_{i\in\mN}$ jest promieniem rosnącym w~$\mH_{M'}(P)$, zbiór \reflectbox{$\mcR$}$_{M'}(P)$ jest jednoelementowy. Istnieje wówczas skojarzenie Morse'a $M$~na $P$~takie, że \reflectbox{$\mcR$}$_{M}(P)=\emptyset$ oraz $\mcC_{M}(P)=\mcC_{M'}(P)\cup\{r_1\}$  
\end{tw}
\end{comment}

Naturalne wydaje się przypuszczenie, że możliwy jest korzystający z~zasady indukcji pozaskończonej dowód wyniku analogicznego do stwierdzenia \ref{reversing_stw} w~sytuacji, gdy $\mcH_{M'}(P)$ nie zawiera multipromieni, lecz klas równoważności promieni malejących w~tym grafie jest nieskończenie wiele. Jednakże dla granicznych liczb porządkowych granicznych przy tym podejściu problem: w~wyniku procesu ,,odwrócenia'' nieskończenie wielu promieni malejących może powstać nowy taki promień. Sytuację taką ilustruje rysunek~\ref{fig-nie_mozna_odwrocic_niesk_wielu}.

\begin{figure}[h]
\[
\xymatrix@C=16pt@R=9pt{
r^0: & & & & \bullet\ar[dr] & & \bullet\ar[dr]& & \bullet\ar[dr]\ar[ddd]  & & \bullet\ar[dr] & & \dots \\
& & & \bullet\ar@{-->}[ur] & & \bullet\ar@{-->}[ur] & & \bullet\ar@{-->}[ur] & & \bullet\ar@{-->}[ur] & & \bullet\ar@{-->}[ur]\\
r^1: & & & \bullet\ar[dr] & & \bullet\ar[dr] & & \bullet\ar[dr]\ar[ddd] & & \bullet\ar[dr] & & \dots \\
& & \bullet\ar@{-->}[ur] & & \bullet\ar@{-->}[ur] & & \bullet\ar@{-->}[ur] & & \bullet\ar@{-->}[ur] & & \bullet\ar@{-->}[ur]\\
r^2: & & \bullet\ar[dr] & & \bullet\ar[dr] & & \bullet\ar[dr]\ar[dd]  & & \bullet\ar[dr] & & \dots \\
& \bullet\ar@{-->}[ur] & & \bullet\ar@{-->}[ur] & & \bullet\ar@{-->}[ur] & & \bullet\ar@{-->}[ur] & & \bullet\ar@{-->}[ur]\\
\vdots & & & & & & \vdots
}
\]
\caption{Diagram Hassego pewnego częściowego porządku z~gradacją po zmianie orientacji krawędzi (zaznaczonych linią przerywaną) należących do skojarzenia Morse'a na tym porządku. Odwrócenie wszystkich promieni z~rodziny $\left(r^i\right)_{i\in\mN}$ powoduje powstanie nowego promienia malejącego (przechodzącego przez pionowe strzałki).}\label{fig-nie_mozna_odwrocic_niesk_wielu}
\end{figure}

Oczywiście w~sytuacji z~rysunku \ref{fig-nie_mozna_odwrocic_niesk_wielu} nietrudno wybrać wspaniałą rodzinę reprezentującą zbiór $\left\{\left[r^i\right]\right\}_{i\in\mN}$,  ale przykład ten można zmodyfikować, czyniąc go bardziej ,,złośliwym''. Przykładowo, pomiędzy promieniami $r^{2k}$, $r^{2k+1}$ dla $k\in\mN$ możemy dorysować nieskończenie wiele pionowych strzałek. W~tej sytuacji wybór wspaniałej rodziny reprezentującej zbiór $\left\{\left[r^i\right]\right\}_{i\in\mN}$ będzie niemożliwy. Jednakże operację ,,odwracania promieni'' można w~tej sytuacji nadal przeprowadzić bez tworzenia nowego promienia malejącego. Gdy nieskończenie wiele strzałek dorysujemy pomiędzy wszystkimi parami promieni $r^{i}, r^{i+1}$, $i\in\mN$, staje się to niemożliwe. 

\begin{problem}\label{prob5}
Niech $P$~będzie częściowym porządkiem z~gradacją oraz~zadanym skojarzeniem Morse'a $M'$~takim, że zbiór $\mcR^{M'}(P)$ jest nieskończony. 
Przy jakich założeniach o~$P$~oraz $M'$~możliwe jest uzyskanie na $P$~skojarzenia Morse'a $M$~bez promieni malejących i~o~tej własności, że \[\mcC^{M}(P)=\mcC^{M'}(P)\cup \left\{c_{[r]}:[r]\in\mcR^{M'}(P)\right\},\] gdzie $c_{[r]}\in P\smallsetminus \mcC^{M'}(P)$, $c_{[r]}\not=c_{[r']}$ dla $[r]\not=[r']$ oraz $\Ht\bigl(c_{[r]}\bigr)=\Ht(r)$ dla wszystkich $[r],[r']\in\mcR^{M'}$? 
\end{problem}

Przez $\mathcal{RG}\bigl(\left\{r^i\right\}_{i\in I}\bigr)$ dla rodziny promieni malejących $\left\{r^i\right\}_{i\in I}$ zawartych w~grafie skierowanym $D$~oznaczmy graf skierowany o~zbiorze wierzchołków $\left\{r^i\right\}_{i\in I}$ taki, że krawędź $\left(r^i,r^j\right)\in \mathcal{RG}\bigl(\left\{r^i\right\}_{i\in I}\bigr)$ wtedy i~tylko wtedy, gdy $i\not=j$ oraz istnieje ścieżka  w~$D$~prowadząca z~elementu należącego do $r^i$ do elementu należącego do~$r^j$.

Wydaje się, że jeśli w~definicji wspaniałej rodziny reprezentującej $\mfR\subseteq \mcR^M(P)$ zastąpimy warunek mówiący o~nieistnieniu ścieżek pomiędzy różnymi promieniami z tej rodziny warunkiem następującym:
\begin{compactitem}
\item[---]promienie malejące $r^{\mfr_0},r^{\mfr_1}$ są rozłączne dla $\mfr_0\not=\mfr_1\in\mfR$ oraz graf skierowany $\mathcal{RG}(\{r^\mfr\}_{\mfr\in\mfR})$ nie zawiera cykli,
\end{compactitem}
to możliwy będzie dowód wyniku analogicznego do lematu \ref{lem-wspaniali_reprezentanci_sa_odwracalni}. Jeśli zaś założymy dodatkowo, że
\begin{compactitem}
\item[---]graf skierowany $\mathcal{RG}(\{r^\mfr\}_{\mfr\in\mfR})$ nie zawiera promieni malejących,
\end{compactitem}
to prawdziwy będzie odpowiednik lematu \ref{lem-wspaniali_reprezentanci_nie_tworza_promieni}. Nie widząc jednak w~tej chwili eleganckich zastosowań dla tych wyników, nie chcemy się wdawać w~ich dowody, pozostając przy słabszych, ale prostszych lematach \ref{lem-wspaniali_reprezentanci_sa_odwracalni}, \ref{lem-wspaniali_reprezentanci_nie_tworza_promieni}.

Autor przypuszcza, że ciekawymi obiektami badań, obok grafu $\mathcal{RG}\bigl(\left\{r^i\right\}_{i\in I}\bigr)$, mogą okazać się kompleksy symplicjalne (czy ogólniejsze typy kompleksów) o~zbiorze wierzchołków $\left\{r^i\right\}_{i\in I}$ oraz~sympleksach wyznaczanych przez (różnego rodzaju) ścieżki pomiędzy promieniami z~tego zbioru.

%==============================================================
%==============================================================
%==============================================================

\section{Topologiczna wersja dyskretnej teorii Morse'a}\label{homotopical}
W~niniejszym podrozdziale udowodnimy wersję głównego twierdzenia dyskretnej teorii Morse'a (twierdzenie \ref{tw-glowne_tw_klasycznej_dyskretnej_teorii_morsea}) prawdziwą dla odpowiednio ,,dobrych'' skojarzeń Morse'a na niezwartych \mbox{CW kompleksach} oraz na tzw.~\mbox{h-regularnych} nieskończonych częściowych porządkach, wprowadzonych przez Miniana \cite{Minian12}.

\subsection{Porządki h-regularne i~dopuszczalne}
Definicje podane w~niniejszej sekcji mają swój pierwowzór w~pracy Miniana \cite{Minian12}, w~której sformułowane zostały dla skończonych częściowych porządków. Poniżej przedstawiamy je zaadaptowane do porządków nieskończonych, w~nieco ogólniejszej formie niż w~artykule autora \cite{Kukiela13}.

Częściowy porządek $P$ nazywamy \textit{h-regularnym}\index{czezzzszzzciowy porzazzzdek@częściowy porządek!h-regularny}, o~ile $P$~jest porządkiem z~rangą oraz dla każdego $x\in P$ kompleks symplicjalny $\mcK(\hat{x}\mathord{\downarrow})$ jest homotopijnie równoważny sferze $\S^{\Ht(x)-1}$. (Przez sferę $(-1)$-wymiarową rozumiemy zbiór pusty.)

Jeżeli $P$~jest częściowym porządkiem z~rangą, to krawędź $(x,y)\in\mcH(P)$  nazywamy \textit{dopuszczalną}\index{krawezzzdzzzz dopuszczalna@krawędź dopuszczalna}, gdy kompleks $\mcK(\hat{x}\mathord{\downarrow}\smallsetminus\{y\})$ jest ściągalny. Mówimy, że częściowy porządek z~rangą jest \textit{dopuszczalny}\index{czezzzszzzciowy porzazzzdek@częściowy porządek!dopuszczalny}, jeśli wszystkie krawędzie jego diagramu Hassego są dopuszczalne.

Skojarzenie Morse'a $M$~na h-regularnym częściowym porządku $P$ nazywamy \textit{dopuszczalnym}\index{skojarzenie Morse'a!dopuszczalne}, o~ile każda krawędź grafu skierowanego $\mH(P)$ należąca do $M$~jest dopuszczalna.

\begin{ex}[{\cite[Remark 2.6]{Minian12}}]\label{ex-regularny_kompleks_daje_dopuszczalny_porzadek}
Jeżeli $X$ jest regularnym CW kompleksem, to częściowy porządek $\mP(X)$ jest dopuszczalny.
\end{ex}

\begin{ex}[{\cite[Subsection 4.1]{Minian12}}]
Niech $X$~będzie CW kompleksem; dla jego komórki $\sigma$~przez $\phi_\sigma\colon \D^{\dim(\sigma)}\to X$ oznaczmy jej odwzorowanie charakterystyczne. CW kompleks $X$~nazywamy \textit{h-regularnym}\index{CW kompleks!h-regularny}, o~ile dla każdej komórki $\sigma$~tego kompleksu funkcja $\phi_\sigma\big|_{\S^{\dim(\sigma)-1}}\colon \S^{\dim(\sigma)-1}\to X$ jest homotopijną równoważnością na swój obraz oraz $\phi_\sigma(\sigma)$ jest podkompleksem kompleksu $X$. Jeśli CW kompleks $X$~jest h-regularny, to częściowy porządek $\mP(X)$ jest h-regularny.
\end{ex}

Ważne okażą się następujące własności omawianych klas częściowych porządków (pominięte dowody przenoszą się bez zmian z~pracy Miniana \cite{Minian12}).

\begin{stw}[{\cite[Remark 2.6]{Minian12}}]\label{stw-minianremark}
Każdy dopuszczalny częściowy porządek $P$~jest \mbox{h-regularny}.
\end{stw}
\begin{proof}
Ustalmy dopuszczalny częściowy porządek $P$~oraz $x\in P$. Wykażemy, że kompleks symplicjalny $\mK(\hat{x}\mathord{\downarrow})$ jest homotopijnie równoważny sferze $\S^{\Ht(x)-1}$.

Jeśli $\Ht(x)=0$, jest to oczywiste. Jeżeli $\Ht(x)=1$, to ponieważ $\hat{x}\mathord{\downarrow}$ jest antyłańcuchem, a~z~założenia kompleks \mbox{$\mK(\hat{x}\mathord{\downarrow}\smallsetminus \{y\})$} jest ściągalny dla każdego $y\prec x$, zbiór $\hat{x}\mathord{\downarrow}$ jest dwuelementowym antyłańcuchem, czyli $\mK(\hat{x}\mathord{\downarrow})\approx \S^0$.

Niech $n=\rk(x)$. Załóżmy, że $\mK(\hat{z}\mathord{\downarrow})\simeq \S^{\Ht(z)-1}$ dla wszystkich $z\in P$ takich, że $\Ht(z)< n$. Istnieje $y\prec x$ takie, że $\Ht(y)=n-1$. Z~założenia indukcyjnego $\mK(\hat{y}\mathord{\downarrow})\simeq \S^{n-2}$. Ponieważ \[\mK(\hat{x}\mathord{\downarrow})=\mK(y\mathord{\downarrow})\cup \mK(\hat{x}\mathord{\downarrow}\smallsetminus \{y\}),\] przy czym \[\mK(y\mathord{\downarrow})\cap \mK(\hat{x}\mathord{\downarrow}\smallsetminus \{y\})=\mK(\hat{y}\mathord{\downarrow}),\] zaś kompleksy $\mK(\hat{x}\mathord{\downarrow}\smallsetminus \{y\})$ i~$\mK(y\mathord{\downarrow})$ są ściągalne (pierwszy z~założenia o~dopuszczalności $P$, zaś drugi jako stożek), na podstawie~lematu \ref{lem-barmak_two_subcomplexes_lemma} zastosowanego do kompleksów $\mK(\hat{x}\mathord{\downarrow}\smallsetminus\{y\})$, $\mK(y\mathord{\downarrow})$ otrzymujemy, że kompleks $\mK(\hat{x}\mathord{\downarrow})$ jest homotopijnie równoważny zawieszeniu $\Sigma \mK(\hat{y}\mathord{\downarrow})$, które z~kolei jest homotopijnie równoważne sferze $\S^{n-1}$.
\end{proof}

\begin{lem}[{\cite[Lemma 2.8]{Minian12}}]\label{minianlemma}
Niech $P$ będzie h-regularnym częściowym porządkiem. Jeśli krawędź $(x,y)\in\mH(P)$ jest dopuszczalna, to $\Ht(x)=\Ht(y)+1$.
\end{lem}

\begin{wn}[{\cite[Lemma 2.8]{Minian12}}]\label{wn-minianlemma}
Dopuszczalny częściowy porządek jest porządkiem z~gradacją.
\end{wn}

\begin{lem}[{por. \cite[Theorem 2.12 (proof)]{Minian12}}]\label{lem-h-regular_cell_gluing_lemma}
Niech $P$ będzie h-regularnym częściowym porządkiem oraz niech $x\in \max(P)$. Wówczas istnieją ciągłe odwzorowanie $h\colon \S^{\Ht(x)-1}\to \mK(P\smallsetminus\{x\})$ oraz homotopijna równoważność \[g\colon \mK(P\smallsetminus \{x\})\cup_{h} \D^{\Ht(x)}\to \mK(P)\] o~tej własności, że $g\big |_{\mK(P\smallsetminus\{x\})}=\id_{\mK(P\smallsetminus\{x\})}$.
\end{lem}
\begin{proof}
Ponieważ częściowy porządek $P$ jest h-regularny, istnieje homotopijna równoważność $f\colon \S^{\Ht(x)-1}\to \mcK(\hat{x}\mathord{\downarrow})$. Niech $j\colon \mcK(\hat{x}\mathord{\downarrow})\hookrightarrow \mcK(P\smallsetminus \{x\})$ oznacza włożenie oraz niech $h=j\circ f$. Przez $\phi\colon \D^{\Ht(x)}\to \mK(P\smallsetminus\{x\})\cup_{h} \D^{\Ht(x)}$ oznaczmy odwzorowanie charakterystyczne doklejonej wzdłuż odwzorowania $h$~komórki.

Istnieje przemienny diagram
\[
\xymatrix{
\S^{\Ht(x)-1} \ar[rr]^{h=j\circ f}\ar@{_{(}->}[dd] & & \mK(P\smallsetminus\{x\})\ar@{_{(}->}'[d][dd]
\\
& \mK(\hat{x}\mathord{\downarrow}) \ar@{<-}[ul]_{f}\ar@{^{(}->}[rr]^(0.35)j\ar@{_{(}->}[dd] & & \mK(P\smallsetminus\{x\}) \ar@{=}[ul]\ar@{_{(}->}[dd]
\\
\D^{\Ht(x)} \ar'[r][rr]^(0.2){\phi} & & \mK(P\smallsetminus\{x\})\cup_{h}\D^{\Ht(x)}
\\
& \mK(x\mathord{\downarrow}) \ar@{^(->}[rr]\ar@{<-}[ul]^{F} & & \mK(P) \ar@{<-}[ul]_{g}
}
\]
taki, że przedni i~tylny kwadrat są w~nim kokartezjańskie, $F([x,t])=[f(x),t]$ dla $x\in \S^{\Ht(x)-1}$ oraz $t\in\I$ (utożsamiamy tutaj $\D^{\Ht(x)}$ oraz $\mK(x\mathord{\downarrow})$ ze stożkami odpowiednio nad $\S^{\Ht(x)-1}$ i~nad $\mK(\hat{x}\mathord{\downarrow})$), zaś odwzorowanie $g$ jest wyznaczone przez przekształcenia $f$,~$F$~oraz $\id_{\mK(P\smallsetminus \{x\})}$. Na podstawie lematu \ref{lem-gluing_lemma_for_adjunction_spaces} funkcja $g$~jest homotopijną równoważnością.
\end{proof}

\subsection{Główne twierdzenie dyskretnej teorii Morse'a dla nieskończonych skojarzeń}\label{subsec-glowne_tw_dyskr_teorii_morsea_dla_nsk_skoj}
Sformułowane niżej twierdzenie, stanowiące jeden z~najważniejszych wyników bieżącego rozdziału, jest uogólnieniem głównego twierdzenia dyskretnej teorii Morse'a \ref{tw-glowne_tw_klasycznej_dyskretnej_teorii_morsea}. Stanowi ono temat publikacji autora \cite{Kukiela13}. Podobne wyniki, które omawiamy w~końcowej części sekcji, uzyskali wcześniej Brown \cite[Proposition 1]{Brown92} oraz Orlik i~Welker \cite[Theorem 4.2.14]{Orlik07}.

\begin{tw}\label{maintw}
Niech $P$~będzie h-regularnym częściowym porządkiem z~zadanym dopuszczalnym skojarzeniem Morse'a $M$~bez promieni malejących. Wówczas kompleks symplicjalny $\mK(P)$ jest homotopijnie równoważny CW kompleksowi, którego zbiór komórek wymiaru $n$~jest równoliczny ze zbiorem $\mcC_n^M(P)$ dla każdego $n\in\mN$.
\end{tw}
\begin{proof}
Niech $P^*$~będzie liniowym rozszerzeniem porządku $P$~o własnościach jak w~lemacie \ref{lem-kozlov_lemma}.

Skonstruujemy liczbę porządkową $\alpha$~oraz pozaskończone ciągi: częściowych porządków $\bigl(P^*_\phi\bigr)_{\phi<\alpha}$, $\left(P_\phi\right)_{\phi<\alpha}$, będących podzbiorami częściowo uporządkowanymi odpowiednio $P^*$ i~$P$, CW kompleksów $\left(X_\phi\right)_{\phi<\alpha}$ oraz homotopijnych równoważności $\left(f_\phi\colon X_\phi\to \mK\left(P_\phi\right)\right)_{\phi<\alpha}$ takie, że dla każdej liczby porządkowej $\phi<\alpha$ spełnione są warunki:
\begin{compactitem}
\item[$(a_\phi)$] $X_\psi$~jest podkompleksem $X_\phi$ oraz $f_\psi\subseteq f_\phi$ dla wszystkich $\psi\leq\phi$;
\item[$(b_\phi)$] zbiór $n$-wymiarowych komórek~kompleksu $X_\phi$~jest równoliczny ze zbiorem $\mcC^M_n(P)\cap P_\phi$;
\item[$(c_\phi)$] zbiór elementów częściowego porządku $P_\phi$~jest równy zbiorowi $P^*\smallsetminus P^*_\phi$ i~stanowi odcinek początkowy w~$P^*$;
\item[$(d_\phi)$] najmniejszy element częściowego porządku $P^*_\phi$ (o~ile $P^*_\phi\not=\emptyset$) albo jest elementem $P$~krytycznym względem skojarzenia $M$, albo mniejszym spośród dwóch elementów $P$~tworzących krawędź należącą do skojarzenia $M$.
\end{compactitem}

Niech $P^*_0=P^*$, $P_0=\emptyset$, $X_0=\emptyset$, $f_0=\emptyset$.

Ustalmy liczbę porządkową $\phi$~i~załóżmy, że dla wszystkich $\psi<\phi$ obiekty $P^*_\psi,P_\psi,X_\psi,f_\psi$ są zdefiniowane i~spełniają warunki $(a_\psi)$, $(b_\psi)$, $(c_\psi)$, $(d_\psi)$.

Jeśli $\phi=\psi+1$ jest następnikiem, to rozważmy trzy przypadki.
\begin{itemize}
\item[---] Jeśli $P^*_\psi=\emptyset$, to przyjmujemy $\alpha=\phi$ i~kończymy konstrukcję. Z~warunków $(b_\psi), (c_\psi)$ wynika, że $X_\psi$~jest szukanym CW kompleksem.
\item[---] Jeżeli $p=\min\bigl(P^*_{\psi}\bigr)$ jest elementem krytycznym względem $M$, to przyjmujemy $P^*_\phi=P^*_\psi\smallsetminus\{p\}$, $P_\phi=P_\psi\cup\{p\}$. Na podstawie lematu \ref{lem-h-regular_cell_gluing_lemma} istnieją odwzorowanie  $\tilde{h}_p\colon \S^{\Ht(p)-1}\to \mK\left(P_\psi\right)$ oraz homotopijna równoważność $g\colon \mK\left(P_\psi\right)\cup_{\tilde{h}_p} \D^{\Ht(p)}\to \mK\left(P_\phi\right)$ taka, że  $g\big |_{\mK\left(P_\psi\right)}=\id_{\mK\left(P_\psi\right)}$. Niech $X_\phi'=X_\psi\cup_{h_p'} \D^{\Ht(p)}$, gdzie $h_p'=f_\psi\circ \tilde{h}_p\colon \S^{\Ht(p)-1}\to X_\psi$. Ponieważ odwzorowanie $f_\psi\colon X_\psi\to \mK\left(P_\psi\right)$ jest, z~założenia indukcyjnego, homotopijną równoważnością, zgodnie z~lematem \ref{lem-doklejanie_komorek_po_homotopijnych_odwzorowaniach} istnieje jego rozszerzenie $k\colon X_\phi'\to \mK\left(P_\psi\right)\cup_{\tilde{h}_p} D^{\Ht(p)}$ będące homotopijną równoważnością. Wobec lematu \ref{lem-cw-kompleks-po-doklejeniu} istnieją CW kompleks $X_\phi=X_\psi\cup_{h_p} \D^{\Ht(p)}$ oraz homotopijna równoważność $l\colon X_\phi\to X_\phi'$ rozszerzająca odwzorowanie identycznościowe na $X_\psi$. Przyjmijmy $f_\phi=(g\circ k\circ l)\colon X_\phi\to \mK\left(P_\phi\right)$.
\item[---] Jeśli $p=\min\bigl(P^*_{\psi}\bigr)$ jest elementem krawędzi należącej do~$M$, to z~własności liniowego rozszerzenia $P^*$ otrzymujemy, że $(q,p)\in M$ dla $q=\min\bigl(P^*_{\psi}\smallsetminus\{p\}\bigr)$. Przyjmijmy $P^*_{\phi}=P^*_\psi\smallsetminus\{p,q\}$, $P_{\phi}=P_\psi\cup\{p,q\}$. Zauważmy, że $\hat{p}\mathord{\uparrow}_{P_\phi}=\{q\}$, zatem $p\in P_\phi$ jest elementem nieredukowalnym pod $q$, wobec czego zgodnie z~lematem \ref{lem-punkt_posetu_ma_sciagalny_link_to_sdr_do_kompleksu_bez_tego_punktu} włożenie $i\colon \mK\left(P_\phi\smallsetminus\{p\}\right)\hookrightarrow \mK\left(P_\phi\right)$ jest homotopijną równoważnością. Ponieważ krawędź $(q,p)\in \mH(P)$ jest dopuszczalna, kompleks $\mK\left(\hat{q}\mathord{\downarrow}_P\smallsetminus\{p\}\right)=\mK\bigl(\hat{q}\mathord{\downarrow}_{P_\phi\smallsetminus\{p\}}\bigr)$ jest ściągalny, więc na podstawie~lematu \ref{lem-punkt_posetu_ma_sciagalny_link_to_sdr_do_kompleksu_bez_tego_punktu} włożenie $j\colon \mK\left(P_\psi\right)\hookrightarrow \mK\left(P_\phi\smallsetminus\{p\}\right)$ jest homotopijną równoważnością. Przyjmijmy $X_\phi=X_\psi$ oraz \mbox{$f_\phi=\left(i \circ j \circ f_\psi\right)\colon X_\phi\to \mK\left(P_\phi\right)$}.
\end{itemize}
Łatwo sprawdza się, że w~drugim i~trzecim spośród powyższych przypadków skonstruowane obiekty spełniają warunki $(a_\phi), (b_\phi), (c_\phi), (d_\phi)$.

Dla $\phi$ będącego graniczną liczbą porządkową definiujemy $P^*_\phi=\bigcap_{\psi<\phi}P^*_\psi$, $P_\phi=\bigcup_{\psi<\phi}P_\psi$, $X_\phi=\bigcup_{\psi<\phi}X_\psi$ (ze słabą topologią) oraz $f_\phi=\bigcup_{\psi<\phi} f_\psi$. Funkcja $f_\phi$ jest na podstawie lematu \ref{lem-ciag_wstepujacy_homotopijnych_rownowaznosci_cw_kompleksow} homotopijną równoważnością. Ponownie nietrudno spostrzec, że spełnione są warunki $(a_\phi), (b_\phi), (c_\phi), (d_\phi)$.
\end{proof}

Twierdzenie \ref{maintw} wraz z~wynikami podrozdziału \ref{reversing} pozwala udowodnić główne twierdzenie dyskretnej teorii Morse'a dla skojarzeń Morse'a o~skończonym zbiorze klas równoważności promieni malejących.

\begin{wn}\label{maincor}
Niech $P$~będzie dopuszczalnym częściowym porządkiem z~zadanym skojarzeniem Morse'a $M$~takim, że zbiór $\mcR^M(P)$ jest skończony. Wówczas kompleks symplicjalny $\mK(P)$ jest homotopijnie równoważny CW kompleksowi, którego zbiór komórek wymiaru $n$~jest równoliczny ze zbiorem $\mcC_n^M(P)\cup \mcR_n^M(P)$ dla każdego $n\in\mN$.
\end{wn}
\begin{proof}
Ponieważ porządek $P$~jest dopuszczalny, jest h-regularnym porządkiem z~gradacją (stwierdzenie~\ref{stw-minianremark} i~wniosek \ref{wn-minianlemma}). W~tej sytuacji stwierdzenie \ref{reversing_stw} pozwala na pozbycie się promieni malejących kosztem utworzenia nowych komórek krytycznych, przy czym otrzymane skojarzenie Morse'a jest dopuszczalne, gdyż porządek $P$~jest dopuszczalny. Teza wniosku wynika z~twierdzenia \ref{maintw}.
\end{proof}

Dzięki obserwacji z~przykładu \ref{ex-regularny_kompleks_daje_dopuszczalny_porzadek} otrzymujemy następujący wynik.

\begin{wn}\label{maincor2}
Niech $X$~będzie regularnym CW kompleksem z~zadanym skojarzeniem Morse'a $M$~takim, że zbiór $\mcR^M(\mP(X))$ jest skończony. Wówczas CW kompleks $X$ jest homotopijnie równoważny CW kompleksowi, którego zbiór komórek wymiaru $n$~jest równoliczny ze zbiorem $\mcC_n^M(\mP(X))\cup \mcR_n^M(\mP(X))$ dla każdego $n\in\mN$.
\end{wn}

Zauważmy, że prawdziwy jest odpowiednik wniosku \ref{maincor} dla dowolnych skojarzeń Morse'a na $1$-wymiarowych, regularnych CW kompleksach. Jego dowód przebiega analogicznie do dowodu wniosku \ref{maincor}, z~tym że w~miejsce stwierdzenia \ref{reversing_stw} korzysta się ze~stwierdzenia \ref{reversing_stw2}.

Dla $i\in\mN$ oraz częściowego porządku $P$~z~rangą i~zadanym skojarzeniem Morse'a $M$~symbol $c^M_i(P)$~niech oznacza moc zbioru $\mcC_i^M(P)$. Jeśli $P$~jest porządkiem z~gradacją, niech $r^M_i(P)$ oznacza moc zbioru $\mcR_i^M(P)$; w~przeciwnym wypadku przyjmujemy $r^M_i(P)=0$.

Uzyskane wyniki umożliwiają podanie nierówności analogicznych do dyskretnych nierówności Morse'a (wniosek \ref{wn-klasyczne_dyskretne_nierownosci_morsea}) przy~przyjętych w~twierdzeniu \ref{maintw} oraz wnioskach \ref{maincor}, \ref{maincor2} założeniach. (Dowód poniższego wniosku jest standardowy, patrz np.~\cite[str. 28-31]{Milnor63}.)

\begin{wn}\label{morse-ineq}
Niech $P$~będzie częściowym porządkiem z~zadanym skojarzeniem Morse'a $M$. Jeśli spełniony jest jeden z~poniższych warunków:
\begin{compactitem}
\item[---] porządek $P$~jest dopuszczalny i~zbiór $\mcR^M(P)$ jest skończony;
\item[---] porządek $P$~jest h-regularny i~$M$~jest dopuszczalnym skojarzeniem Morse'a bez promieni malejących;
\item[---] $P=\mP(X)$ dla pewnego $1$-wymiarowego, regularnego CW kompleksu $X$,
\end{compactitem}
to dla każdej liczby $n\in\mN$ mają miejsce nierówności:
\[c^M_n(P)+r^M_n(P)\geq \beta_n(\mK(P))\]
oraz
\[\sum_{i=0}^{n}(-1)^{n-i}\left(c^M_i(P)+r^M_i(P)\right)\geq \sum_{i=0}^{n}(-1)^{n-i}\beta_i(\mK(P)),\] o~ile $c^M_i(P)+r^M_i(P)<\infty$ dla wszystkich $i\leq n$. 
Ponadto, jeżeli $c^M_i(P)+r^M_i(P)<\infty$ dla wszystkich $i\in\mN$ oraz liczby te są niezerowe jedynie dla skończonej liczby indeksów $i$, to charakterystyka Eulera kompleksu $\mK(P)$~wyraża się wzorem \[\chi(P)=\sum_{i=0}^{\infty}(-1)^i \left(c^M_i(P)+r^M_i(P)\right).\]
\end{wn}
Wniosek \ref{morse-ineq} jest znacząco silniejszy od podobnych wyników, które podali Ayala, Fern{\'a}ndez i~Vilches \cite[Theorem 3.8]{Ayala07}, \cite[Theorem 3.1]{Ayala09}.

Odnotujmy, że jeśli $M'$~jest dopuszczalnym skojarzeniem Morse'a na \mbox{h-regularnym} częściowym porządku $P$~i~graf $\mH_{M'}(P)$ zawiera promień malejący $r$~taki, że $\{r\}$ jest wspaniałą rodziną reprezentującą $\{[r]\}$, to skojarzenie $M$~określone jak w~lemacie \ref{lem-wspaniali_reprezentanci_sa_odwracalni} nie musi być dopuszczalne. Przykład takiej sytuacji przedstawiony jest na rysunku \ref{fig-bad_matching}. W~związku z~tym nie jest znany odpowiednik wniosku \ref{maincor} dla dopuszczalnych skojarzeń Morse'a na dowolnych \mbox{h-regularnych} częściowych porządkach.

\begin{figure}[h]
\[
\xymatrix{
& & \bullet\ar@{<--}[dll]\ar[dl]\ar[d]\ar[dr] & & & \bullet\ar@{<--}[dll]\ar[dl]\ar[d]\ar[dr] & & & \bullet\ar@{<--}[dll]\ar[dl]\ar[d]\ar[dr] & & & \cdots\ar@{<--}[dll]\ar[dl]\\
\bullet\ar[d]\ar[dr] & \bullet\ar[d]\ar[dl] & \bullet\ar[dr]\ar[dl] & \bullet\ar[d]\ar[dr] & \bullet\ar[d]\ar[dl] & \bullet\ar[dr]\ar[dl] & \bullet\ar[d]\ar[dr] & \bullet\ar[d]\ar[dl] & \bullet\ar[dr]\ar[dl] & \bullet\ar[d]\ar[dr] & \bullet\ar[d]\ar[dl] & \cdots\ar[dl]\\
\bullet & \bullet & & \bullet & \bullet & & \bullet & \bullet & & \bullet & \bullet
}
\]
\caption{Diagram Hassego pewnego h-regularnego częściowego porządku z~gradacją po zmianie orientacji krawędzi (zaznaczonych linią przerywaną) należących do dopuszczalnego skojarzenia Morse'a na tym porządku. W~wyniku ,,odwrócenia'' widocznego w~górnej części rysunku promienia (przy użyciu techniki z~lematu \ref{lem-wspaniali_reprezentanci_sa_odwracalni}) otrzymuje się skojarzenie Morse'a, które nie jest dopuszczalne.}\label{fig-bad_matching}
\end{figure}

\begin{uw}
W~pracach Browna \cite[Proposition 1]{Brown92} oraz~Orlika i~Welkera \cite[Theorem 4.2.14]{Orlik07} podano odpowiedniki twierdzenia \ref{maintw} odpowiednio dla nieskończonych zbiorów symplicjalnych oraz dla niezwartych CW kompleksów (niekoniecznie regularnych). Przyjęte przez tych autorów założenia o~skojarzeniu Morse'a są, o~ile rozważany obiekt jest regularnym CW kompleksem (lub w~przypadku zbiorów symplicjalnych może być z~takowym utożsamiany), równoważne temu, że nie indukuje ono promieni malejących. Aby uniknąć wprowadzania stosowanej przez cytowanych autorów terminologii pomijamy dowód tego faktu. Odnotujmy jedynie, że równoważność braku promieni malejących i~warunku wprowadzonego przez Browna \cite[Condition (C2)]{Brown92} wynika prawie natychmiast z~definicji, zaś w~przypadku założenia przyjętego przez Orlika i~Welkera \cite[Definition 4.2.10]{Orlik07} z~udowodnionego w~następnej sekcji lematu \ref{lem-351}. Wniosek \ref{maincor2} jest nieco ogólniejszy niż cytowane wyniki w~tym senise, że w~jego założeniach dopuszcza się istnienie promieni malejących.
\end{uw}

% podejście Bauesa z pseudo-regularnymi CW kompleksami (a może h-pseudo-regularne?)


% ============================================================================================================================
% ============================================================================================================================
% ============================================================================================================================

\section{Algebraiczna wersja dyskretnej teorii Morse'a}\label{homological}
Celem bieżącego podrozdziału jest dowód odpowiednika głównego twierdzenia algebraicznej wersji dyskretnej teorii Morse'a \ref{tw-glowne_tw_algebraicznej_dyskretnej_teorii_morsea} dla nieskończonych kompleksów łańcuchowych, ugólniającego (w~niewielkim stopniu) wyniki J{\"o}llenbecka \cite{Jollenbeck05}. \begin{comment}Twierdzenie to wykorzystujemy w~dowodzie homologicznej wersji wniosków \ref{maincor} oraz \ref{morse-ineq}.
\end{comment}

\begin{comment}
\subsection{Główne twierdzenie dyskretnej teorii Morse'a dla nieskończonych skojarzeń w~wersji algebraicznej}
\end{comment}
Niech $R$~będzie pierścieniem z~jedynką, zaś $C=(C_i,\partial_i,B_i)_{i\in\mN}$ wolnym kompleksem łańcuchowym nad $R$~z~bazą. Jego \textit{podkompleksem}\index{podkompleks!wolnego kompleksu lzzzanzzzcuchowego z bazazzz@wolnego kompleksu łańcuchowego z~bazą} nazywamy wolny kompleks łańcuchowy nad $R$~z~bazą \mbox{$D=\bigl(D_i,\partial_i\big|_{D_i}\colon D_i\to D_{i-1}, A_i\bigr)_{i\in\mN}$} taki, że $D_i$~jest wolnym $R$-podmodułem $C_i$~generowanym przez $A_i\subseteq B_i$~dla każdego $i\in\mN$. Oczywiście skojarzenie Morse'a $M$~na $C$~indukuje w~tej sytuacji skojarzenie Morse'a $M\big|_D$~na~$D$.

Wykażemy, że twierdzenie \ref{tw-glowne_tw_algebraicznej_dyskretnej_teorii_morsea} pozostaje prawdziwe dla nieskończonego skojarzenia Morse'a $M$~na wolnym kompleksie łańcuchowym nad $R$~z bazą $C=(C_i,\partial_i,B_i)_{i\in\mN}$, o~ile graf skierowany $\mcV_M(C)$ nie zawiera promieni malejących. Posłużymy się w~tym celu następującą obserwacją.

\begin{lem}\label{lem-351}
Niech $P$~będzie częściowym porządkiem o~skończonych ideałach głównych z~zadanym odwzorowaniem $\rho\colon P\to\mN$ takim, że $\rho(p)<\rho(q)$ dla wszystkich $p<q$, $p,q\in P$, oraz niech $p_0\in P$. Załóżmy, że $M$~jest skojarzeniem Morse'a bez promieni malejących na częściowym porządku $P$, przy czym $\rho(q)=\rho(p)+1$ dla wszystkich $(q,p)\in M$. Rodzina zbiorów $A\subseteq P$ spełniających warunki:
\begin{compactenum}[\quad\ 1)]
\item\label{351-1} $p_0\in A$;
\item\label{351-2} $q\mathord{\downarrow} \subseteq A$ dla każdego $q\in A$; 
\item\label{351-3} jeśli $q\in A$ oraz $(q,r)\in M$ lub $(r,q)\in M$ dla pewnego $r\in P$, to $r\in A$,
\end{compactenum}
zawiera element najmniejszy $O(p_0)$, który jest zbiorem skończonym.
\end{lem}
\begin{proof}
Zauważmy, że zbiór $P$~spełnia warunki \ref{351-1}), \ref{351-2}), \ref{351-3}), więc rozważana rodzina zbiorów jest niepusta. Ponadto część wspólna dowolnej rodziny zbiorów spełniających te warunki również je spełnia. Istnieje zatem najmniejszy zbiór $O(p_0)$~spełniający te warunki (będący częścią wspólną wszystkich zbiorów je spełniających).

Wykażemy, że zbiór $O(p_0)$ jest skończony. Niech $D$~będzie grafem skierowanym o~zbiorze wierzchołków $O(p_0)$ i~zbiorze krawędzi \[\{(q,p)\in O(p_0)\times O(p_0):q\succ p \text{ lub } (p,q)\in M\}.\] Ponieważ $M$~jest skojarzeniem, zaś $P$~jest porządkiem o~skończonych ideałach głównych, graf $D$~jest lokalnie skończony. Zauważmy, że zbiór \[\{q\in O(p_0):\text{istnieje ścieżka w } D \text{ prowadząca z } p_0 \text{ do } q\}\] spełnia warunki \ref{351-1}), \ref{351-2}), \ref{351-3}), więc jest równy $O(p_0)$. Zatem dla każdego wierzchołka $q\in O(p_0)$ istnieje ścieżka w~$D$~prowadząca z~$p_0$~do $q$.

Przypuśćmy, że zbiór $O(p_0)$ jest nieskończony. Na podstawie lematu K\"oniga \ref{konig} istnieje w~$D$~nieskończona ścieżka prosta $(q_i)_{i\in\mN}$. Dla każdego $i\in\mN$ jeśli $q_i\succ q_{i+1}$, to $\rho(q_{i+1})\leq \rho(q_i)-1$, zaś jeżeli $q_{i}\prec q_{i+1}$, to $(q_{i+1},q_i)\in M$, więc $\rho(q_{i+1})=\rho(q_i)+1$. Ponadto, ponieważ $M$~jest skojarzeniem, niemożliwym jest, aby $q_i\prec q_{i+1}\prec q_{i+2}$. Stąd dla każdego $i\in \mN$ zachodzi jeden z~przypadków:
\begin{compactitem}
\item[---] $q_i\succ q_{i+1}\succ q_{i+2}$, a~zatem $\rho(q_{i+2})\leq \rho(q_{i})-2$;
\item[---] $q_i\prec q_{i+1}\succ q_{i+2}$, a~zatem $\rho(q_{i+2})\leq \rho(q_i)$;
\item[---] $q_i\succ q_{i+1}\prec q_{i+2}$, a~zatem $\rho(q_{i+2})\leq \rho(q_i)$.
\end{compactitem}
Przeciwdziedziną $\rho$~jest dobrze uporządkowany zbiór $\mathbb{N}$. Istnieje wobec tego co najwyżej skończona liczba indeksów $i\in\mN$ takich, że $q_i\succ q_{i+1}\succ q_{i+2}$. Dla pewnego $i_0\in \mN$ mamy więc $q_{i_0}\prec q_{i_0+1}\succ q_{i_0+2}\prec q_{i_0+3} \dots$, czyli $(q_{i_0+k})_{k\in\mN}$ jest promieniem malejącym w~$\mH_M(P)$, co jest sprzeczne z~założeniem, że $M$~jest skojarzeniem Morse'a bez promieni malejących.
\end{proof}

Poniższy wynik jest nieco ogólniejszy niż podobne twierdzenie J{\"o}llenbecka \cite[Theorem 1.4]{Jollenbeck05}, choć opiera się na tym samym pomyśle.

\begin{tw}\label{tw-glowne_tw_nsk_algebraicznej_dyskretnej_teorii_morsea}
Jeżeli $R$~jest pierścieniem z~jedynką, zaś $M$~jest skojarzeniem Morse'a na wolnym kompleksie łańcuchowym nad $R$~z~bazą $C=(C_i,\partial_i,B_i)_{i\in\mN}$ oraz graf skierowany $\mcV_M(C)$ nie zawiera promieni malejących, to istnieje łańcuchowo homotopijnie równoważny kompleksowi \mbox{$C_*=(C_i,\partial_i)_{i\in\mN}$} wolny kompleks łańcuchowy \mbox{$C_*^M=\bigl(C_i^M,\partial_i^M\bigr)_{i\in\mN}$} nad $R$~taki, że dla każdego $i\in\mN$ bazą wolnego $R$-modułu $C_i^M$~jest zbiór $B_i^M\subseteq B_i$ elementów krytycznych względem skojarzenia $M$, zaś homomorfizm $\partial_i^M\colon C_i^M\to C_{i-1}^M$ jest dla $c\in B_i^M$, $c'\in B_{i-1}^M$ zadany wzorem \[\partial^M(c)(c')=\Gamma^M(c,c').\]
\end{tw}
\begin{proof}
Zauważmy, że istnieje częściowy porządek $P$~taki, że $\mcV(C)=\mH(P)$. Niech $\rho\colon P\to \mN$ będzie funkcją dla $n\in\mN$ oraz~$p\in B_n$ zadaną wzorem $\rho(p)=n$. 

Rozważmy częściowy porządek $\Phi=\left(\{A\subseteq P: \moc{A}<\aleph_0\},\subseteq\right)$. Dla $F\in \Phi$ niech $O(F)=\bigcup_{p\in F}O(p)$, gdzie $O(p)$ są skończonymi podzbiorami $P$~o~własnościach jak w~lemacie \ref{lem-351}. Rodziny zbiorów $\{O(F)\}_{F\in \Phi}$ oraz włożeń \mbox{$\{O(F)\hookrightarrow O(F')\}_{F,F'\in\Phi, F\subseteq F'}$} tworzą system skierowany zbiorów częściowo uporządkowanych.

Niech $B(F)_i=\{p\in O(F):\rho(p)=i\}$; oczywiście $B(F)_i\subseteq B_i$. Dzięki warunkowi \ref{351-2}) (z lematu \ref{lem-351}) dla każdego $F\in \Phi$ istnieje wolny kompleks łańcuchowy nad $R$~z~bazą \[C(F)=\left(C(F)_i,\partial_i\big|_{C(F)_i}, B(F)_i\right)_{i\in\mN}\] będący podkompleksem $C$. Ponadto $C(F)_*$ jest podkompleksem $C(F')_*$ dla $F\subseteq F'$. Kompleksy łańcuchowe $\left\{C(F)_*\right\}_{F\in\Phi}$ z~odpowiednimi włożeniami tworzą zatem system skierowany. Korzystając z~warunku \ref{351-1}) otrzymujemy $\colim\left\{C(F)_*\right\}_{F\in\Phi}\cong C$.

Dla $F\in\Phi$ oznaczmy przez $M(F)$ skojarzenie $M\big|_{C(F)}$ w~grafie skierowanym $\mcV(C(F))$. Dzięki warunkom \ref{351-2}), \ref{351-3}) każda ścieżka prosta w~$\mcV_M(C)$ rozpoczynająca się w~elemencie należącym do $C(F)$~jest również ścieżką prostą w~$\mcV_{M(F)}(C(F))$. Zatem dla $F,F'\in \Phi$, $F\subseteq F'$ łańcuchowo homotopijnie równoważny kompleksowi $C(F)$~kompleks łańcuchowy $C(F)^M_*$, określony jak w~twierdzeniu \ref{tw-glowne_tw_algebraicznej_dyskretnej_teorii_morsea}, jest podkompleksem kompleksu $C(F')_*^M$. System skierowany tworzą więc również kompleksy łańcuchowe $\left\{C(F)^M_*\right\}_{F\in\Phi}$, a~ponadto $\colim\left\{C(F)^M_*\right\}_{F\in\Phi}\cong C^M$.

Dla $F\in\Phi$ niech $f_F\colon C(F)_*\to C(F)_*^M$, $g_F\colon C(F)_*^M\to C(F)_*$ będą odwzorowaniami łańcuchowymi zadanymi jak w~twierdzeniu \ref{tw-glowne_tw_algebraicznej_dyskretnej_teorii_morsea}. Wyznaczają one odwzorowania systemów skierowanych $\{C(F)_*\}_{F\in\Phi}\to \{C(F)^M_*\}_{F\in\Phi}$ oraz $\{C(F)_*^M\}_{F\in\Phi}\to \{C(F)_*\}_{F\in\Phi}$, więc indukują również odwzorowania łańcuchowe $f\colon C_*\to C_*^M$, $g\colon C_*^M\to C_*$ granic prostych tych systemów, które opisać możemy wzorami:
\begin{align*}
f_i(c)(c')&=\Gamma^M(c,c')\quad \text{dla } c\in B_i,\ c'\in B_i^M,\\
g_i(c)(c')&=\Gamma^M(c,c')\quad \text{dla } c\in B^M_i,\ c'\in B_i.
\end{align*}
Dla każdej liczby $i\in \mN$ rozważmy homomorfizm $\varphi_{i,i+1}\colon C_i\to C_{i+1}$ zadany dla $c\in B_i$, $c'\in B_{i+1}$ wzorem \[\varphi_i(c)(c')=\Gamma^M(c,c').\] 
Korzystając z~lematu J{\"o}llenbecka \cite[Lemma 2.3]{Jollenbeck05} otrzymujemy równości
\begin{align*}(g_i\circ f_i) - \id_{C_i}&=(\partial_{i+1} \circ \varphi_{i,i+1})+(\varphi_{i-1,i}\circ \partial_i),\\(f_i\circ g_i) - \id_{C_i^M}&= 0,\end{align*} co kończy dowód twierdzenia.
\end{proof}

\begin{uw}Twierdzenie \ref{tw-glowne_tw_nsk_algebraicznej_dyskretnej_teorii_morsea} można wykorzystać do badania tzw.~\textit{homologicznie dopuszczalnych}\index{skojarzenie Morse'a!homologicznie dopuszczalne} skojarzeń Morse'a na \textit{komórkowych}\index{czezzzszzzciowy porzazzzdek@częściowy porządek!komozzzrkowy@komórkowy} częściowych porządkach;  pojęcia te zostały wprowadzone w~przypadku skończonym przez Miniana \cite{Minian12}. Przykładowo, jeśli $P$~jest lokalnie skończonym częściowym porządkiem i~przestrzeń $|\mK(P)|$~jest $n$-wymiarową \textit{rozmaitością homologiczną}\index{rozmaitoszzzczzz homologiczna@rozmaitość homologiczna} (tzn. dla każdego $x\in |\mK(P)|$ istnieje izomorfizm $H_n(|\mK(P)|,|\mK(P)|\smallsetminus\{x\})\cong \mathbb{Z}$ oraz $H_j(|\mK(P)|,|\mK(P)|\smallsetminus\{x\})=0$ dla wszystkich $j\not=n$), to częściowy porządek $P$~jest komórkowy i~każde skojarzenie Morse'a na $P$~jest homologicznie dopuszczalne (por.~\cite[Theorem 4.6]{Minian12}).

Minian \cite{Minian12} wykazał, że grupy homologii (o~współczynnikach całkowitoliczbowych) skończonego, komórkowego częściowego porządku $P$~są izomorficzne homologiom pewnego wolnego kompleksu łańcuchowego $\mbC_*(P)$ takiego, że dla każdego $n\in\mN$ bazą wolnej grupy abelowej $\mbC_n(P)$ jest zbiór $\{p\in P:\Ht(p)=n\}$; wynik ten przenosi się na nieskończone częściowe porządki. Homologicznie dopuszczalne skojarzenie Morse'a na komórkowym częściowym porządku $P$~indukuje skojarzenie Morse'a na wolnym kompleksie łańcuchowym $\mbC_*(P)$ (z~wyżej opisaną bazą). Stosując twierdzenie \ref{tw-glowne_tw_nsk_algebraicznej_dyskretnej_teorii_morsea} do tego skojarzenia otrzymujemy homologiczne odpowiedniki twierdzenia \ref{maintw} oraz, w~konsekwencji, wniosków \ref{maincor}, \ref{morse-ineq}, uogólniając tym samym wyniki pracy Miniana \cite{Minian12}. 

Dowody autorstwa Miniana przenoszą się na nieskończone częściowe porządki prawie bez zmian, jednak ich przedstawienie wymagałoby sporego nakładu pracy i~wprowadzania dodatkowej terminologii; ponieważ z~wyników tych nie korzystamy w~dalszej części rozprawy, niniejszą uwagę pozostawiamy bez dowodu, zainteresowanego Czytelnika odsyłając do pracy Miniana \cite{Minian12} (oraz jej wstępnej wersji \cite{Minian}, zawierającej bardziej szczegółowe dowody).
\end{uw}

%-------------------------------------------------------------------
%-------------------------------------------------------------------
%-------------------------------------------------------------------

\begin{comment}

\subsection{Porządki komórkowe i~homologicznie dopuszczalne}\label{sec-porz_kom_i_hom_dop}
W~niniejszej sekcji podajemy przykłady częściowych porządków, przy badaniu których wykorzystać można twierdzenie \ref{tw-glowne_tw_nsk_algebraicznej_dyskretnej_teorii_morsea}, ale do których nie stosują się twierdzenia sekcji \ref{subsec-glowne_tw_dyskr_teorii_morsea_dla_nsk_skoj}. Opieramy się na pracy Miniana \cite{Minian12} (oraz jej wstępnej, obszerniejszej wersji \cite{Minian-arXiv}). Przedstawione niżej definicje (sformułowane przez Miniana~\cite{Minian12} dla skończonych częściowych porządków) oraz tok rozumowania zostały jedynie dostosowane do porządków nieskończonych. Uzupełniono także niektóre szczegóły dowodów.

\textbf{Wszystkie kompleksy łańcuchowe oraz grupy homologii w~bieżącej sekcji \ref{sec-porz_kom_i_hom_dop} mają~współczynniki w~pierścieniu liczb całkowitych $\mathbb{Z}$.}

Częściowy porządek $P$~z~gradacją nazywamy \textit{komórkowym}\index{czezzzszzzciowy porzazzzdek@częściowy porządek!komozzzrkowy@komórkowy}, jeżeli dla każdego $x\in P$ kompleks $\mK(\hat{x}\mathord{\downarrow})$ ma grupy homologii \mbox{$(\Ht(x)-1)$-wymiarowej} sfery $\S^{\Ht(x)-1}$ (przez \mbox{$(-1)$-wymiarową} sferę rozumiemy zbiór pusty). Dla $d\in\mN$ i~komórkowego porządku $P$~wprowadźmy oznaczenie $P^{(d)}=\{x\in P:\Ht(x)\leq d\}$.\nomenclature[szkielet_cz_porzadku]{$P^{(d)}$}{zbiór elementów komórkowego częściowego porządku $P$, których ranga nie przekracza $d$}

\begin{ex}[{\cite[Remark 3.2]{Minian12}}]
Każdy h-regularny częściowy porządek z~gradacją jest komórkowy. W~szczególności, jeżeli $X$~jest regularnym CW kompleksem, to częściowy porządek $\mP(X)$ jest komórkowy oraz $\mP(X)^{(d)}=\mP\bigl(X^{(d)}\bigr)$.
\end{ex}

Wykażemy, że dla komórkowego częściowego porządku $P$~homologie $H_*(P)$ (z~definicji równe $H_*(\mK(P))$) są izomorficzne homologiom pewnego wolnego kompleksu łańcuchowego $\mbC_*(P)=\left(\mbC_n,\partial_n^\mbC\right)_{n\in\mN}$, zwanego kompleksem łańcuchów komórkowych, który dla każdego $n\in\mN$ ma tę własność, że $\mbC_n$~jest grupą abelową wolną rozpiętą na zbiorze $\{p\in P:\Ht(p)=n\}$.

W~poniższych rozważaniach korzystamy z~faktu, że $\tilde{H}_n(\emptyset)=0$ dla wszystkich $n\geq 0$ oraz $\tilde{H}_{-1}(\emptyset)=\mathbb{Z}$.

\begin{stw}[{\cite[Proposition 3.3]{Minian12}}]\label{stw-minian_o_komorkowych_homologiach_1}
Niech $P$~będzie częściowym porządkiem z~gradacją oraz niech $d,n\in\mN$. Zachodzą izomorfizmy:
\[H_n\bigl(P^{(d)},P^{(d-1)}\bigr)\cong \bigoplus_{\substack{x\in P,\\\Ht(x)=d}}H_n(x\mathord{\downarrow},\hat{x}\mathord{\downarrow})\cong \bigoplus_{\substack{x\in P,\\\Ht(x)=d}}\tilde{H}_{n-1}(\hat{x}\mathord{\downarrow}).\]
\end{stw}
\begin{proof}
Rozważmy przemienny diagram

\[\xymatrix@C0pt{\coprod\limits_{\substack{x\in P,\\\Ht(x)=d}}\!\!\big(\mK(x\mathord{\downarrow}),\mK(\hat{x}\mathord{\downarrow})\big)\ar@{^(->}[r]^{\psi}\ar@{^(->}[d]^(0.6){\beta} & \left(\mK\bigl(P^{(d)}\bigr),\mK\bigl(P^{(d-1)}\bigr)\right)\ar@{^(->}[d]^(0.4){\beta'}\\
%
\coprod\limits_{\substack{x\in P,\\\Ht(x)=d}}\!\!\big(\mK(x\mathord{\downarrow}),\mK(x\mathord{\downarrow}) \smallsetminus\{x\}\big)\ar@{^(->}[r] & \left(\mK\bigl(P^{(d)}\bigr),\mK\bigl(P^{(d)}\bigr)\smallsetminus\!\!\!\!\!\coprod\limits_{\substack{x\in P,\\\Ht(x)=d}}\!\!\!\{x\}\right)\\
%
\bigcup\limits_{\substack{x\in P,\\\Ht(x)=d}}\!\!\!\!\big(\mK(x\mathord{\downarrow})\smallsetminus \mK(\hat{x}\mathord{\downarrow}),\big(\mK(x\mathord{\downarrow}) \smallsetminus\! \mK(\hat{x}\mathord{\downarrow})\big)\!\smallsetminus\!\{x\}\big)\ar@{^(->}[u]_(0.35){\gamma} &{}\save-<4cm,3.3cm>*{\left(\mK\bigl(P^{(d)}\bigr)\smallsetminus\mK\bigl(P^{(d-1)}\bigr),\left(\mK\bigl(P^{(d)}\bigr)\smallsetminus \mK\bigl(P^{(d-1)}\bigr)\right)\smallsetminus\!\!\!\!\!\bigcup\limits_{\substack{x\in P,\\\Ht(x)=d}}\!\!\!\!\{x\}\right)}\ar@{<-_)}[l]_(0.5){\phi}\ar@{^(->}[u]_{\gamma'}\restore
}
\]
par przestrzeni topologicznych, w~którym wszystkie odwzorowania są zanurzeniami. (Występujące w~diagramie kompleksy symplicjalne rozumiemy jako ich realizacje geometryczne, nie jako abstrakcyjne kompleksy.)

Odwzorowania $\beta$, $\beta'$, $\gamma$, $\gamma'$ oraz $\phi$ indukują izomorfizmy grup homologii singularnych ($\beta$, $\beta'$ jako homotopijne równoważności, $\phi$~jako homeomorfizm, zaś $\gamma$, $\gamma'$ na podstawie własności wycinania). Wobec przemienności powyższego diagramu, dla każdego $n\in\mN$ izomorfizmem jest \[H_n(\psi)\colon H_n\left(\coprod\limits_{\substack{x\in P,\\\Ht(x)=d}}\!\!\big(\mK(x\mathord{\downarrow}),\mK(\hat{x}\mathord{\downarrow})\big)\right)\longrightarrow H_n\left(\mK\bigl(P^{(d)}\bigr),\mK\bigl(P^{(d-1)}\bigr)\right).\]
Ponadto: \begin{align*}H_n\left(\coprod\limits_{\substack{x\in P,\\\Ht(x)=d}}\!\!\big(\mK(x\mathord{\downarrow}),\mK(\hat{x}\mathord{\downarrow})\big)\right)&\cong \bigoplus_{\substack{x\in P,\\\Ht(x)=d}}H_n(x\mathord{\downarrow},\hat{x}\mathord{\downarrow}),\\[0.6cm]
H_n\left(\mK\bigl(P^{(d)}\bigr),\mK\bigl(P^{(d-1)}\bigr)\right)&\cong H_n\left(P^{(d)},P^{(d-1)}\right).\end{align*}

Wykażemy, że dla każdej liczby $n\in\mN$ oraz każdego $x\in P$ istnieje izomorfizm $H_n\left(x\mathord{\downarrow},\hat{x}\mathord{\downarrow}\right)\cong \tilde{H}_{n-1}(\hat{x}\mathord{\downarrow})$. Jeżeli $\Ht(x)=0$, to oczywiście \[H_n(x\mathord{\downarrow},\hat{x}\mathord{\downarrow})=H_n(x\mathord{\downarrow},\emptyset)=H_n(x\mathord{\downarrow})=H_n(\{x\})\cong\tilde{H}_{n-1}(\emptyset)=\begin{cases}0 & \text{dla } n\not=0,\\ \mathbb{Z} & \text{dla } n=0.\end{cases}\] Natomiast jeśli $\Ht(x)\geq 1$, to ponieważ kompleks $\mK(x\mathord{\downarrow})$ jest acykliczny (jako stożek), powyższy izomorfizm otrzymujemy z~długiego ciągu dokładnego pary $(x\mathord{\downarrow}, \hat{x}\mathord{\downarrow})$.
\end{proof}

\begin{wn}\label{wniosek-izomorfizm-dla-nizszych-wymiarow}
Niech $P$~będzie komórkowym częściowym porządkiem oraz niech $d\in\mN$. Wówczas włożenie $P^{(d)}\hookrightarrow P$ indukuje dla każdego $n<d$ izomorfizm \[H_n\bigl(P^{(d)}\bigr)\cong H_n(P).\]
\end{wn}
\begin{proof}
Dla $p>0$ oraz $m<p$, mamy \[H_{m}\bigl(P^{(p)},P^{(p-1)}\bigr)\cong \bigoplus_{\Ht(x)=p} \tilde{H}_{m-1}(\hat{x}\mathord{\downarrow})\cong 0,\] gdzie pierwszy izomorfizm wynika ze stwierdzenia \ref{stw-minian_o_komorkowych_homologiach_1}, zaś drugi z~założenia, że dla każdego $x\in P$ kompleks symplicjalny $\mK(\hat{x}\mathord{\downarrow})$ ma homologie \mbox{$(\Ht(x)-1)$-wymiarowej} sfery.

Ustalmy $n<d$.  Z~długiego ciągu dokładnego pary $\bigl(P^{(d+1)},P^{(d)}\bigr)$:
\[\xymatrix@C=12pt{\cdots\ar[r] & H_{n+1}\big(P^{(d+1)},P^{(d)}\big)\ar[r] & H_{n}\big(P^{(d)}\big)\ar[r] & H_n\big(P^{(d+1)}\big)\ar[r] & H_{n}\big(P^{(d+1)},P^{(d)}\big)\ar[r] & \cdots}\] otrzymujemy izomorfizm $H_n\bigl(P^{(d)}\bigr)\cong H_{n}\bigl(P^{(d+1)}\bigr)$. Z~analogicznych powodów włożenia $P^{(d+k)}\hookrightarrow P^{(d+k+1)}$ indukują izomorfizmy $H_n\bigl(P^{(d+k)}\bigr)\cong H_{n}\bigl(P^{(d+k+1)}\bigr)$ dla wszystkich $k>0$. Ponieważ $P=\bigcup_{k\in\mN} P^{(d+k)}$, na podstawie~lematu \ref{lem-homologie_przemienne_z_kogranicami} włożenie $P^{(d)}\hookrightarrow P$ indukuje izomorfizm $H_n\bigl(P^{(d)}\bigr)\cong H_n(P)$.
\end{proof}

Fragmenty długich ciągów dokładnych par $\bigl(P^{(n+1)},P^{n}\bigr)$, $\bigl(P^{(n)},P^{(n-1)}\bigr)$, $\bigl(P^{(n-1)},P^{(n-2)}\bigr)$ można połączyć w~diagram
\begin{align}\label{diag_przemienny_33}
\xymatrix@C=-6pt{ & & & 0\\
 0\ar[dr] & & H_n(P^{(n+1)})\ar[ur]\\
 & H_n(P^{(n)})\ar[ur]\ar[dr]^{j_n}\\
 H_{n+1}(P^{(n+1)},P^{(n)})\ar[ur]^{\mathfrak{d}_{n+1}}\ar[rr]^{\partial^\mbC_{n+1}} & & H_n(P^{(n)},P^{(n-1})\ar[rr]^{\partial^\mbC_n}\ar[dr]_{\mathfrak{d}_n} & & H_{n-1}(P^{(n-1)}, P^{(n-2)})\\
 & & & H_{n-1}(P^{(n-1)})\ar[ur]_{j_{n-1}}\\
 & & 0\ar[ur]
}
\end{align}
przemienny, w~którym $\mathfrak{d}_{n+1},\mathfrak{d}_n$ są homomorfizmami łączącymi, $j_n,j_{n-1}$ są indukowane przez włożenia, zaś $\partial^\mbC_{n+1}=j_n\circ \mathfrak{d}_{n+1}$, $\partial^\mbC_n=j_{n-1}\circ \mathfrak{d}_{n}$. Zauważmy, że \[\partial^\mbC_n\circ \partial^\mbC_{n+1}=j_{n-1}\circ \mathfrak{d}_{n} \circ j_n\circ \mathfrak{d}_{n+1}=j_{n-1}\circ 0\circ \mathfrak{d}_{n+1}=0.\] \textit{Kompleksem łańcuchów komórkowych}\index{kompleks lzzzanzzzcuchowy@kompleks łańcuchowy!lzzzanzzzcuchozzzw komozzzrkowych czezzzszzzciowego porzazzzdku@łańcuchów komórkowych częściowego porządku} komórkowego częściowego porządku $P$~nazywamy kompleks łańcuchowy \nomenclature[kompl_lanc_komorkowych]{$\mbC_*(P)=\left(\mbC_n(P),\partial^\mbC\right)_{n\in\mN}$}{kompleks łańcuchów komórkowych komórkowego częściowego porządku $P$}$\mbC_*(P)=\left(\mbC_n(P),\partial^\mbC\right)_{n\in\mN}$ taki, że dla każdego $n\in\mN$: \[\mbC_n(P)=\bigoplus_{\substack{x\in P,\\\Ht(x)=n}}\tilde{H}_{n-1}(\hat{x}\mathord{\downarrow})\cong H_n\bigl(P^{(n)},P^{(n-1)}\bigr),\]
zaś $\partial^\mbC\colon \mbC_n(P)\to \mbC_{n-1}(P)$ jest złożeniem $\partial_n^\mbC=j_{n-1}\circ \mathfrak{d}_{n}$. 

\begin{tw}[por.~{\cite[Theorem 2.35]{Hatcher02}, \cite{Minian12}}]
Niech $P$~będzie komórkowym częściowym porządkiem. Wówczas $H_*(P)\cong H_*(\mbC(P))$.
\end{tw}
\begin{proof}
Ustalmy $n\in\mN$. Wobec wniosku \ref{wniosek-izomorfizm-dla-nizszych-wymiarow} istnieje izomorfizm $H_{n}\bigl(P^{(n+1)}\bigr)\cong H_n(P)$. Korzystając z~dokładności, z~diagramu (\ref{diag_przemienny_33}) otrzymujemy izomorfizm $H_{n}(P^{(n+1)})\cong H_{n}\bigl(P^{(n)}\bigr)\big/\operatorname{Im}\mathfrak{d}_{n+1}$. Ponieważ homomorfizm $j_n$~jest różnowartościowy, przeprowadza izomorficznie $\operatorname{Im}\mathfrak{d}_{n+1}$ na $\operatorname{Im}(j_n\circ \mathfrak{d}_{n+1})=\operatorname{Im}\partial^\mbC_{n+1}$ oraz $H_n\bigl(P^{(n)}\bigr)$ na $\operatorname{Im}j_n=\operatorname{Ker}\mathfrak{d}_n$. Z~różnowartościowości $j_{n-1}$ wynika, że $\operatorname{Ker}\mathfrak{d}_n = \operatorname{Ker}\partial^\mbC_n$. Zatem $j_n$~indukuje izomorfizm \[H_n\bigl(P^{(n)}\bigr)\big/\operatorname{Im}\mathfrak{d}_{n+1}\cong \operatorname{Ker}\partial^\mbC_n\big/\operatorname{Im} \partial^\mbC_{n+1}=H_n(\mbC_*(P)).\qedhere\]
\end{proof}

Niech $P$~będzie komórkowym częściowym porządkiem oraz niech $n\in\mN$. Jeśli dla każdego $x\in P$, $\Ht(x)=n$, wybierzemy generator grupy \[H_n(x\mathord{\downarrow},\hat{x}\mathord{\downarrow})\cong \tilde{H}_{n-1}(\hat{x}\mathord{\downarrow})\cong \mathbb{Z},\] to możemy grupę $\mbC_n(P)$ utożsamiać z~grupą abelową wolną o~zbiorze generatorów $\{x\in P:\Ht(x)=n\}$. Zastanówmy się, jak przy tym utożsamieniu można opisać odwzorowanie $\partial^\mbC_n\colon \mbC_n(P)\to \mbC_{n-1}(P)$. 

Ustalmy $n\geq 1$ oraz elementy $x,y\in P$ takie, że $\Ht(x)=n$, $\Ht(y)=n-1$. Rozważmy diagram
\begin{align}\label{diag_przem_34}
\xymatrix{
H_n(x\mathord{\downarrow},\hat{x}\mathord{\downarrow})\ar_{\partial_x}[drr]\ar^(0.45){i_n}[r] & H_n\big(P^{(n)},P^{(n-1)}\big)\ar^(0.53){\mathfrak{d}_n}[r] & H_{n-1}\big(P^{(n-1)}\big)\ar^(0.43){j_{n-1}}[r] & H_{n-1}\big(P^{(n-1)},P^{(n-2)}\big)\ar^{\varphi}[d] \\ 
& & H_{n-1}(y\mathord{\downarrow},\hat{y}\mathord{\downarrow}) & \!\!\!\bigoplus\limits_{\substack{z\in P\\\Ht(z)=n-1}}\!\!\!\!H_{n-1}(z\mathord{\downarrow},\hat{z}\mathord{\downarrow})\ar[l]_(0.57){\pi_y}
}
\end{align}
przemienny, w~którym $i_n,j_n$ są homomorfizmami indukowanymi przez włożenija, $\mathfrak{d}_n$ jest homomorfizmem łączącym z~długiego ciągu dokładnego pary $\big(P^{(n)},P^{(n-1)}\big)$, izomorfizm $\varphi$ pochodzi ze~stwierdzenia \ref{stw-minian_o_komorkowych_homologiach_1}, zaś $\pi_y$ jest rzutem na składnik prosty o~indeksie $y$.

Oznaczmy przez $C_*(x\mathord{\downarrow})=(C_n(x\mathord{\downarrow}),\partial_n)_{n\in\mN}$ kompleks łańcuchów singularnych (bądź symplicjalnych {\huge\bf O CO TU CHODZI???}) częściowego porządku $P$. Niech $z$~będzie cyklem wyznaczającym relatywną klasę homologii \mbox{$\alpha\in H_n(x\mathord{\downarrow},\hat{x}\mathord{\downarrow})$}. Zatem $\partial_n(z)\in C_{n-1}(\hat{x}\mathord{\downarrow})$. Ale $(j_{n-1}\circ \mathfrak{d}_n\circ i_n)(\alpha)=[\partial_n(z)]\in H_{n-1}(P^{(n-1)},P^{(n-2)})$. Homomorfizm $\partial_x$ może być zatem niezerowy jedynie wtedy, gdy $y\prec x$. Homomorfizm $\partial^\mbC\colon \mbC_n(P)\to \mbC_{n-1}(P)$ wyraża się zatem dla $x,y\in P$, $\Ht(x)=n$, $\Ht(y)=n-1$ wzorem $\partial^\mbC(x)(y)=\epsilon(x,y)$, gdzie $\epsilon(x,y)\in\mathbb{Z}$ jest stopniem odwzorowania \[\partial_x\colon \mathbb{Z}=H_{n}(x\mathord{\downarrow},\hat{x}\mathord{\downarrow})\longrightarrow H_{n-1}(y\mathord{\downarrow},\hat{y}\mathord{\downarrow})=\mathbb{Z}.\]

Niech $P$~będzie częściowym porządkiem. Mówimy, że krawędź $(x,y)\in\mH(P)$ jest \textit{homologicznie dopuszczalna}\index{krawezzzdzzzz@krawędź!homologicznie dopuszczalna}, jeśli kompleks $\mK(\hat{x}\mathord{\downarrow}\smallsetminus\{y\})$ jest acykliczny. Porządek $P$~nazywamy \textit{homologicznie dopuszczalnym}\index{czezzzszzzciowy porzazzzdek@częściowy porządek!homologicznie dopuszczalny}, jeżeli każda krawędź diagramu Hassego $\mH(P)$ jest homologicznie dopuszczalna.

Skojarzenie Morse'a $M$~na komórkowym częściowym porządku $P$~nazywamy \textit{homologicznie dopuszczalnym}\index{skojarzenie!Morse'a!homologicznie dopuszczalne}, jeśli każda krawędź $\mH(P)$ należąca do $M$~jest homologicznie dopuszczalna.

\begin{ex}[{\cite[Remark 3.10]{Minian12}}]
Jeśli $P$~jest dopuszczalnym częściowym porządkiem, to $P$~jest homologicznie dopuszczalny. W~szczególności, jeżeli $X$~jest regularnym CW kompleksem, to częściowy porządek $\mP(X)$~jest homologicznie dopuszczalny.
\end{ex}

\begin{ex}[{\cite[Theorem 4.6]{Minian12}}]
Przypomnijmy, że lokalnie zwartą przestrzeń Hausdorffa $X$~nazywa się $n$-wymiarową \textit{rozmaitością homologiczną}\index{rozmaitoszzzczzz homologiczna@rozmaitość homologiczna}, o~ile dla każdego $x\in X$ istnieje izomorfizm $H_n(X,X\smallsetminus\{x\})\cong \mathbb{Z}$ oraz $H_j(X,X\smallsetminus\{x\})=0$ dla wszystkich $j\not=n$. 

Lokalnie skończony częściowy porządek $P$~nazywamy $n$-wymiarową \textit{rozmaitością homologiczną}, jeżeli $|\mK(P)|$ jest $n$-wymiarową rozmaitością homologiczną. (Równoważnie moglibyśmy wymagać, aby kompleks symplicjalny $\mK(P)$ był \mbox{$n$-wymiarową} \textit{kombinatoryczną rozmaitością homologiczną}, tzn.~aby dla każdego sympleksu $\sigma\in \mK(P)$ podkompleks $\lk_{\mK(P)}(\sigma)$ miał homologie sfery $\S^{n-\dim(\sigma)-1}$.)

Każdy lokalnie skończony częściowy porządek $P$~będący rozmaitością homologiczną jest homologicznie dopuszczalny.
\end{ex}

\begin{ex}[{\cite[Example 4.7]{Minian12}}]
Niech $K$~będzie triangulacją sfery homologicznej Poincar{\`e} (zob.~np.~\cite{Bjorner03}), będącej przykładem $3$-wymiarowej rozmaitości o~homologiach sfery $\S^3$~i~nietrywialnej grupie podstawowej. Częściowy porządek $\mP(K)\oplus \mP(K)$ jest homologicznie dopuszczalny, ale nie jest dopuszczalny.
\end{ex}

Udowodnimy homologiczny odpowiednik stwierdzenia \ref{stw-minianremark}.

\begin{stw}[{\cite[p.~2865]{Minian12}}]
Homologicznie dopuszczalny częściowy porządek jest komórkowy.
\end{stw}
\begin{proof}
Niech $P$~będzie homologicznie dopuszczalnym częściowym porządkiem. Wykażemy metodą indukcji matematycznej, że dla każdego $x\in P$ kompleks $\mK(\hat{x}\mathord{\downarrow})$ ma grupy homologii \mbox{$(\Ht(x)-1)$-wymiarowej} sfery.

Gdy $\Ht(x)=0$, jest to oczywiste. Jeżeli $\Ht(x)=1$, to zbiór $\hat{x}\mathord{\downarrow}$ jest antyłańcuchem. Ponieważ porządek $\hat{x}\mathord{\downarrow}\smallsetminus\{y\}$ ma trywialne homologie dla każdego $y\prec x$, antyłańcuch ten jest dwuelementowy, zatem $\mK(\hat{x}\mathord{\downarrow})\approx \S^0$.

Rozważmy $x\in P$ rangi $n\geq 2$ i~załóżmy, że $\tilde{H}_*(\mK(\hat{y}\mathord{\downarrow}))\cong \tilde{H}_*(\S^{\Ht(y)-1})$ dla wszystkich $y\in P$~takich, że $\Ht(y)<n$; ustalmy $y\prec x$. Wobec homologicznej dopuszczalności porządku $P$~kompleks $\mK(\hat{x}\mathord{\downarrow})$ jest spójny, więc $\Ht(y)>0$. Rozważmy dwuelementowe pokrycie $\left\{\mK(y\mathord{\downarrow}), \mK(\hat{x}\mathord{\downarrow} \smallsetminus\{y\})\right\}$ kompleksu $\mK(\hat{x}\mathord{\downarrow})$. Oba podkompleksy tworzące to pokrycie są acykliczne (pierwszy jako stożek, drugi z~założenia indukcyjnego); zachodzi ponadto równość $\mK(\hat{x}\mathord{\downarrow}\smallsetminus \{y\}) \cap \mK(y\mathord{\downarrow})=\mK(\hat{y}\mathord{\downarrow})\not=\emptyset$. Rozważmy stowarzyszony z~tym pokryciem ciąg Mayera-Vietorisa (w~wersji zredukowanej):
%\begin{align*}\xymatrix{\cdots\ar[r] & \tilde{H}_{n+1}(\mK(\hat{x}\mathord{\downarrow}\smallsetminus\{y\}))\oplus \tilde{H}_{n+1}(\mK(y\mathord{\downarrow}))\ar[r] & \tilde{H}_{n+1}(\mK(\hat{x}\mathord{\downarrow}))\ar[r] & \tilde{H}_n(\mK(\hat{y}\mathord{\downarrow}))\ar[r]& \\ *+<23pt>{}\ar[r] & \tilde{H}_{n}(\mK(\hat{x}\mathord{\downarrow}\smallsetminus\{y\}))\oplus \tilde{H}_{n}(\mK(y\mathord{\downarrow})) &{}\save []-<1.22cm,0cm>*\txt{ $\cdots$}\ar@{<-}[l]\restore} \end{align*}
\begin{align*}\xymatrix{\cdots\ar[r] & \tilde{H}_{n+1}(\mK(\hat{x}\mathord{\downarrow}\smallsetminus\{y\}))\oplus \tilde{H}_{n+1}(\mK(y\mathord{\downarrow}))\ar[r] & \tilde{H}_{n+1}(\mK(\hat{x}\mathord{\downarrow}))\ar[r] &}\\
\xymatrix{\ar[r] & \tilde{H}_n(\mK(\hat{y}\mathord{\downarrow}))\ar[r] & \tilde{H}_{n}(\mK(\hat{x}\mathord{\downarrow}\smallsetminus\{y\}))\oplus \tilde{H}_{n}(\mK(y\mathord{\downarrow})) &\cdots\ar@{<-}[l]}
\end{align*}
stowarzyszony z~tym pokryciem. Otrzymujemy z~niego, że \[\tilde{H}_*(\mK(\hat{x}\mathord{\downarrow}))\cong \tilde{H}_{*-1}(\hat{y}\mathord{\downarrow})\cong \tilde{H}_{*-1}(S^{\Ht(y)-1}).\] Gdyby istniały $y_1,y_2\prec x$ takie, że $\Ht(y_1)\not=\Ht(y_2)$, mielibyśmy \[\tilde{H}_{*-1}(S^{\Ht(y_1)-1})\cong \tilde{H}_*(\mK(\hat{x}\mathord{\downarrow}))\cong \tilde{H}_{*-1}(S^{\Ht(y_2)-1}),\] co jest niemożliwe. Ponieważ istnieje $y_0\prec x$ rangi $\Ht(x)-1$, mamy $\Ht(y)=\Ht(x)-1$ dla wszystkich $y\prec x$. Wobec tego $P$~jest porządkiem z~gradacją oraz \[\tilde{H}_*(\mK(\hat{x}\mathord{\downarrow}))\cong \tilde{H}_{*-1}(S^{\Ht(x)-1})\cong \tilde{H}_{*}(S^{\Ht(x)}).\qedhere\]
\end{proof}

\begin{stw}[{\cite[Remark 3.9]{Minian12}}]
Niech $P$~będzie komórkowym częściowym porządkiem. Jeśli krawędź $(x,y)\in\mH(P)$ jest homologicznie dopuszczalna, to $\epsilon(x,y)\in\{1,-1\}$.
\end{stw}
\begin{proof}
Niech $n=\Ht(x)$. Nietrudno sprawdzić, że homomorfizm $d^C\colon H_{n}(x\mathord{\downarrow},\hat{x}\mathord{\downarrow})\to H_{n-1}(y\mathord{\downarrow},y\mathord{\downarrow})$ pokrywa się z~homomorfizmem łączącym $\tilde{\partial}$~w~relatywnym długim ciągu Mayera-Vietorisa par $(x\mathord{\downarrow}, \hat{x}\mathord{\downarrow}\smallsetminus\{y\})$, $(y\mathord{\downarrow}, y\mathord{\downarrow})$:
\begin{align*}\xymatrix@C=22pt{\cdots & {}\save []-<0.25cm,0cm>*\txt{\ ${H}_{n}(x\mathord{\downarrow},\hat{x}\mathord{\downarrow}\!\smallsetminus\{y\})\oplus {H}_{n}(y\mathord{\downarrow},y\mathord{\downarrow})$}\ar[r]\ar@{<-}[l]\restore & {H}_{n}(x\mathord{\downarrow},\hat{x}\mathord{\downarrow})\ar^(0.475){\tilde{\partial}}[r] & {H}_{n-1}(y\mathord{\downarrow},\hat{y}\mathord{\downarrow})\ar[r] & \\ \ar[r] & {H}_{n-1}(x\mathord{\downarrow},\hat{x}\mathord{\downarrow}\!\smallsetminus\{y\})\oplus {H}_{n-1}(y\mathord{\downarrow},y\mathord{\downarrow}) &{}\save []-<0.25cm,0cm>*\txt{ $\cdots$.}\ar@{<-}[l]\restore} \end{align*}
Oczywiście ${H}_{k}(y\mathord{\downarrow},y\mathord{\downarrow})=0$ oraz $H_{k}(x\!,\hat{x}\mathord{\downarrow}\!\smallsetminus\{y\})\cong \tilde{H}_{k-1}(\hat{x}\mathord{\downarrow}\!\smallsetminus\{y\})$ dla wszystkich $k\in\mN$. Dopuszczalność krawędzi $(x,y)\in\mH(P)$ oznacza, że $\tilde{H}_{k-1}(\hat{x}\mathord{\downarrow}\smallsetminus\{y\})=0$.

Wobec tego $\tilde{\partial}=d^C$ jest izomorfizmem.
\end{proof}

Zauważmy, że z~ostatniego stwierdzenia wynika w~szególności, że każdy porządek homologicznie dopuszczalny ma skończone ideały główne, co nie jest prawdą dla dowolnych porządków komórkowych.

Powyższe obserwacje wraz ze~stwierdzeniem \ref{reversing_stw} oraz twierdzeniem \ref{tw-glowne_tw_nsk_algebraicznej_dyskretnej_teorii_morsea}, składają się na dowód następującego wyniku.

\begin{tw}
Niech $P$~będzie komórkowym częściowym porządkiem z~zadanym homologicznie dopuszczalnym skojarzeniem Morse'a. Jeśli spełniony jest jeden z~poniższych warunków:
\begin{compactitem}
\item[---] porządek $P$~jest homologicznie dopuszczalny i~zbiór $\mcR_M(P)$ jest skończony;
\item[---] $M$~jest skojarzeniem Morse'a bez promieni malejących,
\end{compactitem}
to istnieje łańcuchowo homotopijnie równoważny kompleksowi $\mbC(P)$ wolny kompleks łańcuchowy $C^M=(C_i^M,\partial^M)_{i\in\mN}$ taki, że $C^M_i$~jest, dla każdego $i\in\mN$, wolną grupą abelową rozpiętą na zbiorze $\mcC^i_M(P)\cup \mcR^i_M(P)$. W~szczególności $H_*(P)\cong H_*(C^M)$.
\end{tw}
Postać odwzorowań $\partial^M\colon C_i^M\to C_{i-1}^M$ w~powyższym kompleksie łańcuchowym odczytać możemy z~twierdzenia \ref{tw-glowne_tw_algebraicznej_dyskretnej_teorii_morsea} oraz, o~ile $\mcR_M(P)\not=\emptyset$, z~dowodu stwierdzenia \ref{reversing_stw}.

Przypomnijmy, że dla $i\in\mN$ oraz częściowego porządku $P$~z~gradacją i~zadanym skojarzeniem Morse'a $M$ symbol $m_i(M)$ oznacza liczbę elementów zbioru $\mcC^i_M(P)$, zaś $r_i(M)$ oznacza liczbę elementów zbioru $\mcR^i_M(P)$. Symbolem $b_i$ oznaczamy $i$-tą liczbę Bettiego kompleksu $\mK(P)$.

\begin{wn}\label{homological-morse-ineq}
Niech $P$~będzie niepustym, komórkowym częściowym porządkiem z~zadanym homologicznie dopuszczalnym skojarzeniem Morse'a. Jeśli spełniony jest jeden z~poniższych warunków:
\begin{compactitem}
\item[---] porządek $P$~jest homologicznie dopuszczalny i~zbiór $\mcR_M(P)$ jest skończony;
\item[---] $M$~jest skojarzeniem Morse'a bez promieni malejących,
\end{compactitem}
to dla każdej liczby $n\in\mN$ mają miejsce nierówności:
\[m_n(M)+r_n(M)\geq b_n\]
oraz
\[\sum_{i=0}^{n}(-1)^{n-i}(m_i(M)+r_i(M))\geq \sum_{i=0}^{n}(-1)^{n-i}b_i,\] o~ile $m_i(M)+r_i(M)<\infty$ dla wszystkich $i\leq n$. 
Ponadto, jeżeli $m_i(M)+r_i(M)<\infty$ dla wszystkich $i\in\mN$ oraz liczby te są niezerowe jedynie dla skończonej liczby indeksów $i$, to dla charakterystyki Eulera zachodzi równość \[\chi(X)=\sum_{i=0}^{\infty}(-1)^i (m_i(M)+r_i(M)).\]
\end{wn}

Zakończmy podrozdział uwagą, że jeśli $M'$~jest homologicznie dopuszczalnym skojarzeniem Morse'a na komórkowym częściowym porządku $P$~i $\mH_{M'}(P)$ zawiera promień malejący $r$~taki, że $\{r\}$ jest wspaniałą rodziną reprezentującą $\{[r]\}$, to skojarzenie $M$~określone jak w~lemacie \ref{lem-wspaniali_reprezentanci_sa_odwracalni} może nie być homologicznie dopuszczalne. Przykładu dostarcza ponownie rysunek \ref{fig-bad_matching}.
\end{comment}
% ============================================================================================================================
% ============================================================================================================================
% ============================================================================================================================


\section{\texorpdfstring{Kompleksy $\infty$-zgniatalne}{Kompleksy ∞-zgniatalne}}\label{sec-infty-zgniatalnosc}

Regularny CW kompleks $X$~nazywamy \textit{$\infty$-zgniatalnym do podkompleksu $Y$}\index{0zgniatalnoszzzczzz@$\infty$-zgniatalność}\index{zgniecenie!0zgniecenie@$\infty$-zgniecenie} i~piszemy $X\infcoll Y$\nomenclature[7e]{$X\infcoll Y$}{regularny CW kompleks $X$~jest $\infty$-zgniatalny do podkompleksu $Y$}, o~ile istnieje skojarzenie Morse'a bez promieni malejących na $\mP(X)$, którego zbiorem komórek krytycznych jest $\mP(Y)$. Mówimy, że $X$~jest \mbox{\textit{$\infty$-zgniatalny}}\index{CW kompleks!regularny!NSK-zgniatalny@$\infty$-zgniatalny}, co oznaczamy przez $X\infcoll *$\nomenclature[7f]{$X\infcoll *$}{regularny CW kompleks $X$~jest $\infty$-zgniatalny (do punktu)}, jeżeli $X$~jest $\infty$-zgniatalny do punktu. (Podobne pojęcie rozważane było w~kontekście własności metryk na wielościanach w~pracy Jiehua i~Yun \cite{Jiehua91}; jest też znane pojęcie nieskończonego prostego typu homotopijnego, zob.~np.~\cite{Baues01,Hughes96}.)

Powyższa definicja uogólnia klasyczne pojęcie zgniatalności skończonych, regularnych CW kompleksów. Celem bieżącego podrozdziału jest dowód podstawowych obserwacji dotyczących tego uogólnienia, porównanie go z~pojęciem (ko)rozbieralności oraz podanie przykładów klas niezwartych, $\infty$-zgniatalnych kompleksów symplicjalnych.

\subsection{\texorpdfstring{Podstawowe obserwacje dotyczące $\infty$-zgniatalności}{Podstawowe obserwacje dotyczące ∞-zgniatalności}}
Przedstawione w~niniejszej sekcji obserwacje wskazują, że pojęcie \mbox{$\infty$-zgniatalności} jest faktycznie naturalnym uogólnieniem zgniatalności. Ponadto służą one jako użyteczne lematy w~dalszej części rozdziału.

\begin{lem}\label{lem-charakteryzacja_inf_zgniatalnosci}
Niech $X$~będzie regularnym CW kompleksem, zaś $Y$ jego podkompleksem. Wówczas $X\infcoll Y$ wtedy i~tylko wtedy, gdy istnieją liczba porządkowa $\alpha$~oraz pozaskończony ciąg $\left(X_\phi\right)_{\phi<\alpha}$ podkompleksów $X$~takie, że:
\begin{compactitem}
\item[---] $X_0=Y$;
\item[---] $X_\alpha=X$;
\item[---] $X_{\phi+1}\elcoll X_{\phi}$ dla każdej liczby porządkowej $\phi<\alpha$;
\item[---] $X_{\phi}=\bigcup_{\psi<\phi} X_{\psi}$ dla każdej granicznej liczby porządkowej $\phi\leq\alpha$.
\end{compactitem}
\end{lem}
\begin{proof}
Załóżmy, że istnieje na $\mP(X)$~skojarzenie Morse'a $M$~bez promieni malejących, którego zbiorem elementów krytycznych jest $\mP(Y)$. Jest ono oczywiście również skojarzeniem Morse'a bez promieni malejących na zbiorze \mbox{$\mP(X)\smallsetminus \mP(Y)$}. Niech $\leq^*$ będzie liniowym rozszerzeniem relacji porządkującej zbiór \mbox{$\mP(X)\smallsetminus \mP(Y)$} o~własnościach jak w~lemacie \ref{lem-kozlov_lemma}.

Skonstruujemy liczbę porządkową $\alpha$~oraz pewien wstępujący, pozaskończony ciąg $\left(X_\phi\right)_{\phi\leq\alpha}$ podkompleksów kompleksu $X$. 

Niech $X_0=Y$. Załóżmy, że $\phi>0$ jest liczbą porządkową oraz kompleksy $X_\psi\subseteq X$ zostały określone dla wszystkich liczb porządkowych $\psi<\phi$ w~ten sposób, że $\mP\left(X_\psi\right)\smallsetminus\mP(Y)$ jest odcinkiem początkowym $(\mP(X)\smallsetminus\mP(Y),\leq^*)$ oraz $(\tau,\sigma)\in M$, gdzie $\sigma$~oraz $\tau$~są odpowiednio najmniejszym elementem zbioru uporządkowanego $\bigl(\mP(X)\smallsetminus\mP\left(X_\psi\right),\leq^*\big|_{\mP(X)\smallsetminus\mP\left(X_\psi\right)}\bigr)$ i~jego pokryciem górnym w~tym porządku. Jeżeli $\phi$~jest następnikiem, $\phi=\psi+1$, to niech $X_\phi$~będzie podkompleksem $X$~powstałym przez doklejenie do kompleksu $X_\psi$ komórek $\sigma,\tau$~(tzn. $\mP(X_\phi)=\mP(X_\psi)\cup\{\sigma,\tau\}$). Zauważmy, że $\sigma$~jest w~$X_\phi$ ścianą wolną (zawartą w~$\tau$); wobec tego $X_\phi\elcoll X_\psi$. Jeśli natomiast $\phi$~jest graniczną liczbą porządkową, niech $X_\phi=\bigcup_{\psi<\phi} X_\psi$. W~każdym wypadku, jeżeli $X_\phi=X$, to przyjmujemy $\alpha=\phi$ i~kończymy konstrukcję. Jeżeli $X_\phi\not= X$, zauważmy, że ponieważ skojarzenie $M$~nie ma elementów krytycznych w~zbiorze $\mP(X)\smallsetminus \mP(Y)$, to do pewnej krawędzi tego skojarzenia należy element $\sigma'$~najmniejszy w~porządku $\bigl(\mP(X)\smallsetminus\mP\left(X_\phi\right),\leq^*|_{\mP(X)\smallsetminus\mP\left(X_\phi\right)}\bigr)$. Wobec własności relacji porządkującej $\leq^*$, dla $\tau'\in \mP(X)\smallsetminus\mP\left(X_\phi\right)$ będącego pokryciem górnym elementu $\sigma'$~względem porządku $\leq^*$ mamy $(\tau',\sigma')\in M$. Skonstruowany ciąg $\left(X_\phi\right)_{\phi<\alpha}$ ma wymienione w~treści lematu własności.

Udowodnimy teraz przeciwną implikację. Niech $\alpha$~będzie liczbą porządkową, zaś $(X_\phi)_{\phi<\alpha}$ pozaskończonym ciągiem podkompleksów CW kompleksu $X$~o~własnościach wymienionych w~sformułowaniu lematu.

Dla każdej liczby porządkowej $\phi<\alpha$ niech $\tau_\phi,\sigma_\phi\in\mP(X)$ będą komórkami takimi, że $\sigma_\phi$ jest ścianą wolną $\tau_\phi$ w~$X_{\phi+1}$ oraz $\mP\left(X_{\phi}\right)=\mP\left(X_{\phi+1}\right)\smallsetminus\left\{\tau_\phi,\sigma_\phi\right\}$. Przyjmijmy $M=\left\{\left(\tau_\phi,\sigma_\phi\right):\phi<\alpha\right\}$. Oczywiście $M$~jest skojarzeniem w~grafie skierowanym $\mH(\mP(X))$, zaś elementy zbioru $\mP(X)$, które nie należą do żadnej krawędzi skojarzenia $M$, to dokładnie komórki tworzące podkompleks~$Y$. Korzystając z~lematu \ref{lem-niesk_sciezka_a_skojarzenie_morse} wykażemy, że $M$~jest skojarzeniem Morse'a bez promieni malejących na $X$.

Zauważmy na początek, że jeżeli $(x,y)\in\mH_M(\mP(X))$, przy czym $x\in \mP\left(X_\phi\right)$ dla pewnej liczby porządkowej $\phi\leq \alpha$, to $y\in \mP\left(X_\phi\right)$. Jeśli bowiem $y\prec x$, to \mbox{$y\in \mP\left(X_\phi\right)$}, gdyż $X_\phi$ jest podkompleksem $X$; w~przeciwnym wypadku, gdy \mbox{$x\prec y$}, zachodzi oczywiście równość $(y,x)=\left(\tau_\psi,\sigma_\psi\right)$ dla pewnej liczby porządkowej $\psi< \phi$.

Przypuśćmy, że $(x_i)_{i\in \mN}$ jest nieskończoną ścieżką w~grafie $\mH_{M}(\mP(X))$. Przyjmijmy: \begin{align*}\phi_0&=\min\left\{\phi\leq \alpha:\text{istnieje liczba } i\in\mN \text{ taka, że } x_i\in X_\phi\right\},\\ i_0&=\min\left\{i\in\mN: x_{i_0}\in X_{\phi_0}\right\}.\end{align*}
Na podstawie obserwacji poczynionej w~poprzednim akapicie dowodu $x_j\in X_{\phi_0}$ dla wszystkich $j\geq i_0$. Ciąg $\left(x_{i_0+k}\right)_{k\in\mN}$ jest więc nieskończoną ścieżką w~podgrafie grafu $\mH_{M}(\mP(X))$ indukowanym na zbiorze wierzchołków $\mP\left(X_{\phi_0}\right)$. Ciąg taki oczywiście nie istnieje gdy $\phi_0=0$. Zatem $\phi_0>0$. Ponieważ $X_\phi=\bigcup_{\psi<\phi} X_\psi$ dla każdej granicznej liczby porządkowej $\phi\leq \alpha$, liczba $\phi_0$~jest następnikiem, $\phi_0=\psi_0+1$. Ponadto wobec~wyboru liczby $\phi_0$~zachodzi zawieranie \[\left\{x_{i_0+k}:k\in\mN\right\}\subseteq \mP\left(X_{\phi_0}\right)\smallsetminus \mP\left(X_{\psi_0}\right)=\left\{\tau_{\psi_0},\sigma_{\psi_0}\right\},\] co jest oczywistą sprzecznością. 

Graf skierowany $\mH_M(\mP(X))$ nie zawiera zatem nieskończonej ścieżki. Zgodnie z~lematem \ref{lem-niesk_sciezka_a_skojarzenie_morse} $M$~jest skojarzeniem Morse'a bez promieni malejących.
\end{proof}

Gdyby w~twierdzeniu \ref{maintw} ograniczyć się do rozważania regularnych CW kompleksów, lemat \ref{lem-charakteryzacja_inf_zgniatalnosci} mógłby posłużyć jako jeden z~głównych kroków jego dowodu (pełniąc w~nim rolę podobną do twierdzenia \ref{tw-formana_o_homotopijnej_rownowaznosci_miedzy_poziomami} w~dowodzie twierdzenia \ref{tw-glowne_tw_klasycznej_dyskretnej_teorii_morsea}).

\begin{lem}\label{lem-niesk_zgniatalnosc_mocny_retrakt_deformacyjny}
Niech $X, Y$~będą regularnymi CW kompleksami. Jeżeli $X\infcoll Y$, to istnieje mocna retrakcja deformacyjna $r\colon X\to Y$. 
\end{lem}
\begin{proof}
Załóżmy, że $X\infcoll Y$. Niech $\alpha$~będzie liczbą porządkową, zaś $\left(X_\phi\right)_{\phi<\alpha}$ ciągiem podkompleksów $X$~o~własnościach jak w~lemacie \ref{lem-charakteryzacja_inf_zgniatalnosci}. Dla wszystkich $\phi<\alpha$ włożenie $X_\phi\hookrightarrow X_{\phi+1}$ jest homotopijną równoważnością, gdyż $X_{\phi+1}\elcoll X_\phi$. Na podstawie lematu \ref{lem-o_nieskonczonym_skladaniu_homotopijnych_rownowaznosci} włożenie $Y\hookrightarrow X$ jest homotopijną równoważnością, a~zatem, wobec  lematu \ref{lem-wlozenie_hom_rown_to_retrakcja_sdr} oraz  stwierdzenia \ref{stw-domkniety_podzbior_anr_jest_korozwloknieniem}, podkompleks $Y$~jest mocnym retraktem deformacyjnym CW kompleksu $X$.
\end{proof}

Kolejny lemat, dotyczący możliwości ,,składania'' $\infty$-zgnieceń, jest natychmiastową konsekwencją lematu \ref{lem-charakteryzacja_inf_zgniatalnosci}.

\begin{lem}\label{lem-ciag_wstepujacy_zgniecen}
Niech $X$~będzie regularnym CW kompleksem, $\alpha>0$~liczbą porządkową, zaś $(X_\phi)_{\phi<\alpha}$ pozaskończonym ciągiem podkompleksów $X$~takim, że:
\begin{compactitem}
\item[---] $X_{\phi+1}\infcoll X_{\phi}$ dla każdej liczby porządkowej $\phi<\alpha$;
\item[---] $X_{\phi}=\bigcup_{\psi<\phi} X_{\psi}$ dla każdej granicznej liczby porządkowej $\phi\leq\alpha$;
\item[---] $X=X_\alpha$.
\end{compactitem}
Wówczas $X\infcoll X_0$. 
\end{lem}

Poniższy wynik jest w~przypadku skończonych kompleksów symplicjalnych dobrze znaną obserwacją (zob.~np.~\cite{Kahn84}).

\begin{lem}\label{lem-link_zgniatalny_to_star_zgniatalny}
Niech $K$~będzie kompleksem symplicjalnym, zaś $v\in K$ wierzchołkiem takim, że $\lk_K(v)\infcoll *$, tzn.~istnieje skojarzenie Morse'a bez promieni malejących $\widetilde{M}$ na $\lk_K(v)$ o~dokładnie jednej, $0$-wymiarowej komórce krytycznej $\{w\}$. Wówczas \[M=\left\{(\tau\cup\{v\},\sigma\cup\{v\}):(\tau,\sigma)\in\widetilde{M}\right\}\cup\bigl\{\left(\{v,w\},\{v\}\right)\bigr\}\] jest skojarzeniem Morse'a bez promieni malejących na $K$ wyznaczającym $\infty$-zgniecenie $K\infcoll K-v$.
\end{lem}
\begin{proof}
Oczywiście $M$~jest skojarzeniem. Ponieważ każdy sympleks $\lk_K(v)$, z~wyjątkiem $\{w\}$, należy do którejś z~krawędzi skojarzenia $\widetilde{M}$, każdy sympleks $\sigma\in K$ zawierający wierzchołek $v$~należy do którejś spośród krawędzi $M$. Nietrudno sprawdzić, że $M$~jest skojarzeniem Morse'a bez promieni malejących.
\end{proof}

\subsection{\texorpdfstring{(Ko)rozbieralność implikuje $\infty$-zgniatalność}{(Ko)rozbieralność implikuje ∞-zgniatalność}}
Poniżej opisujemy niektóre sytuacje, w~których rozbieralność oraz~korozbieralność implikują $\infty$-zgniatalność. Wyniki tego typu były dowodzone, zazwyczaj przy dużo mocniejszych niż w~tej sekcji założeniach, przez różnych autorów i~w~wielu wariantach \cite{Baclawski, Baclawski12,Barmak12,Kozlov06,Kozlov08}. % odpowiednio: [Theorem 8.1], [Proposition 23], [Remark 2.4], bez konkretnego twierdzenia,  [Theorems 13.12, 13.22]

\begin{lem}\label{lem-wierzch_zdominowany_to_star_collapsible}
Niech $K$~będzie kompleksem symplicjalnym, zaś $v\in K$ wierzchołkiem zdominowanym przez $w\in K$. Wówczas \[M=\{(\sigma,\sigma\smallsetminus\{w\}):\sigma\in \mP(K), \{v,w\}\subseteq\sigma\}\] jest skojarzeniem Morse'a bez promieni malejących na $K$, wyznaczającym $\infty$-zgniecenie $K\infcoll K-v$.
\end{lem}
\begin{proof}
Ponieważ \[\widetilde{M}=\bigl\{(\sigma,\sigma\smallsetminus\{w\}):\sigma\in\mP(\lk_K(v)), \sigma\not=\{w\}\bigr\}\] jest, jak łatwo zauważyć, skojarzeniem Morse'a bez promieni malejących na $\lk_K(v)$~wyznaczającym $\infty$-zgniecenie $\lk_K(v)\infcoll \{w\}$, teza wynika z~lematu \ref{lem-link_zgniatalny_to_star_zgniatalny}.
\end{proof}

\begin{lem}\label{lem-I-rozb_to_zgniatalne}
Niech $K$~będzie kompleksem symplicjalnym, zaś $L$~jego podkompleksem takim, że $K$~jest $\mItriang$-rozbieralny do $L$. Wówczas $K\infcoll L$.
\end{lem}
\begin{proof}
Ustalmy liczbę porządkową $\alpha$~oraz $\mItriang$-rozbierający $K$~do $L$~ciąg retrakcji $\left(\rho_{\phi,\phi+1}\colon K_{\phi}\to K_{\phi+1}\right)_{\phi<\alpha}$, zaś $\left(v_\phi\right)_{\phi<\alpha}$, $\left(w_\phi\right)_{\phi<\alpha}$ niech oznaczają ciągi wierzchołków kompleksu $K$~takie, że $K_{\phi+1}=K_{\phi}-v_{\phi}$ oraz $\rho_{\phi,\phi+1}\left(v_\phi\right)=w_{\phi}$ dla wszystkich $\phi<\alpha$. 

Dla każdej liczby porządkowej $\phi<\alpha$ niech \[M_\phi=\left\{\left(\sigma,\sigma\smallsetminus\left\{w_\phi\right\}\right):\sigma\in \mP\left(K_{\phi+1}\right), \left\{v_\phi,w_\phi\right\}\subseteq\sigma\right\}\] oznacza skojarzenie Morse'a bez promieni malejących wyznaczające \mbox{$\infty$-zgniecenie} $K_{\phi+1}\infcoll K_{\phi}$ (patrz lemat \ref{lem-wierzch_zdominowany_to_star_collapsible}).

Przyjmijmy $M=\bigcup_{\phi<\alpha} M_\phi$. Oczywiście $M$~jest skojarzeniem w~grafie skierowanym $\mH(\mP(K))$, a~$\mP(L)$~jest zbiorem tych elementów $\mP(K)$, które nie należą do żadnej krawędzi $M$. Wykażemy, że $M$~jest skojarzeniem Morse'a bez promieni malejących. 

Rozważmy w~tym celu relację częściowego porządku $\sqsubseteq$~na zbiorze $\mP(K)$ taką, że dla $x,y\in\mP(K)$ mamy $y\sqsubseteq x$, o~ile zachodzi jeden z~warunków:
\begin{compactitem}
\item[---]$\dim(y)< \dim(x)$;
\item[---]$\dim(y)=\dim(x)$ oraz $y\subseteq \bigcup_{\phi<\alpha} \infcomp\left(\rho_{\psi,\psi+1}\right)_{0\leq\psi<\phi}(x)$.
\end{compactitem}
Ponieważ ciąg $\left(\rho_{\phi,\phi+1}\right)_{\phi<\alpha}$ jest nieskończenie składalny, a~elementy $\mP(K)$ są skończonymi zbiorami, porządek $(\mP(K),\sqsubseteq)$ jest dobrze ufundowany. Udowodnimy metodą indukcji noetherowskiej ze względu na porządek $\sqsubseteq$, że każda ścieżka w~grafie $\mH_M(\mP(K))$ jest skończona.

Ustalmy $x\in \mP(K)$ i~załóżmy, że skończone są wszystkie ścieżki w~grafie $\mH_M(\mP(K))$ rozpoczynające się w~sympleksach $y\in \mP(K)$ takich, że $y\sqsubset x$. Rozważmy ścieżkę w~$\mH_M(\mP(K))$, której pierwszym wierzchołkiem jest $x$, zaś kolejnymi trzema $a,b,c\in\mP(K)$. Jeśli $(a,x)\not\in M$, to $a\subset x$, więc \mbox{$\dim(a)<\dim(x)$} i~w~konsekwencji $a\sqsubset x$. Jeżeli $(a,x)\in M$, to $a=x\cup\left\{w_\phi\right\}$, gdzie $\phi=\min\left\{\psi<\alpha:v_\psi\in x\right\}$. W~przypadku, gdy $b=a\smallsetminus\left\{v_\phi\right\}=x\cup\left\{w_\phi\right\}\smallsetminus\left\{v_\phi\right\}$, mamy $b\sqsubset x$, gdyż $w_\phi=\infcomp\left(\rho_{\psi,\psi+1}\right)_{0\leq\psi<\phi+1}\left(v_\phi\right)$. Jeśli natomiast $b=a\smallsetminus\{v\}$ dla pewnego $v\in a\smallsetminus\left\{w_\phi, v_\phi\right\}$, to ponieważ $\left(b,b\smallsetminus\left\{w_\phi\right\}\right)\in M$, a~$M$~jest skojarzeniem, krawędź $(c,b)\not\in M$, czyli $c=b\smallsetminus \{v'\}$ dla pewnego $v'\in b\smallsetminus\left\{w_\phi\right\}$. Oznacza to, że $\dim(c)<\dim(x)$, więc $c\sqsubset x$. Wykazaliśmy, że w~każdym przypadku któryś z~elementów ścieżki rozpoczynającej się w~$x$~jest mniejszy od $x$~w~porządku $\sqsubseteq$, co z~założenia indukcyjnego oznacza, że ścieżka ta jest skończona.

Na podstawie lematu \ref{lem-niesk_sciezka_a_skojarzenie_morse} $M$~jest skojarzeniem Morse'a bez promieni malejących.
\end{proof}

\begin{stw}\label{stw-rozb_to_zgniatalne}
Niech $K$~będzie kompleksem symplicjalnym,
zaś $L$~jego podkompleksem takim, że $K\dism L$ lub $L\codism K$. Wówczas $K\infcoll L$.
\end{stw}
\begin{proof}
Jeśli $K\dism L$, teza stanowi natychmiastowy wniosek z~lematów \ref{lem-mCtriang-rozbieralnosc-wtw-mItriang-rozbieralnosc} oraz \ref{lem-I-rozb_to_zgniatalne}.

Załóżmy, że $L\codism K$. Ustalmy liczbę porządkową $\beta$~oraz \mbox{$\mCtriang$-korozbierający} $K$~z~$L$~ciąg $\left(\varsigma_{\phi+1,\phi}\colon L_{\phi+1}\to L_\phi
\right)_{\phi<\beta}$. Ponieważ $L_{\phi+1}\dism L_{\phi}$ dla każdej liczby porządkowej $\phi<\beta$, na podstawie pierwszej części stwierdzenia $L_{\phi+1}\infcoll L_\phi$. Zastosowanie lematu \ref{lem-ciag_wstepujacy_zgniecen} kończy dowód.
\end{proof}

Zauważmy, że lemat \ref{lem-niesk_zgniatalnosc_mocny_retrakt_deformacyjny} w~połączeniu ze stwierdzeniem \ref{stw-rozb_to_zgniatalne} dostarcza alternatywnego dowodu stwierdzenia \ref{stw-zgniatalny_jest_sciagalny}.

Korzystając z~lematu \ref{lem-rozbieralnosc_tu_i_tu} oraz~stwierdzenia \ref{stw-rozb_to_zgniatalne} moglibyśmy udowodnić, że jeśli $P$~jest częściowym porządkiem bez nieskończonych łańcuchów, $A\subseteq P$ oraz \mbox{$P\dism A$}, to $\mK(P)\infcoll \mK(A)$. Założenie o~braku nieskończonych łańcuchów w~$P$~można jednak pominąć, co wykażemy korzystając z~następującego lematu, główna idea dowodu którego pochodzi z~książki Kozlova \cite{Kozlov08}.

\begin{lem}[por.~{\cite[Remark 13.13]{Kozlov08}}]\label{lem-kozlova_rozbieralnosc_implikuje_zgniatalnosc}
Niech $P$~będzie częściowym porządkiem, zaś $r\colon P\to r(P)$ retrakcją należącą do klasy $(\mathcal{U}\cup\mathcal{D})$. Wówczas $\mK(P)\infcoll \mK(r(P))$.
\end{lem}
\begin{proof}
Bez utraty ogólności możemy zakładać, że $r\in \mathcal{D}$.

Każdy element $x\in\mP(\mK(P))\smallsetminus \mP(\mK(r(P))$ jest niepustym, skończonym, liniowo uporządkowanym podzbiorem $P$~postaci $x=\bigl\{x_0<x_1<\ldots <x_{\dim(x)}\bigr\}$. Niech $i_x=\min\{i:x_i\not=r(x_i)\}$. Przyjmujemy:
\[M\!=\!\left\{\left(x\cup \left\{r\left(x_{i_x}\right)\right\},x\right)\!:\!x\in\mP(\mK(P))\!\smallsetminus\!\mP(\mK(r(P))),\ i_x=0 \text{ lub }r(x_{i_x})\!\not=\!x_{i_x-1}\right\}.\]
Wykażemy, że $M$~jest skojarzeniem Morse'a bez promieni malejących na kompleksie symplicjalnym $\mK(P)$, którego zbiorem elementów krytycznych jest $\mK(r(P))$.

Ponieważ $r\in\mathcal{D}$, to dla każdego $x\in\mP(\mK(P))$ mamy $r\left(x_{i_x}\right)<x_{i_x}$ oraz, o~ile $i_x>0$, $x_{i_x-1}=r\left(x_{i_x-1}\right)<r\left(x_{i_x}\right)$. Zatem $x\cup\left\{r\left(x_{i_x}\right)\right\}$ jest elementem $\mP(\mK(P))$, czyli $M\subseteq \mP(\mK(P))\times \mP(\mK(P))$ jest podzbiorem zbioru krawędzi grafu skierowanego $\mH(\mP(\mK(P)))$.

Nietrudno spostrzec, że element $x\in \mP(\mK(P))$ należy do \mbox{$\mP(\mK(P))\smallsetminus \mP(\mK(r(P)))$} wtedy i~tylko wtedy, gdy jest elementem krawędzi ze zbioru $M$. Łatwo również sprawdzić, że $M$~jest skojarzeniem w~grafie $\mH(\mP(\mK(P)))$.

Wobec lematu \ref{lem-niesk_sciezka_a_skojarzenie_morse} pozostaje do udowodnienia, że nie istnieje nieskończona ścieżka w~grafie skierowanym $\mH_M(\mP(\mK(P)))$. Oczywiście skończona jest każda ścieżka rozpoczynające się w~wierzchołku należącym do zbioru $\mP(\mK(r(P)))$. Dla ścieżek rozpoczynających się w~wierzchołku $x\in\mP(\mK(P))\smallsetminus\mP(\mK(r(P)))$ dowód przeprowadzimy metodą indukcji ze względu na $\dim(x)$ oraz $\dim(x)-i_x$. Ustalmy w~tym celu $x\in\mP(\mK(P))\smallsetminus\mP(\mK(r(P)))$ i~załóżmy, że każda ścieżka w~$\mH_M(\mP(\mK(P)))$ rozpoczynająca się w~wierzchołku $y\in \mP(\mK(P))\smallsetminus\mP(\mK(r(P)))$ takim, że $\dim(y)<\dim(x)$ lub $\dim(y)=\dim(x)$ oraz $\dim(y)-i_y<\dim(x)-i_x$, jest skończona.

Element $x$~jest skończonym, niepustym zbiorem liniowo uporządkowanym $x=\bigl\{x_0<\ldots<x_{\dim(x)}\bigr\}\subseteq P$. Rozważmy ścieżkę w~$\mH_M(\mP(\mK(P)))$ rozpoczynającą się w~$x$, której kolejnymi elementami są $a,b,c\in\mP(\mK(P))$. Możemy zakładać, że $a,b,c\not\in\mP(\mK(r(P))$. Jeżeli $a\subsetneq x$, to ścieżka ta jest skończona z~założenia indukcyjnego, gdyż $\dim(a)<\dim(x)$. Załóżmy wobec tego, że $a\supsetneq x$. Oznacza to, że $(a,x)\in M$, czyli~$a=\bigl\{x_0<\ldots<r(x_{i_x})<x_{i_x}<\ldots<x_{\dim(x)}\bigr\}$. Ponieważ $M$~jest skojarzeniem, $b=a\smallsetminus\left\{x_j\right\}=x\cup\left\{r\left(x_{i_x}\right)\right\}\smallsetminus \left\{x_j\right\}$ dla pewnego $j\in\{0,\ldots,\dim(x)\}$. Jeśli $j\not=i_x$, to $\left(b,x\smallsetminus\left\{x_j\right\}\right)\in M$, więc $c=a\smallsetminus\bigl\{x_j,x_{j'}\bigr\}$ dla pewnego $j'\in \{0,\ldots,\dim(x)\}\smallsetminus\{j\}$; wówczas $\dim(c)<\dim(x)$, czyli rozważana ścieżka jest skończona z~założenia indukcyjnego. Jeżeli natomiast $j=i_x$, to zauważmy, że $\dim(b)=\dim(x)$, ale $i_b=i_{a\smallsetminus\{x_{i_x}\}}>i_x$, więc $\dim(b)-i_b<\dim(x)-i_x$. Rozważana ścieżka jest więc, wobec założenia indukcyjnego, skończona.
\end{proof}

\begin{stw}\label{stw-rozb_posetow_implikuje_collapsibility_kompleksow}
Niech $P$~będzie częściowym porządkiem, zaś $A\subseteq P$ takim jego podzbiorem, że $P\dism A$ lub $A\codism P$. Wówczas $\mK(P)\infcoll \mK(A)$.
\end{stw}
\begin{proof}
Załóżmy, że $P\dism A$. Na podstawie lematu \ref{lem-Irozbieralny-wtw-Crozbieralny} oraz następującej po nim~uwagi odnośnie dowodu, istnieją liczba porządkowa $\alpha$~oraz ciąg $\left(r_{\phi,\phi+1}\colon P_\phi\to P_{\phi+1}\right)_{\phi<\alpha}$ retrakcji $(\mathcal{U}\cup \mathcal{D})$-rozbierający $P$~do $A$. 

Dla $\phi<\alpha$ niech $M_\phi$~będzie dla $r_{\phi,\phi+1}$ skojarzeniem Morse'a na $\mK\left(P_\phi\right)$ zdefiniowanym jak w~dowodzie lematu \ref{lem-kozlova_rozbieralnosc_implikuje_zgniatalnosc} oraz niech $M=\bigcup_{\phi<\alpha}M_\phi$. Wobec lematu \ref{lem-kozlova_rozbieralnosc_implikuje_zgniatalnosc} zbiór $M$~jest skojarzeniem w~grafie $\mH(\mP(\mK(P)))$, a~wierzchołki tego grafu nie należące do żadnej krawędzi skojarzenia $M$~tworzą zbiór $\mP(\mK(A))$. 

Wykażemy, że graf skierowany $\mH_M(\mP(\mK(P)))$ nie zawiera nieskończonej ścieżki, co na podstawie lematu \ref{lem-niesk_sciezka_a_skojarzenie_morse} oznaczało będzie, że $M$~jest skojarzeniem Morse'a bez promieni malejących.

Podobnie jak w~dowodzie lematu \ref{lem-I-rozb_to_zgniatalne} rozważmy dobrze ufundowany częściowy porządek $\sqsubseteq$~na zbiorze $\mP(\mK(P))$ taki, że dla $x,y\in\mP(\mK(P))$ mamy $y\sqsubseteq x$, o~ile zachodzi jeden z~warunków:
\begin{compactitem}
\item[---] $\dim(y)<\dim(x)$;
\item[---] $\dim(y)=\dim(x)$ oraz $y\subseteq \bigcup_{\phi<\alpha} \infcomp \left(r_{\psi,\psi+1}\right)_{0\leq \psi<\phi}(x)$.
\end{compactitem} 
Przeprowadzimy indukcję noetherowską ze względu na porządek $\sqsubseteq$.

Ustalmy $x\in \mP(\mK(P))$ i~załóżmy, że wszystkie ścieżki w~grafie $\mH_M(\mP(\mK(P)))$ rozpoczynające się w~wierzchołkach $y\in \mP(\mK(P))$ takich, że $y\sqsubset x$, są skończone. Rozważmy ścieżkę w~$\mH_M(\mP(\mK(P)))$, której pierwszym wierzchołkiem jest $x$, zaś kolejnymi trzema $a,b,c\in\mP(\mK(P))$. Jak zauważyliśmy w~dowodzie lematu \ref{lem-kozlova_rozbieralnosc_implikuje_zgniatalnosc}, albo $(a,x)\not\in M$ i~wówczas $a\sqsubset x$, albo $(a,x)\in M$, tzn.~$(a,x)\in M_\phi$ dla pewnego $\phi<\alpha$. W~tym ostatnim przypadku zachodzi jedna z~możliwości: $c\subsetneq x$, czyli $c\sqsubset x$, albo $b=(x\smallsetminus\{x_i\})\cup \left\{r_{\phi,\phi+1}(x_i)\right\}\subseteq \bigcup_{\phi<\alpha} \infcomp \left(r_{\psi,\psi+1}\right)_{0\leq \psi<\phi}(x)$ dla pewnego $i\in\{0,\ldots,\dim(x)\}$ takiego, że $r_{\phi,\phi+1}(x_i)\not=\{x_i\}$, co oznacza, że $b\sqsubset x$. W~każdym wypadku rozważana ścieżka jest na podstawie założenia indukcyjnego skończona. W~konsekwencji $M$~jest skojarzeniem Morse'a bez promieni malejących; wyznacza ono $\infty$-zgniecenie $\mK(P)\infcoll \mK(A)$.

Jeżeli $A\codism P$, teza stwierdzenia wynika natychmiast z~lematu \ref{lem-Ckorozb_wtw_Ikorozb} (wraz z~następującą po nim uwagą) oraz lematów \ref{lem-ciag_wstepujacy_zgniecen},~\ref{lem-kozlova_rozbieralnosc_implikuje_zgniatalnosc}.
\end{proof}

% UOGÓLNIĆ CHYBA MOŻNA KOZLOVA TWIERDZENIE O COLLAPSING ALONG MONOTONE POSET MAPS: 
% jeśli P (bez nsk. łańcuchów ?????) i f ~ id, to \mK(P) zgniatalne do \mK(f(P))
% a także:
% JEŚLI phi sąsiednie id, to K zgniatalne do phi(K) (bo przeprowadzamy wierzcholki v nalezace do K - phi(K) na phi(v), a sa one przez phi(v) zdominowane!) 
% ALE TO SOBIE ODPUŚCIMY -- MOŻE W PUBLIKACJI? 

%------------------------------------------------------------------
\subsection{Kraty bez dopełnień}
Badanie topologicznych własności ściętych krat bez dopełnień ma dość długą historię. Z~pracy Crapo \cite{Crapo66} z~1966~roku wynika (dzięki związkowi między tzw.~funkcją M{\"o}biusa częściowego porządku a~jego charakterystyką Eulera \cite[(9.14)]{Bjorner95}), że charakterystyka Eulera skończonej, ściętej kraty bez dopełnień jest równa $1$. Wynik ten stanowił motywację dla Baclawskiego \cite[Corollary 6.3]{Baclawski77}, który wykazał (korzystając między innymi z~ciągu spektralnego Leraya), że krata taka ma trywialne homologie całkowitoliczbowe, a~następnie wraz z~Bj{\"o}rnerem \cite[Theorem 3]{Baclawski81} zastosował dowód w~nieco ogólniejszej sytuacji. W~późniejszej pracy Bj{\"o}rner \cite[Theorem 3.3]{Bjorner81} udowodnił ściągalność realizacji geometrycznej kompleksu symplicjalnego stowarzyszonego z~kratą bez mocnych dopełnień (bez założenia o~skończoności tej kraty). Ważne i~daleko idące uogólnienia tego wyniku zawiera praca Bj{\"o}rnera i~Walkera \cite{Bjorner83}. 

Kozlov \cite[Theorem 2.4]{Kozlov98} oraz~Baclawski \cite[Theorem 27]{Baclawski12} udowodnili, iż kompleks symplicjalny stowarzyszony ze skończoną, ściętą kratą bez dopełnień (w~przypadku Baclawskiego również bez mocnych dopełnień) jest zgniatalny (a~nawet ma silniejszą własność \mbox{\textit{non-evasiveness}}). W~niepublikowanych notatkach Baclawski \cite{Baclawski} częściowo przeniósł ten wynik na nieskończone kraty. Celem bieżącej sekcji, w~dużej mierze opartej na wspomnianych notatkach \cite{Baclawski},  jest przedstawienie dopracowanych i~nieco uogólnionych niepublikowanych rezultatów Baclawskiego jako przykładu zastosowania pojęcia $\infty$-zgniatalności.

Idea poniższego wyniku oraz jego dowodu pochodzi z~notatek Baclawskiego \cite{Baclawski}; podane  sformułowanie jest ogólniejsze niż w~oryginale.
% TYM MOZNA CHYBA UDOWODNIC LEMAT KOZLOVA, ALE... NIE MA DOBREJ KONTROLI NAD MATCHINGIEM WTEDY, WIEC NASTEPUJACE PO LEMACIE STWIERDZENIE NIE PRZEJDZIE 
\begin{lem}[por.~{\cite[Theorem 6.2, Proposition 10.2]{Baclawski}}]\label{lem-prop62-Baclawskiego}
Niech $K$~będzie kompleksem symplicjalnym o~zbiorze wierzchołków $V$ oraz niech $W\subseteq V$, $U=V\smallsetminus W$. Jeżeli $\st_K(\sigma)\big|_U\infcoll *$ dla każdego sympleksu $\sigma$~kompleksu $K\big|_W$, to $K\infcoll K\big|_U$.
\end{lem}
\begin{proof}
Ustawmy elementy zbioru $W$~w~ciąg pozaskończony: $W=\left\{w_\phi\right\}_{\phi<\alpha}$ dla pewnej liczby porządkowej $\alpha$. Jeśli $\alpha=0$, tzn.~$W=\emptyset$, to $K=K\big|_U$, więc $K\infcoll K\big|_U$ w~sposób trywialny.

Załóżmy, że lemat jest prawdziwy dla wszystkich kompleksów symplicjalnych $L$~oraz podzbiorów $W_L$~zbiorów ich wierzchołków takich, że $W_L=\left\{v_\phi\right\}_{\phi<\beta}$ dla pewnej liczby porządkowej $\beta<\alpha$.

Rozważmy najpierw sytuację, gdy $\alpha$~jest następnikiem: $\alpha=\gamma+1$. Wykażemy, że $\lk_K\left(w_\gamma\right)\infcoll *$. Przyjmijmy  oznaczenie $\widetilde{K}=\lk_K(w_\gamma)$. Niech $\widetilde{V}$~będzie zbiorem wierzchołków kompleksu $\widetilde{K}$, zaś \begin{align*}&\widetilde{W}=\left\{w_\phi\right\}_{\phi<\gamma}\cap \widetilde{V},& \widetilde{U}=\widetilde{V}\smallsetminus \widetilde{W}=\widetilde{V}\cap U.\end{align*} Oczywiście elementy zbioru $\widetilde{W}$~można, dla pewnej liczby porządkowej $\beta<\alpha$, ustawić w~ciąg pozaskończony $\left\{\widetilde{w}_\phi\right\}_{\phi<\beta}$. Dla każdego sympleksu $\sigma$~kompleksu $\widetilde{K}\big|_{\widetilde{W}}$ mamy $\st_{\widetilde{K}}(\sigma)\big|_{\widetilde{U}}=\st_K\left(\sigma\cup\left\{w_\gamma\right\}\right)\big|_U$, więc $\st_{\widetilde{K}}(\sigma)\big|_{\widetilde{U}}\infcoll *$ zgodnie z~założeniami lematu. Na podstawie~założenia indukcyjnego $\widetilde{K}\infcoll \widetilde{K}\big|_U$. Zauważmy jednak, że $\widetilde{K}\big|_{\widetilde{U}}=\st_K\left(w_\gamma\right)\big|_U$, więc $\widetilde{K}\big|_{\widetilde{U}}\infcoll *$ z~założeń lematu. Z~lematu \ref{lem-ciag_wstepujacy_zgniecen} otrzymujemy $\widetilde{K}\infcoll *$. 

Ponieważ $\lk_K\left(w_\gamma\right)=\widetilde{K}\infcoll*$, na podstawie lematu \ref{lem-link_zgniatalny_to_star_zgniatalny} zachodzi \mbox{$\infty$-zgniecenie} $K\infcoll K-w_\gamma$. Z~założenia indukcyjnego \[K-w_\gamma\infcoll K-w_\gamma\big|_{U\smallsetminus \left\{w_\gamma\right\}}=K\big|_U.\] Wobec lematu \ref{lem-ciag_wstepujacy_zgniecen} oznacza to, że $K\infcoll K\big|_U$.

Dowiedliśmy tezy indukcyjnej w~przypadku, gdy $\alpha$~jest następnikiem. Załóżmy, że liczba porządkowa $\alpha$~jest graniczna. Dla $\phi\leq \alpha$ niech $W_\phi=\left\{w_\psi\right\}_{\psi<\phi}$ oraz $K_\phi=K\big|_{U\cup W_\phi}$. Jak zauważyliśmy wyżej, jeśli $\phi<\alpha$, to $K_{\phi+1}\infcoll K_\phi$. Zatem $K=K_\alpha\infcoll K_0=K\big|_U$ na podstawie lematu \ref{lem-ciag_wstepujacy_zgniecen}.
\end{proof}

Poniższe twierdzenie uogólnia niepublikowany wynik Baclawskiego \cite[Theorem 9.1]{Baclawski}, który udowodnił je przy dodatkowym założeniu, że krata $L$~jest skończonej wysokości. (Podobne wyniki dla krat skończonych uzyskali wcześniej Baclawski \cite[Theorem 26]{Baclawski12} oraz Kozlov \cite[Theorem 2.4]{Kozlov98}.) Prezentowany niżej dowód różni się od przedstawionego przez Baclawskiego \cite{Baclawski}~przede wszystkim zastosowaniem stwierdzenia \ref{stw-rozb_posetow_implikuje_collapsibility_kompleksow}, nieco ogólniejszego niż jego wykorzystany w~oryginalnym dowodzie odpowiednik.

\begin{tw}[por.~{\cite[Theorem 9.1]{Baclawski}}]\label{tw-baclawskiego-kraty-bez-dopelnien-sa-zgniatalne}
Niech $L$~będzie kratą z~zerem i~jedynką. Załóżmy, że $x\in \check{L}$, zaś $B\subseteq \check{L}$ jest zbiorem zawierającym wszystkie dopełnienia elementu $x$~oraz zawartym w~zbiorze dolnych półdopełnień tego elementu. Wówczas $\mK\left(\check{L}\smallsetminus B\right)\infcoll *$. 
\end{tw}
\begin{proof}
Przyjmijmy oznaczenie $P=\check{L}\smallsetminus B$. Niech \begin{align*}\operatorname{lsc}(x)&=\left\{y\in\check{L}:x\land y=\lbottom_L\right\},\\ \operatorname{c}(x)&=\left\{y\in\check{L}:x\land y=\lbottom_L,\ x\lor y=\ltop_L\right\}\end{align*} oznaczają odpowiednio zbiór dolnych półdopełnień elementu $x$~oraz zbiór dopełnień tego elementu. Z~założenia $\operatorname{c}(x)\subseteq B\subseteq \operatorname{lsc}(x)$. Niech $W=\operatorname{lsc}(x)\smallsetminus B$ oraz $U=\check{L}\smallsetminus \operatorname{lsc}(x)$.

Jeżeli $p\in U$, $q\in P$ oraz $q\geq p$, to $q\land x\geq p\land x> \lbottom_L$, więc $q\in U$. Podobnie, jeśli $p\in W$, $q\in P$, $q\leq p$, to $q\land x\leq p\land x=\lbottom_L$.

Zastosujemy lemat \ref{lem-prop62-Baclawskiego} do kompleksu $\mK(P)$ i~wyżej określonych zbiorów $U,W$. Wykażemy w~tym celu, że dla każdego sympleksu $\sigma$~kompleksu $\mK(W)=\mK(P)\big|_W$ zachodzi \mbox{$\infty$-zgniecenie} $\st_{\mK(P)}(\sigma)\big|_U\infcoll *$.

Rozważmy sympleks $\sigma$~kompleksu $\mK(W)$. Jest on niepustym, skończonym podzbiorem liniowo uporządkowanym $\{w_0<w_1<\ldots<w_n\}\subseteq W$. Każdy sympleks $\tau$~kompleksu $\st_{\mK(P)}(\sigma)\big|_U$ jest niepustym, skończonym, liniowo uporządkowanym podzbiorem $U$~takim, że zbiór $\sigma\cup \tau\subseteq P$ jest liniowo uporządkowany. Zauważmy, że takie podzbiory to dokładnie skończone, niepuste łańuchy w~zbiorze $U\cap \left(w_n\mathord{\uparrow}_P\right)$, czyli $\st_{\mK(P)}(\sigma)\big|_U=\mK(U\cap (w_n\mathord{\uparrow}_P))$. 

Przyjmijmy oznaczenie $w=w_n$. Ponieważ \[w\in W=\operatorname{lsc}(x)\smallsetminus B\subseteq \operatorname{lsc}(x)\smallsetminus\operatorname{c}(x),\] to $w\lor x<\ltop_L$, więc $w\lor x\in w\mathord{\uparrow}_P$. Ponadto $x \land (w\lor x)=x\not=\lbottom_L$, a~zatem $w\lor x\in U$. Rozważmy zachowujące porządek odwzorowanie \[r_0\colon U\cap (w\mathord{\uparrow}_P)\to r_0(U\cap (w\mathord{\uparrow}_P))\subseteq L\] zadane dla $u\in U\cap (w\mathord{\uparrow}_P)$ wzorem $r_0(u)=u\land (w\lor x)$. Wykażemy, że \mbox{$r_0(U\cap (w\mathord{\uparrow}_P))$} jest podzbiorem $U\cap (w\mathord{\uparrow}_P)$. Ustalmy w~tym celu $u\in U\cap (w\mathord{\uparrow}_P)$. Ponieważ $u\geq w$ oraz $w\lor x\geq w$, mamy $r_0(u)=u\land (w\lor x)\geq w$, czyli $r_0(u)\in w\mathord{\uparrow}_L$. Z~drugiej strony \[r_0(u)\land x=(u\land (w\lor x))\land x=u\land((w\lor x)\land x)=u\land x\not=\lbottom_L,\] gdyż $u\in U$. Zatem $r_0(u)\in U\cap (w\mathord{\uparrow}_L)=U\cap (w\mathord{\uparrow}_P$). Zauważmy, że $r_0(u)\leq u$, a~ponadto \[r_0(r_0(u))\!=\!(u\land (w\lor x))\!\land\!(w\lor x)\!=\!u\land\!((w\lor x)\!\land\!(w\lor x))\!=\!u\land\!(w\lor x)\!=\!r_0(u),\] czyli $r_0$~jest $\mathcal{D}$-retrakcją. Odwzorowanie stałe $r_1\colon r_0(U\cap (w\mathord{\uparrow}_P))\to \{w\lor x\}$ jest \mbox{$\mathcal{U}$-retrakcją}. Zatem $\{w\lor x\}\codism \mK(U\cap (w\mathord{\uparrow}_P))=\st_{\mK(P)}(\sigma)\big|_U$, czyli $\st_{\mK(P)}(\sigma)\big|_U\infcoll *$ na podstawie stwierdzenia \ref{stw-rozb_posetow_implikuje_collapsibility_kompleksow}.

Wobec lematu \ref{lem-prop62-Baclawskiego} zachodzi \mbox{$\infty$-zgniecenie} $\mK(P)\infcoll \mK(P)\big|_U=\mK(U)$. Udowodnimy, że $\mK(U)\infcoll *$, co będzie, dzięki lematowi \ref{lem-ciag_wstepujacy_zgniecen}, oznaczało, iż $\mK(P)\infcoll *$.

Rozważmy zachowujące porządek odwzorowanie $r_0'\colon U\to r_0'(U)\subseteq L$ dane dla $u\in U$ wzorem $r_0'(u)=x\land u$. Wykażemy, że $r_0'(U)\subseteq U$. W~tym celu ustalmy $u\in U$. Ponieważ $u\not\in\operatorname{lsc}(x)$, mamy $x\land u>\lbottom_L$, czyli $r_0'(u)\in\check{L}$. Ponadto $r_0'(u)=x\land u<x$, więc $x\land r_0'(u)=r_0'(u)>\lbottom_L$, a~zatem $r_0'(u)\in U$. Zauważmy także, że $r_0'(r_0'(u))=x\land r_0'(u)=r_0'(u)$ oraz $r_0'(u)=x\land u<u$, czyli funkcja $r_0'$~jest $\mathcal{D}$-retrakcją na zbiorze $U$. Odwzorowanie stałe $r_1'\colon r_0'(U)\to \{x\}$ jest $\mathcal{U}$-retrakcją. Wobec tego $\{x\}\codism U$, więc $\mK(U)\infcoll *$ na podstawie stwierdzenia \ref{stw-rozb_posetow_implikuje_collapsibility_kompleksow}.
\end{proof}

Oczywiście twierdzenie \ref{tw-baclawskiego-kraty-bez-dopelnien-sa-zgniatalne} pozostaje prawdziwe, gdy zbiór dolnych półdopełnień w~jego sformułowaniu zastąpimy zbiorem górnych półdopełnień.

Jeżeli $L$~jest kratą bez dopełnień, możemy w~twierdzeniu \ref{tw-baclawskiego-kraty-bez-dopelnien-sa-zgniatalne} przyjąć $B=\emptyset$, co prowadzi do następującego wniosku.

\begin{wn}\label{wn-baclawskiego-o-kratach-bez-dopelnien}
Jeżeli $L$~jest kratą z~zerem i~jedynką, bez dopełnień, to $\mK\left(\check{L}\right)\!\infcoll *$.
\end{wn}

Przy dodatkowym założeniu, że krata $L$~nie zawiera nieskończonego łańcucha, jesteśmy w~stanie wykazać $\infty$-zgniatalność kompleksu $\mK\left(\check{L}\right)$, o~ile $L$~nie ma mocnych dopełnień. Wynik ten nieznacznie uogólnia niepublikowane twierdzenie Baclawskiego \cite[Theorem 9.2]{Baclawski}, zawierające mocniejsze założenie o~skończonej wysokości kraty $L$. (Analogiczny wynik dla skończonych krat znajduje się w~artykule Baclawskiego \cite[Theorem 27]{Baclawski12}.) Idea zaprezentowanego niżej dowodu pochodzi z~notatek Baclawskiego \cite{Baclawski}; od oryginalnego rozumowania różni się on szczegółami technicznymi. 

\begin{tw}[por.~{\cite[Theorem 9.2]{Baclawski}}]\label{tw-baclawskiego-o-kratach-bez-mocnych-dopelnien}
Jeżeli $L$~jest kratą bez mocnych dopełnień, która nie zawiera nieskończonego łańcucha, to $\mK\left(\check{L}\right)\infcoll *$.
\end{tw}
\begin{proof}
Ustalmy element $x\in \check{L}$ taki, że  nie istnieje dopełnienie $x$~będące kresem górnym zbioru zawartego w~$\min\left(\check{L}\right)$ bądź nie istnieje dopełnienie $x$~będące kresem dolnym zbioru zawartego w~$\max\left(\check{L}\right)$. Bez utraty ogólności możemy zakładać, że zachodzi pierwsza z~tych możliwości.
Symbolem $\operatorname{c}(x)$ oznaczmy zbiór dopełnień elementu $x$.

Określamy indukcyjnie pewien pozaskończony ciąg podzbiorów zbioru $\operatorname{c}(x)$, dla każdej liczby porządkowej $\phi$~przyjmując \[C_\phi=\min\left(\operatorname{c}(x)\smallsetminus \bigcup_{\psi<\phi}C_\psi\right).\] Niech $\beta$~będzie najmniejszą liczbą porządkową o~tej własności, że $C_\beta=\emptyset$. Częściowy porządek $L$~nie zawiera nieskończonych łańcuchów, więc jest dobrze ufundowany, skąd wynika, że zachodzi równość $\operatorname{c}(x)\smallsetminus \bigcup_{\psi<\beta}C_\psi=\emptyset$.

Dla wszystkich $\phi\leq\beta$~niech \[
L_\phi=\check{L}\smallsetminus \bigcup_{\phi\leq \psi<\beta} C_\psi\] oraz $K_\phi=\mK\left(L_\phi\right)$.
Wykażemy, że dla każdej liczby porządkowej $\phi<\beta$ zachodzi \mbox{$\infty$-zgniecenie} $K_{\phi+1}\infcoll K_{\phi}$. Ustalmy w~tym celu $\phi<\beta$.

Zastosujemy lemat \ref{lem-prop62-Baclawskiego} do kompleksu $K=K_{\phi+1}$ oraz zbiorów $W=C_\phi$, $U=\check{L}\smallsetminus \bigcup_{\phi\leq \psi<\beta} C_\psi$. Ponieważ zbiór $C_\phi$ jest antyłańcuchem, wszystkie sympleksy kompleksu $K_{\phi+1}\big|_{C_\phi}=\mK\left(C_\phi\right)$ są $0$-wymiarowe oraz $\st_{K_{\phi+1}}(\{c\})\big|_U=\lk_{K_{\phi+1}}(c)$ dla każdego $c\in C_{\phi}$. Ustalmy $c\in C_{\phi}$. Ponieważ $c\in\operatorname{c}(x)$, z~wyboru elementu $x$~nie istnieje podzbiór zbioru $\min\left(\check{L}\right)$, którego kresem górnym jest $c$. Jeśli więc $z=\sup\left(\min\left(c\mathord{\downarrow}_{\check{L}}\right)\right)$ (ten kres górny istnieje, gdyż $L$~nie zawiera nieskończonych łańcuchów, jest więc kratą zupełną na podstawie lematu \ref{lem-krata_bez_nsk_lancuchow_jest_zupelna}), to $z<c$, a~zatem $z\in L_{\phi}$.

Niech $Q=\{q\in L_\phi: q\sim c\}$. Określmy zachowujące porządek odwzorowanie $r_0\colon Q\to r_0(Q)\subseteq L$, zadane dla $q\in Q$ wzorem $r_0(q)=q\land z$. Wykażemy, że $r_0(Q)\subseteq Q$, a~funkcja $r_0$~jest $\mathcal{D}$-retrakcją. Dla $q\in Q$ mamy $r_0(q)\leq z< c$. Wystarczy zatem pokazać, że $r_0(q)\not=\lbottom_L$. Jeśli $q>c$, to $q>z$, więc $r_0(q)=z>\lbottom_L$. Jeżeli zaś $q<c$, to $q\geq m$ dla pewnego elementu $m\in \min\left(c\mathord{\downarrow}_{\check{L}}\right)$, więc $r_0(q)\geq m>\lbottom_L$. Oczywiście $r_0(q)\leq q$ oraz $r_0$~jest retrakcją. Funkcja stała $r_1\colon r_0(Q)\to \{z\}$ jest \mbox{$\mathcal{U}$-retrakcją}, wobec czego $*\codism \mK(Q)=\lk_{K_{\phi+1}}(c)$. Na podstawie stwierdzenia \ref{stw-rozb_posetow_implikuje_collapsibility_kompleksow} kompleks $\lk_{K_{\phi+1}}(c)\infcoll *$. Z~lematu \ref{lem-prop62-Baclawskiego} wnioskujemy, że \mbox{$K_{\phi+1}\infcoll K_{\phi+1}\big|_U= K_{\phi}$}.

Stosując lemat \ref{lem-ciag_wstepujacy_zgniecen} do ciągu $\left(K_\phi\right)_{\phi\leq\beta}$ otrzymujemy \[\mK\left(\check{L}\right)=K_\beta\infcoll K_0=\mK(\check{L}\smallsetminus\operatorname{c}(x)).\] Na podstawie twierdzenia \ref{tw-baclawskiego-kraty-bez-dopelnien-sa-zgniatalne} zachodzi $\infty$-zgniecenie $\mK\left(\check{L}\smallsetminus \operatorname{c}(x)\right)\infcoll *$. Zastosowanie lematu \ref{lem-ciag_wstepujacy_zgniecen} kończy dowód.
\end{proof}

Bj{\"o}rner \cite[Theorem 3.3]{Bjorner81} udowodnił, że kompleks $\mK\left(\check{L}\right)$ ma ściągalną realizację geometryczną dla dowolnej (również zawierającej nieskończone łańcuchy) kraty $L$~bez mocnych dopełnień. Zastanawiające jest, czy można przy tych założeniach wykazać $\infty$-zgniatalności $\mK\left(\check{L}\right)$.
 
\begin{problem}\label{prob6}
Czy jeśli $L$~jest dowolną kratą z~zerem i~jedynką, bez mocnych dopełnień, to $\mK\left(\check{L}\right)\infcoll *$?
\end{problem}

Twierdzenie \ref{tw-baclawskiego-o-kratach-bez-mocnych-dopelnien} posłuży w~sekcji \ref{subsec-tw_Baclawskiego} jako fragment kombinatorycznego dowodu twierdzenia o~istnieniu punktu stałego zachowującego porządek odwzorowania ściętej kraty bez promieni i~bez mocnych dopełnień. 

\begin{comment}
% -----------------------------------------------------------------
\subsection{Zgniatalność kompleksów $M_\kappa$-symplicjalnych}
Korzystając z~pomysłów K.~Adiprasito i~B.~Benedettiego \cite{Adiprasito13}, którzy udowodnili podobny wynik dla skończonych kompleksów symplicjalnych, wykażemy \mbox{$\infty$-zgniatalność} kompleksów symplicjalnych, na których można wprowadzić strukturę kompleksu $M_\kappa$-symplicjalnego o~pewnych szczególnych własnościach. Wykorzystamy w~tym celu pojęcia funkcji z~minimami na gwiazdach\footnote{ang.~\textit{star-minimal}} oraz skojarzenia gradientowego, wprowadzone przez wspomnianych autorów \cite{Adiprasito13}. (Wyniki dotyczące zgniatalności $\CAT(0)$ kompleksów symplicjalnych uzyskała wcześniej Crowley \cite{Crowley08}; zob.~też pracę Barali{\'c} i~Laz{\u a}r \cite{Baralic} wraz z~cytowanymi w~niej publikacjami.)

Niech $K$~będzie kompleksem symplicjalnym. Ciągłe odwzorowanie $f\colon |K|\to \mathbb{R}$ nazywamy \textit{funkcją z~minimami na gwiazdach}, o~ile dla każdego sympleksu $\sigma$~kompleksu $K$~funkcja $f\big|_{|\st_K(\sigma)|}\colon |\st_K(\sigma)|\to \mathbb{R}$ ma globalne minimum, osiągane w~dokładnie jednym punkcie.

Ustalmy dowolny liniowy porządek $\unlhd$ na zbiorze $V(K)$~wierzchołków kompleksu $K$~o~tej własności, że jeśli $f(v)<f(w)$ dla $v,w\in V(K)$, to $v\lhd w$. Określmy funkcję $y_f\colon \mP(K)\to V(K)$. Dla sympleksu $\sigma\in \mP(K)$ istnieje na zbiorze $|\st_K(\sigma)|$ jedyne minimum globalne $m_f(\sigma)\in |\st_K(\sigma)|$. Niech $\mu_f(\sigma)$~będzie minimalnym (w~sensie inkluzji) sympleksem $K$, którego realizacja $|\mu_f(\sigma)|$ zawiera punkt $m_f(\sigma)$. Za $y_f(\sigma)$ przyjmujemy minimalny w~porządku $\unlhd$~wierzchołek sympleksu $\mu_f(\sigma)$.

Niech $f\colon |K|\to \mathbb{R}$ będzie funkcją z~minimami na gwiazdach. Określimy indukcyjnie pewną rodzinę par uporządkowanych $M_f\subseteq \mP(K)\times \mP(K)$ stowarzyszoną z~$f$. Ustalmy $n\geq 0$ i~załóżmy, że zbiory $M_k\subseteq \mP(K)\times \mP(K)$~są określone dla wszystkich liczb naturalnych $k<n$. Niech $S_n$~będzie zbiorem tych sympleksów $\sigma\in \mP(K)$ o~wymiarze $n$, które nie są elementami żadnej pary uporządkowanej należącej do zbioru $\bigcup_{k<n}M_k$; definiujemy $M_n=\{(\sigma\cup y_f(\sigma), \sigma):\sigma\in S_n, y_f(\sigma)\not\in \sigma\}$. Rodzinę $M_f$~zadajemy jako sumę $M_f=\bigcup_{n\in\mN}M_n$.
\begin{tw}[{\cite[Theorem 3.1.3]{Adiprasito13}}]
Niech $K$~będzie kompleksem symplicjalnym, zaś $f\colon |K|\to \mathbb{R}$ funkcją z~minimami na gwiazdach. Wówczas $M_f$~jest skojarzeniem Morse'a na $K$, a~odwzorowanie \begin{align*}
\mcC_{M_f}(\mP(K)) \quad &\longrightarrow \quad \left\{(v,\tau)\in V(K)\times \mP(K):v\in \tau,\ y_f(\tau)=v,\ y_f(\tau\smallsetminus\{v\})\not=v\right\}\\
\sigma \quad &\longmapsto \quad \left(y_f(\sigma),\sigma\right)
\end{align*}
jest bijekcją.
\end{tw}
Skojarzenie Morse'a $M_f$ nazywamy \textit{skojarzeniem gradientowym} stowarzyszonym z~funkcją $f$.


\begin{tw}[por.~{\cite[Theorem 3.2.1]{Adiprasito13}}]\label{tw-cat0-sa-zgniatalne}
Ustalmy liczbę $\varkappa\in\mathbb{R}$. Niech $\mathbb{K}=\left(K,\ \Shapes(\mathbb{K}),\ \{f_\sigma\}_{\sigma\in K}\right)$~będzie kompleksem $M_\varkappa$-symplicjalnym spełniającym założenia twierdzenia \ref{tw-bridsona}. Jeżeli spełnione są następujące warunki:
\begin{compactitem}
\item[---] $(|K|,d_{\mathbb{K}})$ jest przestrzenią $\CAT(\kappa)$ dla pewnej liczby $\kappa\in\mathbb{R}$;
\item[---] $\kappa\leq 0$ lub $\kappa> 0$ oraz~$\operatorname{diam}(\mathbb{K})\leq \frac{\pi}{2\sqrt{\kappa}}$;
\item[---] dla każdego sympleksu $\sigma\in K$~przestrzeń $|\st_K(\sigma)|\subseteq |K|$ z~metryką indukowaną z~$(|K|,d_\mathbb{K})$~jest wypukła,
\end{compactitem}
to kompleks symplicjalny $K$~jest $\infty$-zgniatalny.
\end{tw}
\begin{proof}
Ustalmy wierzchołek $w\in K$; utożsamiamy go z~jego realizacją geometryczną w~$|K|$. Niech $f\colon |K|\to \mathbb{R}$ będzie funkcją zadaną dla $x\in |K|$ wzorem $f(x)=d_{\mathbb{K}}(x,w)$.
Zauważmy, że dla każdego sympleksu $\sigma\in K$ przestrzeń $|\st_K(\sigma)|$ jest zupełna, jako domknięty podzbiór zupełnej (patrz twierdzenie \ref{tw-bridsona}) przestrzeni $(|K|,d_{\mathbb{K}})$.  Na podstawie lematu \ref{lemat-bridsona-o-jedynym-punkcie} odwzorowanie $f$~jest zatem funkcją z~minimami na gwiazdach (dla pewnego zgodnego z~nią liniowego porządku na zbiorze wierzchołków kompleksu $K$). Niech $M_f$~będzie skojarzeniem gradientowym stowarzyszonym z~$f$.

Oczywiście $\{w\}\in \mcC_{M_f}(\mP(K))$. Wykażemy, że zbiór $\mcC_{M_f}(\mP(K))$ jest jednoelementowy. Ustalmy wierzchołek $v\in K$. Zauważmy, że jeśli $v\not=w$, to $m_f(\{v\})\in |\lk_K(v)|$, a~zatem $y_f(\{v\})\not=v$ i~w~konsekwencji $\left(\{v,y_f(\{v\})\},\{v\}\right)\in M_f$, czyli $\{v\}\not\in \mcC_{M_f}(\mP(K))$.

Przypuśćmy, że istnieje $\sigma\in \mcC_{M_f}(\mP(K))$, $\dim(\sigma)\geq 1$. Niech $\tau\in \st_{K}(\sigma\smallsetminus\{y_f(\sigma)\})$~będzie dowolnym sympleksem zawierającym $\mu_f(\sigma)$. Oczywiście $y_f(\sigma)\in \tau$. 
\end{proof}


\begin{wn}[por.~{\cite[Corollaries 3.2.3, 3.2.4]{Adiprasito13}}]
Jeżeli $\mathbb{K}=\left(K,\ \Shapes(\mathbb{K}),\ \{f_\sigma\}_{\sigma\in K}\right)$ jest regularnym kompleksem $M_0$-symplicjalnym spełniającym założenia twierdzenia \ref{tw-bridsona} oraz $(|K|,d_{\mathbb{K}})$ jest przestrzenią $\CAT(0)$, to kompleks symplicjalny $K$~jest $\infty$-zgniatalny. 
\end{wn}

Zastanawiające jest, które z~pozostałych wyników pracy Adiprasito i~Benedettiego \cite{Adiprasito13} można przenieść na niezwarte obiekty.
\end{comment}

% ============================================================================================================================
% ============================================================================================================================
% ============================================================================================================================


\section{Dyskretna teoria Morse'a a~topologia w~nieskończoności}\label{topologia-w-nsk}

W~tym podrozdziale przedstawiamy wyrażające się w~języku dyskretnej teorii Morse'a kryteria pozwalające na stwierdzenie, że dany lokalnie skończony, regularny CW kompleks ma kołnierzyk do wewnątrz lub ma kołnierzyk na zewnątrz.

\subsection{Kołnierzyki do wewnątrz}
\begin{stw}\label{stw-dyskretna-teoria-morsea-oswojone-do-wewnatrz}
Niech $X$~będzie spójnym, regularnym, lokalnie skończonym CW kompleksem z~zadanym dyskretnym skojarzeniem Morse'a $M$~bez promieni malejących i~takim, że zbiór $\mcC_M(\mP(X))$ jest skończony. Wówczas $X$~ma kołnierzyk do wewnątrz.
\end{stw}
\begin{proof}
Stosując lemat \ref{lem-351} do częściowego porządku $P=\mP(X)$~oraz funkcji $\rho=\Ht\colon \mP(X)\to \mN$ dla każdego $x\in\mP(X)$~otrzymujemy skończony zbiór $O(x)\subseteq \mP(X)$~o~własnościach opisanych w~tym lemacie.

Zbiór komórek CW kompleksu $X$~jest przeliczalny. Ustawmy go w~ciąg $(x_n)_{n\in\mN}$. Dla $n\in\mN$ niech $X_n$ będzie podzbiorem $X$~będącym sumą komórek należących do zbioru $\bigcup_{m\leq n} O(x_m)$; warunek \ref{351-2}) z~lematu \ref{lem-351} gwarantuje, że $X_n$~jest podkompleksem $X$, zaś dzięki warunkowi \ref{351-1}) zachodzi równość $\bigcup_{n\in\mN} X_n=X$.

Rodzina krawędzi $M_n=\{(x,y)\in M: x,y\in \mP(X_n)\smallsetminus \mP(X_{n-1})\}$ jest skojarzeniem Morse'a na $X_n$. Ponieważ zbiór $\mcC_M(\mP(X))$ jest skończony, korzystając z~warunku \ref{351-3}) otrzymujemy dla odpowiednio dużych $n\in\mN$ równość $\mcC_{M_n}(\mP(X_n))=\mP(X_{n-1})$, tzn.~$X_n\infcoll X_{n-1}$. Na podstawie lematu \ref{lem-ciag_wstepujacy_zgniecen}, dla wystarczająco dużych $n\in\mN$, mamy $X\infcoll X_n$, a~zatem wobec lematu \ref{lem-niesk_zgniatalnosc_mocny_retrakt_deformacyjny} włożenia $X_n\hookrightarrow X$ są (dla dostatecznie dużych $n\in\mN$) homotopijnymi równoważnościami. Zgodnie z~twierdzeniem \ref{tw-charakteryzacja_kolnierzyka_do_wewnatrz} oznacza to, że przestrzeń $X$~ma kołnierzyk do wewnątrz.
\end{proof}

Zauważmy, że na podstawie~stwierdzenia \ref{stw-dyskretna-teoria-morsea-oswojone-do-wewnatrz} każdy regularny, lokalnie skończony CW kompleks $X$, który jest $\infty$-zgniatalny do swojego zwartego podkompleksu, ma kołnierzyk do wewnątrz.

Ze~stwierdzeń \ref{reversing_stw}, \ref{stw-dyskretna-teoria-morsea-oswojone-do-wewnatrz} otrzymujemy ponadto następujący wniosek.
\begin{wn}
Jeśli na regularnym, lokalnie skończonym CW kompleksie $X$~zadane jest skojarzenie Morse'a $M$~takie, że zbiory $\mcC_M(\mP(X))$ oraz $\mcR_M(\mP(X))$ są skończone, to przestrzeń $X$~ma kołnierzyk do wewnątrz.
\end{wn}

\subsection{Kołnierzyki na zewnątrz}
Zanim przystąpimy do dowodu analogicznego do stwierdzenia \ref{stw-dyskretna-teoria-morsea-oswojone-do-wewnatrz} wyniku  dotyczącego~przestrzeni z~kołnierzykami na zewnątrz, wprowadźmy pomocnicze oznaczenia. Niech $P$~będzie częściowym porządkiem z~zadanym skojarzeniem $M$~w~diagramie Hassego $\mH(P)$. Dla podzbioru $A\subseteq P$ przez $M_-(A)$ oznaczmy zbiór tych elementów $p\in P$, dla których istnieje ścieżka w~$\mH_M(P)$ zaczynająca się w~$p$~i~kończąca w~którymś z~elementów zbioru $A$. Przyjmijmy ponadto \[M^*_-(A)=M_-(A)\cup \{x\in P:(x,y)\in M \text{ dla pewnego }y\in M_-(A)\}.\]

\begin{lem}\label{lem-dopelnienie_M_minus_gwiazdka_tworzy_podkompleks}
Niech $X$~będzie regularnym, lokalnie skończonym CW kompleksem z~zadanym skojarzeniem Morse'a $M$ takim, że $\mH_M(\mP(X))$ nie zawiera promieni rosnących. Wówczas dla dowolnego skończonego zbioru $A\subseteq \mP(X)$ zbiór $M_-^*(A)$ jest skończony, zaś komórki kompleksu $X$~należące do $\mP(X)\smallsetminus M_-^*(A)$ tworzą koograniczony podkompleks CW kompleksu~$X$.
\end{lem}
\begin{proof}
Ustalmy skończony zbiór $A\subseteq \mP(X)$. Przypuśćmy, że zbiór $M_-(A)$ jest nieskończony. Ponieważ zbiór $A$~jest skończony oraz $M_{-}(A)=\bigcup_{a\in A}M_{-}(\{a\})$, istnieje $a\in A$~takie, że zbiór $M_{-}(\{a\})$ jest nieskończony. Rozważmy podgraf $D$~diagramu Hassego $\mH_M(\mP(X))$ indukowany na zbiorze $M_-(\{a\})$. Niech $D^{d}$~oznacza graf skierowany powstały z~$D$~przez zmianę orientacji wszystkich krawędzi. Ponieważ porządek $\mP(X)$~jest lokalnie skończony, to lokalnie skończony jest też graf $D^{d}$. Z~lematu K\"oniga \ref{konig} wynika, że $D^{d}$ zawiera nieskończoną ścieżkę prostą, czyli $D$~zawiera promień rosnący, co jest sprzeczne z~założeniem o~braku promieni rosnących w~$\mH_M(\mP(X))$. Wobec tego zbiór $M_-(A)$ jest skończony, więc skończony jest też zbiór $M_-^*(A)$.

Wykażemy, że elementy zbioru $\mP(X)\smallsetminus M_-^*(A)$ tworzą podkompleks kompleksu $X$, tzn.~że $x\mathord{\downarrow}_{\mP(X)}\subseteq \mP(X)\smallsetminus M_-^*(A)$ dla każdego $x\in \mP(X)\smallsetminus M_-^*(A)$. W~tym celu wystarczy udowodnić, że dla wszystkich $x\in \mP(X)\smallsetminus M_-^*(A)$ oraz $y\in\mP(X)$ takich, że $y\prec x$, element $y\not\in M_-^*(A)$. Ustalmy $x\in \mP(X)\smallsetminus M_-^*(A)$ oraz $y\in\mP(X)$, $y\prec x$. Przypuśćmy, że $y\in M_-^*(A)$. Jeżeli $(x,y)\in M$, to ponieważ $y$~nie należy do żadnej innej niż $(x,y)$ krawędzi z~$M$, mamy $y\in M_-(A)$. Ale to oznacza, że $x\in M_-^*(A)$, co jest sprzeczne z~wyborem $x$. Zatem $(x,y)\not\in M$. Gdyby $y\in M_-(A)$, to wtedy również $x\in M_-(A)$, co wykluczyliśmy. Wobec tego $y\in M_-^*(A)\smallsetminus M_-(A)$, czyli istnieje $z\in M_-(A)$ takie, że $(y,z)\in M$. Mamy $z\prec y\prec x$, więc na podstawie lematu \ref{lem-miedzy_komorkami_leza_dwie_komorki} istnieje $y'\in\mP(X)\smallsetminus\{y\}$ o~tej własności, że $z\prec y'\prec x$. Ponieważ $(y,z)\in M$, mamy $(y',z)\not\in M$, zatem $y'\in M_-(A)$. Ale $y'\prec x$. Jak zauważyliśmy przed chwilą, taka sytuacja jest niemożliwa. Otrzymaliśmy sprzeczność; wobec tego $y\not\in M_{-}^*(A)$.
\end{proof}

\begin{lem}\label{lem-374}
Niech $X$~będzie regularnym CW~kompleksem z~zadanym skojarzeniem Morse'a $M$ takim, że $\mcC_M(\mP(X))=\emptyset$. Załóżmy, że $y\in X$ oraz $M_{-}(\{y\})=\{y\}$. Wówczas istnieje $x\in \mP(X)$ takie, że $(x,y)\in M$, a~ponadto $\hat{y}\mathord{\uparrow}_{\mP(X)}=\{x\}$, tzn.~istnieje podkompleks $Y\subseteq X$ o~tej własności, że $\mP(Y)=\mP(X)\smallsetminus\{x,y\}$ oraz $X\elcoll Y$.
\end{lem}
\begin{proof}
Ponieważ $\mcC_M(\mP(X))=\emptyset$, element $y$~nie jest krytyczny względem $M$. Zatem istnieje $x\in\mP(X)$ takie, że $(x,y)\in M$ lub $(y,x)\in M$. Ponieważ $M_{-}(\{y\})=\{y\}$, druga z~tych możliwości jest wykluczona; zatem $(x,y)\in M$.

Ustalmy $a\in \hat{y}\mathord{\uparrow}_{\mP(X)}$. Przypuśćmy, że $a\not\geq x$. Istnieje wobec tego \mbox{$b\in \mP(X)\smallsetminus \{x\}$} takie, że $a>b\succ y$. Ponieważ $(x,y)\in M$ oraz $M$~jest skojarzeniem, krawędź $(b,y)\not\in M$, więc $b\in M_{-}(\{y\})$, co jest sprzeczne z~założeniem. Zatem $a\geq x$.

Przypuśćmy, że $a>x$. Istnieje $c\in \mP(X)$ takie, że $a> c\succ x$. Na podstawie lematu \ref{lem-miedzy_komorkami_leza_dwie_komorki} istnieje $d\in\mP(X)\smallsetminus\{x\}$ o~tej własności, że $c\succ d\succ y$. To jednak, jak zauważyliśmy, jest niemożliwe. Wobec tego $a=x$, czyli $\hat{y}\mathord{\uparrow}_{\mP(X)}=\{x\}$.

Ostatnia część tezy wynika z~definicji elementarnego zgniecenia.
\end{proof}


\begin{stw}\label{stw-dyskretna-teoria-morsea-oswojone-na-zewnatrz}
Niech $X$~będzie regularnym, lokalnie skończonym CW kompleksem z~zadanym skojarzeniem Morse'a $M$~takim, że graf skierowany $\mH_M(\mP(X))$ nie zawiera promieni rosnących oraz zbiór $\mcC_M(\mP(X))$ jest skończony. Wówczas $X$~ma kołnierzyk na zewnątrz.
\end{stw}
\begin{proof}
Skonstruujemy pewien zstępujący ciąg $(V_n)_{n\in\mN}$ podkompleksów kompleksu $X$~oraz ciągi: skojarzeń Morse'a $\bigl(M_n\subseteq \mH(\mP(V_n))\bigr)_{n\in\mN}$, retrakcji deformacyjnych $(r_n\colon V_n\to V_{n+1})_{n\in\mN}$ i~homotopii $(h_{n}\colon V_{n}\times \I\to V_{n})_{n\in\mN}$.

Niech $V_0$~oznacza koograniczony podkompleks kompleksu $X$~składający się z~komórek należących do zbioru $\mP(X)\smallsetminus M_-^*(\mcC_M(\mP(X)))$ (patrz lemat \ref{lem-dopelnienie_M_minus_gwiazdka_tworzy_podkompleks}) oraz niech \[M_0=\{(x,y)\in M:x,y\in\mP(V_0)\}.\] Zauważmy, że $M_0$~jest skojarzeniem Morse'a na $V_0$, $\mH_{M_0}(\mP(V_0))$ nie zawiera promieni rosnących~oraz $\mcC_{M_0}(\mP(V_0))=\emptyset$, bowiem komórki krytyczne kompleksu $X$~względem skojarzenia $M$~leżą poza $V_0$, a~wobec własności zbioru $M_-^*(\mcC_M(\mP(X)))$ dla każdej krawędzi $(x,y)\in M$ albo oba jej elementy należą do~$V_0$, albo do $V_0$~nie należy żaden z~nich.

Ustalmy $n\geq 0$ i~załóżmy, że określiliśmy $V_n$~oraz skojarzenie Morse'a $M_n$~na $V_n$~takie, że $\mH_{M_n}(\mP(V_n))$ nie zawiera promieni rosnących~oraz $\mcC_{M_n}(\mP(V_n))=\emptyset$. Rozważmy zbiór \[A_n=\bigl\{y\in\mP(V_n):{M_n}_-(\{y\})=\{y\}\bigr\}.\] Na podstawie lematu \ref{lem-374} dla każdego $y\in A_n$ istnieje  komórka $x_y\in \mP(V_n)$ o~tej własności, że $(x,y)\in M_n$ oraz zachodzi elementarne zgniecenie $V_n\elcoll V_n(y)$, gdzie $V_n(y)$~jest podkompleksem $V_n$~takim, że $\mP(V_n(y))=\mP(V_n)\smallsetminus\left\{x_y,y\right\}$. Niech $r(y)\colon V_n\to V_n(y)$ oznacza mocną retrakcję deformacyjną wybraną w~ten sposób (patrz s.~\pageref{mocna_retr_def_przy_el_zgnieceniu_ze_jest_fajna}), że \[r(y)\left(x_y\cup y\right)\subseteq \bigcup\{z\in \mP(V_n(y)):z \text{ jest ścianą } x_y \text{ w } V_n\}.\] Przez $V_{n+1}$ oznaczmy taki podkompleks CW kompleksu $V_n$, że $\mP(V_{n+1})=\mP(V_n)\smallsetminus\bigcup_{y\in A_n}\{y,x_y\}$, przez $i_{n}\colon V_{n+1}\hookrightarrow V_n$ włożenie, zaś przez $r_n\colon V_{n}\to V_{n+1}$ mocną retrakcją deformacyjną zadaną, dla $a\in V_n$, wzorem \[r_n(a)=\begin{cases}r(y)(a), & \text{jeżeli } a\in x_y\cup y \text{ dla pewnego } y\in A_n,\\ a & \text{w przeciwnym wypadku.}\end{cases}\] Wobec wyboru retrakcji $r(y)$, $y\in A_n$, retrakcja $r_n$~jest właściwa. Istnieje właściwa homotopia $\id_{V_{n-1}}\stackrel{p}{\simeq} i_{n}\circ r_n \operatorname{rel} V_n$; oznaczmy ją przez $h_{n}\colon V_{n}\times \I\to V_{n}$. Niech \[M_{n+1}=\left\{(x,y)\in M_n:x,y\in\mP(V_{n+1})\right\}.\] Nietrudno spostrzec, że $M_{n+1}$ jest skojarzeniem Morse'a na $V_{n+1}$ o~tej własności, że $\mH_{M_{n+1}}(\mP(V_{n+1}))$ nie zawiera promieni rosnących oraz $\mcC_{M_{n+1}}(\mP(V_{n+1}))=\emptyset$.

Wykażemy, że $\bigcap_{n\in\mN} V_n=\emptyset$. Dla $n\in\mN$ oraz $y\in \mP(V_n)$ niech $l_n(y)$ oznacza maksymalną długość ścieżki w~grafie skierowanym ${M_{n}}_{-}(\{y\})$. Łatwo zauważyć, że jeśli $y\in \mP(V_{n+1})$, to $l_{n+1}(y)<l_n(y)$. Ponadto $l_n(y)=0$ wtedy i~tylko wtedy, gdy ${M_{n}}_{-}(\{y\})=\{y\}$. Zatem dla każdego $y\in \mP(V_0)$ istnieje $n\in \mN$ takie, że $y\in \mP(V_n)\smallsetminus \mP(V_{n+1})$. 

Zdefiniujemy pewien ciąg odwzorowań $(H_n\colon V_0\times [n,n+1]\to V_0)_{n\in\mN}$. Niech $H_0=h_0\colon V_0\times [0,1]\to V_0$. Załóżmy, że określiliśmy $H_{n-1}\colon V_0\times [n-1,n]\to V_0$ dla pewnego $n\in\mN$. Niech odwzorowanie $H_n\colon V_0\times [n,n+1]\to V_0$ będzie dla $v\in V_0$, $t\in [n,n+1]$ zadane wzorem $H_n(v,t)=h_{n}(H_{n-1}(v,n),t-n)$. Funkcja $H\colon V_0\times [0,\infty)\to V_0$ taka, że $H(v,t)=H_n(v,t)$ dla $v\in V_0$, $t\in [n,n+1]$, jest żądanym w~definicji przestrzeni z~kołnierzykiem na zewnątrz właściwym przedłużeniem odwzorowania $V_0\times\{0\}\to V_0$.  
\end{proof}

\begin{comment}
Dzięki twierdzeniu \ref{reversing_tw2} oczywisty jest następujący wniosek.
\begin{wn}
Jeśli na regularnym, lokalnie skończonym CW kompleksie $X$~zadane jest skojarzenie Morse'a $M$~takie, że zbiory $\mcC_M(\mP(X))$, \reflectbox{$\mcR$}$_M(\mP(X))$ są skończone, to przestrzeń $X$~ma kołnierzyk na zewnątrz.
\end{wn}
\end{comment}
\begin{comment}
\subsection{Dyskretne nierówności Morse'a w nieskończoności}
(do napisania)

Czy da się tym ograniczać liczby Bettiego w nieskończoności? (Choćby nawet przy założeniu, że wszystkie promienie w jedną stronę (a przynajmniej w danym końcu w jedną stronę -- może można założyć sobie po prostu, że mamy przestrzeń z jednym końcem. Wszak jeśli środek nas nie interesuje, to niewiele to zmienia.) Może zrobić na to osobną sekcję -- DMT a topologia w nieskończoności? Jak jest dobra przestrzeń, to brzeg Waldhausena jest równoważny odpowiedniemu collarowi. Mamy homologie collara, to mamy też brzegu. Czy można ogólniej? Odwracanie promieni -- przy skonczenie wielu promieniach mamy zarowno forward jak i reverse tame 
\end{comment}
% ============================================================================================================================
% ============================================================================================================================
% ============================================================================================================================


\section{Skojarzenia Morse'a a dyskretne funkcje Morse'a}\label{skojarzenia-a-funkcje}
Uogólniając dyskretną teorię Morse'a na nieskończone kompleksy i~częściowe porządki nie korzystaliśmy, przynajmniej jawnie, z~pojęcia dyskretnej funkcji Morse'a. Powodem tego stanu rzeczy jest fakt, iż skojarzenia Morse'a okazały się dla autora rozprawy podczas formułowania przedstawionych uogólnień narzędziem wygodniejszym i~bardziej elastycznym. Garść uwag dotyczących dyskretnych funkcji Morse'a na niezwartych (nieskończonych) obiektach wydaje się jednak być w~tym miejscu stosowna.

\subsection{Przegląd literatury}
Podejście do dyskretnej teorii Morse'a na nieskończonych kompleksach symplicjalnych bazujące na dyskretnych funkcjach Morse'a zostało zaproponowane przez Formana \cite{Forman02}. Zdefiniował on \textit{właściwą dyskretną funkcję Morse'a}\index{dyskretna funkcja Morse'a!wlzzzaszzzciwa@właściwa} na kompleksie symplicjalnym $K$~jako dyskretną funkcję Morse'a na~$K$~o~tej własności, że podkompleksy $K(a)$ są skończone dla wszystkich $a\in\mR$. Dla tego typu dyskretnych funkcji Morse'a można przeprowadzić dowody najważniejszych twierdzeń dyskretnej teorii Morse'a analogicznie jak w~przypadku skończonym. Jednakże, jak zauważył Forman, w~wielu przypadkach założenie to jest nienaturalne.

Pojęcie właściwej dyskretnej funkcji Morse'a uogólnili Ayala, Fern{\'a}ndez i~Vilches \cite{Ayala07}; autorzy ci zdefiniowali właściwą dyskretną funkcję Morse'a na kompleksie symplicjalnym $K$~jako taką dyskretną funkcję Morse'a na~$K$, dla której zbiory $f^{-1}[a,b]$ są skończone dla wszystkich $a<b\in\mR$. Mówimy, że funkcje tego typu są \textit{właściwe w~słabym sensie}\index{dyskretna funkcja Morse'a!wlzzzaszzzciwa w slzzzabym sensie@właściwa w~słabym sensie}. Dla dyskretnej funkcji Morse'a właściwej w~słabym sensie można wykazać, w~zasadzie przepisując dowody z~pracy Formana \cite{Forman98}, że jeżeli zbiór $f^{-1}[a,b]$ nie zawiera komórek krytycznych, to podkompleksy $K(a),K(b)$ są homotopijnie równoważne, a~także że ,,przejście'' przez poziom, którego przeciwobraz zawiera komórkę krytyczną, odpowiada doklejeniu komórki. Ponadto, w~specjalnych wypadkach, udowodniono dla dyskretnych funkcji Morse'a właściwych w~słabym sensie dyskretne nierówności Morse'a \cite{Ayala07,Ayala09}, oraz innego typu wyniki \cite{Ayala08,Ayala09a,Ayala10,Ayala11}.

\subsection{Uogólnienia pojęcia dyskretnej funkcji Morse'a}
Niech $P$~będzie dobrze ufundowanym częściowym porządkiem nie zawierającym podzbioru izomorficznego z~$\omega+ 1$. Funkcję $f\colon P\to \mathbb{R}$ nazywamy \textit{dyskretną funkcją Morse'a}\index{dyskretna funkcja Morse'a!na dobrze ufundowanym czezzzszzzciowym porzazzzdku@na dobrze ufundowanym częściowym porządku}, jeżeli dla każdego $p\in P$ poniższe zbiory są co najwyżej jednoelementowe oraz przynajmniej jeden z~nich jest pusty:
\begin{align*}
&d_f(p)=\{q\prec p:f(p)\leq f(q)\},\\
&u_f(p)=\{q\succ p:f(p)\geq f(q)\}.
\end{align*}
Element $p\in P$ nazywamy \textit{krytycznym}\index{element!krytyczny|see{krytyczność}}\index{krytycznoszzzczzz wzglezzzdem@krytyczność względem!dyskretnej funkcji Morse'a!navczezzzszzzciowym porzazzzdku@na częściowym porządku} względem dyskretnej funkcji Morse'a $f$, jeżeli $u_f(p)=\emptyset=d_f(p)$. 

Oczywiście, jeśli $P=\mP(X)$ dla pewnego regularnego CW kompleksu $X$, to funkcja $f$~spełnia powyższe warunki wtedy i~tylko wtedy, gdy jest dyskretną funkcją Morse'a na $X$~w klasycznym sensie. 

Nietrudno zauważyć, że dla tych porządków $P$, dla których prawdziwy jest odpowiednik lematu \ref{lem-miedzy_komorkami_leza_dwie_komorki}, wystarczy w~definicji dyskretnej funkcji Morse'a zakładać, że zbiory $d_f(p),u_f(p)$ są co najwyżej jednoelementowe dla wszystkich $p\in P$; jeden z~nich musi wówczas być pusty.

Jeśli $f\colon P\to \mathbb{R}$ jest dyskretną funkcją Morse'a, to \textit{skojarzeniem Morse'a na $P$~indukowanym przez $f$}\index{skojarzenie Morse'a!indukowane przez dyskretnazzz funkcjezzz Morse'a@indukowane przez dyskretną funkcję Morse'a!na czezzzsciowym porzazzzdku@na częściowym porządku}~nazywamy skojarzenie $M=\left\{(p,q):q\in d_f(p)\right\}$ w~grafie $\mH(P)$. (Zauważmy, że elementy krytyczne względem tego skojarzenia pokrywają się z~elementami krytycznymi względem funkcji $f$.)

Mówimy, że dyskretna funkcja Morse'a $f$~na częściowym porządku z~rangą $P$~jest \textit{samoindeksująca}\index{dyskretna funkcja Morse'a!samoindeksujazzzca@samoindeksująca}, jeżeli $f(p)=\Ht(p)$ dla każdego elementu $p\in P$  krytycznego względem tej funkcji.

\begin{lem}\label{lem-skonczonosc_zbioru_koncow_wychodzacych_z_wierzcholka_sciezek}
Niech $P$~będzie częściowym porządkiem z~gradacją, o~skończonych ideałach głównych oraz niech $M$~będzie skojarzeniem Morse'a na $P$~bez promieni malejących. Dla każdego $x\in P$ zbiór tych $y\in P$, dla których istnieje skierowana w~$\mH_M(P)$~prowadząca z~$x$~do~$y$~jest skończony.
\end{lem}
\begin{proof}
Ustalmy $x\in P$; niech \[D(x)=\left\{y\in P:\text{istnieje ścieżka w } \mH_M(P)\text{ prowadząca z } x\text{ do } y\right\}.\]  Zastosujmy lemat \ref{lem-351} do zbioru $P$, punktu $x$~oraz~skojarzenia $M$, przyjmując \mbox{$\rho=\rk$}; oczywiście $D(x)\subseteq O(x)$, gdzie $O(x)$ jest skończonym zbiorem uzyskanym z~lematu \ref{lem-351}.
\begin{comment}
 i~rozważmy podgraf $D(x)\subseteq \mH_M(P)$ indukowany na zbiorze tych $y\in P$, dla których istnieje ścieżka skierowana w~$\mH_M(P)$~prowadząca z~$x$~do~$y$. Udowodnimy, że graf $D(x)$ jest skończony. Rozważmy drzewo $T$~rozpinające ten graf. Liczba krawędzi wychodzących z~każdego wierzchołka grafu $D(x)$ jest skończona, gdyż $P$~ma skończone ideały główne oraz każdy taki wierzchołek znajduje się w~co najwyżej jednej krawędzi należącej do $M$. Zatem drzewo $T$~jest lokalnie skończone. Ponieważ graf $\mH_M(P)$ nie zawiera promieni malejących, $T$~jest, wobec lematu K\"oniga \ref{konig}, drzewem skończonym. Skończony jest zatem również graf $D(X)$.
\end{comment}
\end{proof}

\begin{stw}\label{stw-istnieje_dmf_indukujaca_dane_skojarzenie}
Niech $P$~będzie częściowym porządkiem z~gradacją, o~skończonych ideałach głównych oraz niech $M$~będzie skojarzeniem Morse'a na $P$~bez promieni malejących. Istnieje wówczas samoindeksująca dyskretna funkcja Morse'a $f\colon P\to \mathbb{R}$ taka, że~$M$~jest skojarzeniem Morse'a indukowanym przez $f$.
\end{stw}
\begin{proof}
Niech $P_0=\left\{x\in \mcC_M(P):\Ht(x)=0\right\}$. Definiujemy indukcyjnie dla $n\geq 0$: \begin{align*}P_n^*&=P_n\cup \bigcup\big\{\{x,u(x)\}: x\in P,\ \Ht(x)=n \text{ oraz } (u(x),x)\in M\big\},\\
P_{n+1}&=P_n^*\cup\left\{x\in \mcC_M(P):\Ht(x)=n+1\right\}.\end{align*}

Dla elementu $x\in P^*_n\smallsetminus P_n$ niech $F(x)$ oznacza podgraf grafu skierowanego $\mH_M(P)$ indukowany na zbiorze tych elementów $y\in P$, $\Ht(y)\in\{n,n+1\}$, dla których istnieje ścieżka w~$\mH_M(P)$ prowadząca z~$x$~do $y$. Na podstawie lematu \ref{lem-skonczonosc_zbioru_koncow_wychodzacych_z_wierzcholka_sciezek} graf $F(x)$ jest skończony. Przez $L(x)$ oznaczmy maksymalną długość ścieżki prostej w~grafie skierowanym $F(x)$. (Oczywiście ścieżka o~maksymalnej długości musi zaczynać się w~$x$.)

Zdefiniujemy dyskretną funkcję Morse'a $f\colon P\to \mathbb{R}$. Dla elementów krytycznych $x\in P$ przyjmijmy $f(x)=\Ht(x)$. Każdy element $x\in P$, który nie jest krytyczny, należy do zbioru postaci $P^*_n\smallsetminus P_n$ dla pewnego $n\in\mN$. Jeżeli $\Ht(x)=n$, niech $f(x)=n+\left(1-\frac{1}{2^{L(x)}}\right)$. Jeśli natomiast $\Ht(x)=n+1$, to $x=u(y)$ dla pewnego $y\in P^*_n\smallsetminus P_n$ takiego, że $\Ht(y)=n$. Przyjmujemy wówczas $f(x)=n+\left(1-\frac{1}{2^{L(y)}}\right)$.

Czytelnikowi pozostawiamy sprawdzenie, że otrzymana funkcja jest samoindeksującą dyskretną funkcją Morse'a na $P$~indukującą skojarzenie $M$.
\end{proof}

Otrzymana w~dowodzie stwierdzenia \ref{stw-istnieje_dmf_indukujaca_dane_skojarzenie} dyskretna funkcja Morse'a nie jest na ogół właściwa (nawet w~słabym sensie), choćby z~tego powodu, że może istnieć nieskończenie wiele elementów krytycznych ustalonej rangi. Charakteryzację tych skojarzeń Morse'a na lokalnie skończonych kompleksach symplicjalnych, które są indukowane przez właściwe w~słabym sensie dyskretne funkcje Morse'a, podali (przy założeniu o~skończoności zbioru klas równoważności promieni malejących oraz zbioru komórek krytycznych) Ayala, Vilches, Jer{\v{s}}e i~Kosta \cite{Ayala11}. W~ogólności charakteryzacja takich skojarzeń nie jest znana.

W~dowodach Formana wykorzystywany jest często fakt, że dyskretną funkcję Morse'a można zastąpić różnowartościową dyskretną funkcją Morse'a indukującą to samo skojarzenie. Jednakże w~przypadku dyskretnych funkcji Morse'a na nieskończonych CW kompleksach operacja taka nie zawsze jest możliwa: przykładowo, jeśli rozważany kompleks składa się z~ponad $2^{\aleph_0}$ komórek, to nie istnieje na nim żadna różnowartościowa dyskretna funkcja Morse'a. Jako środek zaradczy na ten problem proponujemy rozważanie dyskretnych funkcji Morse'a o~wartościach w~innych niż $\mR$~zbiorach uporządkowanych. 

Niech $L$~będzie liniowym porządkiem. \textit{Uogólnioną dyskretną funkcją Morse'a}\index{dyskretna funkcja Morse'a!uogozzzlniona@uogólniona} o~wartościach w~$L$, zadaną~na dobrze ufundowanym częściowym porządku $P$~nie zawierającym podzbioru izomorficznego z~$\omega+ 1$, nazywamy funkcję $f\colon P\to L$ o~tej własności, że dla każdego $p\in P$ poniższe zbiory są co najwyżej jednoelementowe oraz przynajmniej jeden z~nich jest pusty:
\begin{align*}
&d_f(p)=\{q\prec p:f(p)\leq f(q)\},\\
&u_f(p)=\{q\succ p:f(p)\geq f(q)\}.
\end{align*}
Podobnie jak w~przypadku klasycznych dyskretnych funkcji Morse'a definiujemy skojarzenie indukowane przez uogólnioną dyskretną funkcję Morse'a.\index{skojarzenie Morse'a!indukowane przez dyskretnazzz funkcjezzz Morse'a@indukowane przez dyskretną funkcję Morse'a!uogozzzlnionazzz@uogólnioną}

Odpowiednim kandydatem na kodziedzinę uogólnienej dyskretnej funkcji Morse'a w~przypadku, gdy interesują nas jedynie funkcje indukujące skojarzenia Morse'a bez promieni malejących, okazują się być dobre porządki.

\begin{stw}\label{stw-charakteryzacja_skojarzen_bez_promieni_poprzez_funkcje_morsea_i_porzadki}
Niech $P$~będzie dobrze ufundowanym częściowym porządkiem nie zawierającym podzbioru izomorficznego z~$\omega+ 1$. Dla skojarzenia $M$~w~diagramie Hassego $\mH(P)$ następujące warunki są równoważne:
\begin{compactenum}[\quad\ 1)]
\item $M$~jest skojarzeniem Morse'a bez promieni malejących na~$P$;
\item istnieje liniowe rozszerzenie $P^*=(P,\leq^*)$ częściowego porządku $P$~będące dobrym porządkiem o~tej własności, że jeśli $(p,q)\in M$ dla pewnych $p,q\in P$, to $p$~jest pokryciem górnym $q$ w~$P^*$;
\item istnieje indukująca skojarzenie $M$~uogólniona dyskretna funkcja Morse'a na $P$~o~wartościach w~pewnym dobrym porządku;
\item istnieje różnowartościowa, indukująca skojarzenie $M$~uogólniona dyskretna funkcja Morse'a na $P$~o~wartościach w~pewnym dobrym porządku.
\end{compactenum}
\end{stw}
\begin{proof}
1)$\iff$2)\,: O~równoważności warunków 1)~i~2)~mówi lemat \ref{lem-kozlov_lemma}.

2)$\implies$4)\,: Zadajemy funkcję $f\colon P\to P^*$ wzorem 
\[f(p)=\begin{cases}q, & \text{jeżeli istnieje } q\in P \text{ takie, że } (p,q)\in M \text{ lub } (q,p)\in M,\\
p & \text{w przeciwnym wypadku.}
 \end{cases}\]
Oczywiście jest ona różnowartościowa i~wobec własności porządku $P^*$ jest uogólnioną dyskretną funkcją Morse'a indukującą skojarzenie $M$.

4)$\implies$3)\,: Oczywiste.

3)$\implies$1)\,: Niech $f\colon P\to L$ będzie uogólnioną dyskretną funkcją Morse'a o~wartościach w~dobrym porządku $L$, indukującą skojarzenie $M$. Przypuśćmy, że $M$~nie jest skojarzeniem Morse'a bez promieni malejących, czyli że istnieje w~grafie $\mH_M(P)$ nieskończona ścieżka $(p_i)_{i\in\mN}$ (patrz lemat \ref{lem-niesk_sciezka_a_skojarzenie_morse}). Dla każdej liczby $i\in\mN$ krawędź $(p_i,p_{i+1})\in\mH_M(P)$. Oznacza to, że albo $p_{i+1}\succ p_{i}$ w~$P$ oraz $(p_{i+1},p_i)\in M$, albo $p_i\succ p_{i+1}$ w~$P$ oraz $(p_i,p_{i+1})\not\in M$.

W~pierwszym przypadku $f(p_i)\geq f(p_{i+1})$ oraz $f(p_{i+1})>f(p_{i+2})$, gdyż $(p_{i+1},p_{i+2})\not\in M$. Natomiast w~drugim przypadku zachodzą nierówności $f(p_i)>f(p_{i+1})$ oraz $f(p_{i+1})\geq f(p_i)$. 

Ma zatem miejsce nierówność $f(p_i)>f(p_{i+2})$, wobec czego $(f(p_{2k}))_{k\in\mN}$ jest nieskończonym łańcuchem zstępującym w~$L$, co jest sprzeczne z~założeniem, że $L$~jest dobrym porządkiem.
\end{proof}

\begin{problem}\label{prob7}
Scharakteryzować inne klasy skojarzeń Morse'a niż skojarzenia Morse'a bez promieni malejących za pomocą uogólnionych dyskretnych funkcji Morse'a o~odpowiednich kodziedzinach.
\end{problem}

\begin{comment}
Jeżeli natomiast rozważać chcemy uogólnione dyskretne funkcje Morse'a indukujące dowolne skojarzenia Morse'a, to interesującym kandydatem na kodziedzinę dyskretnej funkcji Morse'a wydają się być produkty leksykograficzne postaci $P(\alpha^d,\alpha)$, gdzie $\alpha$ jest dobrym porządkiem. Ma bowiem miejsce następujący fakt.

\begin{stw}
Niech $P$~będzie dobrze ufundowanym częściowym porządkiem nie zawierającym podzbioru izomorficznego z~$\omega\oplus 1$. Niech $M$~będzie skojarzeniem na grafie skierownym $\mH(P)$. Następujące warunki są równoważne.
\begin{compactenum}
\item Skojarzenie $M$~jest skojarzeniem Morse'a na~$P$.
\item Istnieje indukująca skojarzenie $M$~uogólniona dyskretna funkcja Morse'a na $P$~o~wartościach w~pewnym produkcie leksykograficznym $\Lex(\alpha^d,\alpha)$, gdzie $\alpha$ jest dobrym porządkiem.
\item Istnieje indukująca skojarzenie $M$~różnowartościowa uogólniona dyskretna funkcja Morse'a na $P$~o~wartościach w~pewnym produkcie leksykograficznym $\Lex(\alpha^d,\alpha)$, gdzie $\alpha$ jest dobrym porządkiem.
\end{compactenum}
\end{stw}
\begin{proof}

\end{proof}

Dla uogólnionej dyskretnej funkcji Morse'a $f\colon P\to L$ oraz $l\in L$ definiujemy podzbiór częściowo uporządkowany $P_l\subseteq P$ wzorem \[P_l=\bigcup_{f(p)<l}p\mathord{\downarrow}.\] Prawdziwy jest następujący odpowiednik twierdzenia \ref{tw-formana_o_homotopijnej_rownowaznosci_miedzy_poziomami}.

\begin{tw}
Niech $\alpha$~będzie pewnym dobrym porządkiem, zaś $f\colon P\to \Lex(\alpha^d,\alpha)$ uogólnioną dyskretną funkcją Morse'a na h-regularnym częściowym porządku $P$. Jeżeli $l,l'\in \Lex(\alpha^d,\alpha)$ są takie, że  $l<l'$ oraz nie istnieje element krytyczny $p\in P$ taki, że $l< f(p)\leq l'$, to kompleks $\mK(P_l)$ jest mocnym retraktem deformacyjnym kompleksu $\mK(P_{l'})$.
\end{tw}
\begin{proof}

\end{proof}

Odnotujmy jedynie, że możliwe są różnorakie uogólnienia powyższego wyniku, którymi jednak nie chcemy się w~tej chwili zajmować.
\end{comment}

\begin{comment}
PROBLEM GDZIEŚ UMIEŚCIĆ: Another direction of research is understanding and handling the multirays and the situation of infinitely many rays in complexes of dimension greater than $1$. At least in the locally finite case using the methods of locally finite homology, homology at infinity and proper homotopy theory seems to be a promising approach, that in the smooth case has led to some new results \cite{gemmeren}. 
\end{comment}

\newpage\thispagestyle{empty}
